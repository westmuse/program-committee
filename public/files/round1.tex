\documentclass{report}

\title{ Western Museums Association Program Committee: Session Proposals (Round 1)}
\date{ Last updated: April 20, 2022}
\author{Western Museums Association}

\begin{document}
  \pagenumbering{gobble}
  \maketitle
  \newpage
  \tableofcontents
  \newpage
  \pagenumbering{arabic}
  
    \newpage
    \chapter*{ Regular session/panel (roundtable, single speaker, etc.) }

      
        
        
        
        
        
          \newpage
          \section{ National Endowment for the Humanities: What you need to know }
            \begin{description}
              \item [ID:]
              WMA2022\_001

              \item [Assigned to:]Jason Jones~
                \item [Track:]Other~
              \end{description}

              Join Program Officers for a discussion of funding opportunities from the NEH that support interpretive programs, collections, and institutional capacity building. We will provide a brief overview of grant programs followed by tips on navigating the application process and creating competitive proposals. The main focus will be on answering your questions.

              \subsection*{Session Information}
                \begin{description}
                  \item [Format:] Regular session/panel (roundtable, single speaker, etc.)
							    
								  \item [Fee:]None
							     
							    \item [Uniqueness:]As a federal funder, NEH is committed to serving museum professionals in communities across the nation. We are eager to help navigate what we know can be the daunting federal application process.
							    \item [Objectives:]Attendees will learn how grants from the National Endowment for the Humanities can support interpretation and collections preservation efforts. The NEH funds exhibitions, site interpretation, public programs, collections preservation, and capacity building that underlies projects grounded in humanities disciplines, such as history, art history, religious studies, and other fields that critically examine human culture. The session will begin with an overview of the grant opportunities most relevant to museums and address how to navigate the application process, what makes an application competitive, and tips for applying.  We will pay special attention to grant programs targeting smaller organizations. The majority of the time will focus on responding to questions and drawing from the needs expressed by those in attendance to explore how NEH support can be leveraged to further their institutional goals.
							    \item [Engagement:]Participants will have the opportunity to learn about NEH funding opportunities for museums; discuss the funding process; develop strategies and tips for crafting a successful proposal; and learn about additional resources offered by NEH.
							    \item [Relationship to Theme:]By helping museum staff to craft competitive funding applications, we hope to provide financial resources for museums of every size to move Forward in serving their communities.
							    
                    \item [Additional Comments: ]There will be two NEH staff members presenting. We will have handouts and contact information for all museum-eligible funding opportunities. We will also be available before and after the session for one-on-one meetings.

                \end{description}
              \subsection*{Audience}
                \begin{description}
                  \item [Audiences:]Curators/Scientists/Historians~
                  \item[Professional Level:]All professional levels~
                \item[Scalability:] We are currently working to expand the funding opportunities for small and mid-size museums.

							
              \item[Other Comments:] We prioritize audience participation and discussion.
              \end{description}
            \subsection*{Participants}
              \subsubsection*{ Carolina Cortina }
              Submitter, Moderator, Presenter\newline
              Senior Program Officer, Division of Public Programs\newline
              National Endowment for the Humanities, Washington, DC
              \newline
              ccortina@neh.gov\newline
              
              202-606-8305\newline

              For ten years, I have been an NEH Program Officer for NEH's Division of Public Programs. I work directly with applicants to our museums exhibitions, historic sites and humanities discussion grant programs.\newline


              
                \subsubsection*{ Carolina Cortina }
                Moderator\newline
                Senior Program Officer, Division of Public Programs\newline
                National Endowment for the Humanities, Washington, DC
                \newline
                ccortina@neh.gov\newline
                
                202-606-8305\newline

                See above.\newline
                \emph{ (confirmed) }
              

              
                \subsubsection*{ Carolina Cortina }
                Presenter\newline
                Senior Program Officer, Division of Public Programs\newline
                National Endowment for the Humanities, Washington, DC
                \newline
                ccortina@neh.gov\newline
                
                202-606-8305\newline

                See Above
                \emph{ (confirmed) }
              

              
                \subsubsection*{ NEH Program Officer - To Be Decided based on availability NEH Program Officer - To Be Decided based on availability }
                Presenter\newline
                Senior Program Officer\newline
                National Endowment for the Humanities, Washington, DC
                \newline
                ccortina@neh.gov\newline
                
                202-606-8305\newline

                This will be a Program Officer from another NEH Division.
                \emph{ (not confirmed) }
              

              

              
        
          \newpage
          \section{ Fostering Institutional Knowledge in Exhibit Design, Fabrication, and Installation }
            \begin{description}
              \item [ID:]
              WMA2022\_171

              \item [Assigned to:]
                \item [Track:]Other~
              \end{description}

              Expertise in exhibit design, fabrication, and installation is often tied to the talents of individual staff and outside partners; their techniques rarely turn into institutional knowledge. How can we maintain staff and preserve standards and innovations so we can build upon them? 
Join panelists for a roundtable discussion of installation case studies, exploring how we can go from technique to training within a sustainable culture of innovation and productivity.

              \subsection*{Session Information}
                \begin{description}
                  \item [Format:] Regular session/panel (roundtable, single speaker, etc.)
							    
								  \item [Fee:]NA
							     
							    \item [Uniqueness:]Exhibit design and execution are often behind the scenes and not often examined, but enhancing these areas means a stronger product for our visitors.
							    \item [Objectives:]Institutional Knowledge
The session will promote the importance of maintaining institutional knowledge. In addition, several real world techniques will be examined that the audience can execute the day after the conference.
Staffing
Experienced staff with high morale means fewer mistakes are made, the development and installation is quicker, and the end result is highly engaging for the visitor. The audience will walk away with actionable ideas for onboarding new staff as well as retaining experienced staff. Maintaining high morale is a central part of increasing the capacity of staff, fun and quirky ways of raising morale will be discussed.
Innovation
The audience will learn the prerequisites to innovation and learn to recognize opportunities to help it flourish. Without the building blocks of previous work history (the institutional knowledge) coupled with experienced and efficient staff, innovation can stall. Additionally the audience will come away with methods of fostering and encouraging innovation in a time and budget friendly way.
							    \item [Engagement:]The audience is encouraged to ask questions of the panelists after their presentations. There will actionable techniques and ideas as take aways, the audience should come prepared to take a lot of notes. We intend to make the Powerpoint available via MuseumTrade.org.
							    \item [Relationship to Theme:]How do you move FORWARD without slipping backward? You retain your institutional knowledge and you encourage innovation. That may mean an improved documentation systems, an enhanced staff orientation, or an improvement to the culture and morale.
							    
                    \item [Additional Comments: ]Right now I have two strong yes' from my 2020 proposal panelists. One from Science, one from Art. One person from 2020 has left the field. I have feelers out for a third panelist. I have reached out to Seattle Children's Museum, the Burke Museum, and Oregon Historical Society. I haven't heard back either way from them. 
That said, I am very open to help in finding our third voice. Preferably female, of color, and not in art or science. Thanks in advance there.
I'm of course willing to merge if it makes for a better WMA 2022.

                \end{description}
              \subsection*{Audience}
                \begin{description}
                  \item [Audiences:]Directors/Executive/C-Suite~Educators~Emerging Museum Professionals~Exhibit Designers, Installers, Fabricators~General Audience~
                  \item[Professional Level:]Emerging Professional~General Audience~Mid-Career~Senior Level~
                \item[Scalability:] The session will cover issues faced by every museum size and type. Maintaining institutional knowledge and fostering innovation is a perennial problem for any business, but this session will focus solely on the design, installation, and maintenance of exhibits. Panelists for the session are from three museums types, each working with a different budget size.

							
              \item[Other Comments:] This session will focus on those that take the given exhibit topic or concept and get it across the finish line, opening day. This includes the design of the exhibit, the installation, maintenance, as well as the educational components that give context to what the visitor experiences.
This session seeks to help its audience retain the successes and failures along the way so they can build from and innovate on that knowledge.
              \end{description}
            \subsection*{Participants}
              \subsubsection*{ Matt Isble }
              Submitter, Moderator\newline
              Exhibition Designer and Head Preparator \& Founder of MuseumTrade.org\newline
              Crocker Art Museum, Sacramento, CA
              \newline
              misble@crockerart.org\newline
              matt@museumtrade.org\newline
              408-823-5939\newline

              I have chosen to moderate this panel. With over 20 years in the field, in various capacities, and in a variety of museum types I have a keen sense of the pain points and solutions. I'm also quite enthusiastic about innovations and solutions in exhibit design, installation, and maintenance. I'm sure my nerdy passion will come out and be infectious.\newline


              
                \subsubsection*{ Matt Isble }
                Moderator\newline
                Exhibition Designer and Lead Preparator\newline
                Crocker Art Museum, Sacramento, CA
                \newline
                misble@crockerart.org\newline
                matt@museumtrade.rg\newline
                408-823-5939\newline

                I have chosen to moderate this panel. With over 20 years in the field, in various capacities, and in a variety of museum types I have a keen sense of the pain points and solutions. I'm also quite enthusiastic about innovations and solutions in exhibit design, installation, and maintenance. I'm sure my nerdy passion will come out and be infectious.\newline
                \emph{ (confirmed) }
              

              
                \subsubsection*{ Ben Wildenhaus }
                Presenter\newline
                Head Preparator and Exhibition Designer\newline
                Tacoma Art Museum, Tacoma, WA
                \newline
                bwildenhaus@TacomaArtMuseum.org\newline
                
                253-272-4258 ext.3025\newline

                With more than a decade in mid sized art museums, Ben's deep knowledge of exhibition design and installation are further enhanced by experience opening two new museum wings.
                \emph{ (confirmed) }
              

              
                \subsubsection*{ Dave Laubenthal }
                Presenter\newline
                Creative Director\newline
                Oregon Museum of Science and Industry, Portland, OR
                \newline
                DLaubenthal@omsi.edu\newline
                
                503-797-4502\newline

                Dave brings expertise from a science museum, a strong background in design, and years of real world experience.
                \emph{ (confirmed) }
              

              

              
        
          \newpage
          \section{ Collaborating with community partners to enhance equity in STEM }
            \begin{description}
              \item [ID:]
              WMA2022\_173

              \item [Assigned to:]
                \item [Track:]
              \end{description}

              **In an attempt to be more inclusive, The Natural History Museum of Utah is collaborating with community groups on a continuing suite of workshops called “STEM Rooted in Culture.” **
** **
<strong>Western cultures are overrepresented in STEM and diversifying the voices that share science can make space for the many cultures that do science but are ignored by the limitations of our definition of “modern” science. This presentation will share our process and practices.</strong>

              \subsection*{Session Information}
                \begin{description}
                  \item [Format:] Regular session/panel (roundtable, single speaker, etc.)
							    
							    \item [Uniqueness:]Science is often viewed through a western lens. Our workshops are case studies of emerging practices that help undo this monochromatic view of science programs.
							    \item [Objectives:]By the end of the presentation the goal is for the audience to start to understand that modern science is cultural and how it is rooted in western culture. 
 
We will have group discussions to help identify new tactics in program development in order to help overcome exclusionary issues. 
 
We will give action points on what we have been doing during our program development process that has been working to reframe our own programming. 
 
We’ll show how engaging community partners in the process effectively shifts the end-goal of programming to be more aware and inclusive.
							    \item [Engagement:]This presentation is geared towards museum educators and program developers to help them reframe their pedagogical and facilitating approaches to science in a more inclusionary manner. 
 
We will have group discussions to help audience members identify their own definitions of STEM and actively work on changing mental models. We will share our experiences with attempting to shift our own departmental thought process and give effective practices that our museum implemented in our program development process.
							    \item [Relationship to Theme:]“traditional” science has culture affiliated to it; and it is overrepresented by a western lens. In an attempt to improve and decolonize our museums for the future, we need to start to acknowledge this bias. We need to redefine modern science and how it is communicated so we can better appreciate the many ways science can be practiced, shared, and understood.
							    
                \end{description}
              \subsection*{Audience}
                \begin{description}
                  \item [Audiences:]Educators~
                  \item[Professional Level:]All levels~
                \item[Scalability:] STEM Rooted in Culture is community based and focuses on understanding and learning from the community. As such any museum, big or small, can learn from our process. By putting community first, we are allowing museums to be learners alongside their visitors, not all-knowing entities. Our program is a step forward in reframing our place as museums within our communities and better understanding the people we serve. 

							
              \item[Other Comments:] Museum educators, exhibit developers, and curators that are interested in reframing their programs, collections, and interpretives to be more inclusive.
              \end{description}
            \subsection*{Participants}
              \subsubsection*{ Bonnie Jean Knighton }
              Submitter, Moderator, Presenter\newline
              Education Coordinator\newline
              Natural History Museum of Utah, Salt Lake city
              \newline
              bknighton@nhmu.utah.edu\newline
              bonniejean.knighton@gmail.com\newline
              385-722-6128\newline

              Bonnie Jean Knighton develops Educator Workshops and educational programming
for NHMU. She is a co-developer for the STEM Rooted in Culture workshops and has an educational background in Cultural Anthropology with 12 years of informal Museum educational experiences.\newline


              

              
                \subsubsection*{ Bonnie Jean Knighton }
                Presenter\newline
                Education Coordinator\newline
                Natural History Museum of Utah, Salt Lake city
                \newline
                bknighton@nhmu.utah.edu\newline
                
                4353270803\newline

                Bonnie Jean Knighton develops Educator Workshops and educational programming for NHMU. She is a co-developer for the STEM Rooted in Culture workshops and has an educational background in Cultural Anthropology with 12 years of informal Museum educational experiences.
                \emph{ (confirmed) }
              

              
                \subsubsection*{ Fanny Guadalupe Blauer }
                Presenter\newline
                Director\newline
                Artes De Mexico en Utah, Salt Lake City
                \newline
                admin@artesmexut.org\newline
                
                801-581-6927\newline

                Fanny has developed and harvested positive and trusting relationships with many influential Utah community members as well as multicultural organizations that we’ve collaborated with such as: University Neighborhood Partners, and the Center for Latin American Studies. She has been vital in pushing for more inclusive programming at our museum and advocating for the museum to work towards creating a more welcoming and trusted atmosphere for minority groups in our community.
                \emph{ (confirmed) }
              

              
                \subsubsection*{ Mariana Alliatti Joaquim }
                Presenter\newline
                School Outreach Coordinator\newline
                Natural History Museum of Utah, Salt Lake City
                \newline
                mjoaquim@nhmu.utah.edu\newline
                
                801-471-3100\newline

                Mariana Alliatti Joaquim is an historian from Brazil with an emphasis in Natural History, Latin American History, History of Science, Jesuit History, and Theory. Mariana has a BA in History Teaching from the Universidade do Vale do Rio dos Sinos, Brazil, and an MA in History from the University of Utah. Mariana leads the planning and development for the STEM Rooted in Culture educator workshop series that combines culture, social studies, and STEM through focusing on community partnerships and teaching about equity, inclusion, and diversity in classrooms.
                \emph{ (confirmed) }
              

              
                \subsubsection*{ Katie Worthen }
                Presenter\newline
                STEM Education Coordinator\newline
                Natural History Museum of Utah, Salt Lake City
                \newline
                kworthen@nhmu.utah.edu\newline
                
                801-581-5567\newline

                Katie Worthen has worked in the museum field and informal science education for over 15 years and has Masters in Geology as well as Museum Practices from Brigham Young University. In her current role as STEM Education Coordinator she helps to develop and deliver STEM teacher professional development opportunities, particularly the STEM in Stories and STEM Rooted in Culture series.
                \emph{ (confirmed) }
              
        
          \newpage
          \section{ The Art of Observation: Museums and Medicine }
            \begin{description}
              \item [ID:]
              WMA2022\_174

              \item [Assigned to:]
                \item [Track:]
              \end{description}

              This session examines three different partnerships between museums and medical institutions. Each offers a different approach on how a museum\&rsquo;s collection can provide new ways for training medical professionals and addressing patient care and wellness. Panelists representing work from the  Seattle Art Museum, Crocker Art Museum, and The Hammer Museum will discuss how a museum\&rsquo;s collection, open ended inquiry, close looking, and art-making can engage healthcare professionals and patients, while panelists from Virginia Mason Hospital, UC Davis School of Medicine, and UCLA School of Medicine will dive into current research around this budding field, and ways for museums to further engage medical schools, hospitals, and patients.

              \subsection*{Session Information}
                \begin{description}
                  \item [Format:] Regular session/panel (roundtable, single speaker, etc.)
							    
							    \item [Uniqueness:]This session will examine innovative approaches to engage new communities, increase museum accessibility, build lifelong learners, and further center museums as community anchor.
							    \item [Objectives:]* Audiences will walk away with the tools and practical approaches to engaging the medical community, and ways that their institutions can begin the process of building partnerships, or exploring community needs, and ways they can be more accessible to healthcare professionals and patients.

* Audiences will gain a deeper knowledge of existing partnerships regionally and nationally and the variety of medical partnerships, and how they manifest themselves.

*Audiences will dive further into various concepts around museum accessibility as it pertains to  museums as places of wellness and  healing for communities experiencing chronic illness and pain.
							    \item [Engagement:]* Handouts will be provided to audiences that explore regional and national programs around medicine and museums, as well as creative aging, Alzheimer and the Arts, patient accessibility, and wellness. 

• Presenters will provide a dynamic Q\&A session designed to further explore creative strategies to encourage audience members to explore these unconventional partnerships in their own institutions

• Panelists will share concrete examples that garnered success or presented challenges for their own institutions and/or their audiences.
							    \item [Relationship to Theme:]As we think FORWARD, a conversation about healthcare and the museum is extremely relevant: manifesting a future where museums are valued as a matter of public health. As our nation grapples with access to affordable healthcare, this session will resonate on multiple levels, looking towards a future where art and culture become an indispensable part of a healthy society.
							    
                    \item [Additional Comments: ]At least one presenter will need to be zoomed in from Europe. Since the pandemic, most of the panelists I had confirmed in 2020 have had significant shifts in their careers, locations, and insitutions.

                \end{description}
              \subsection*{Audience}
                \begin{description}
                  \item [Audiences:]Board Members~Curators/Scientists/Historians~Diversity and inclusion specialists~Educators~Emerging Museum Professionals~Evaluators~General Audience~Marketing \& Communications (Including Social Media)~Programs~Registrars, Collections Managers~Volunteer Managers~
                  \item[Professional Level:]All levels~
                \item[Scalability:] The outcomes of this proposal are easily custom tailored to work effectively with various types and sizes of institutions. It is definitely more easily tailored to various sizes, but will most likely appeal to institutions whose collections trend towards visual art, living histories, historical and cultural institutions.


							
              \end{description}
            \subsection*{Participants}
              \subsubsection*{ Sarah Bloom }
              Submitter, Moderator\newline
              Head of Exhibitions, Education \& Interpretation\newline
              Bill \& Melinda Gates Foundation, Seattle
              \newline
              sarahtee15@gmail.com\newline
              sarah.bloom@gatesfoundation.org\newline
              6177972081\newline

              I will moderate this session\newline


              
                \subsubsection*{ Sarah Bloom }
                Moderator\newline
                Head of Exhibitions, Education \& Interpretation\newline
                Bill \& Melinda Gates Foundation, Discovery Center, Seattle
                \newline
                sarahtee15@gmail.com\newline
                sarah.bloom@gatesfoundation.org\newline
                6177972081\newline

                Moderator is familiar with this topic and has experience leading the Art + Medicine program as former associate director of education at SAM.\newline
                \emph{ (confirmed) }
              

              
                \subsubsection*{ Amish Dave }
                Presenter\newline
                Rheumatology Specialist at Virginia Mason\newline
                Virginia Mason Medical Center, Seattle
                \newline
                amish.dave@virginiamason.org\newline
                
                

                Dr. Dave  piloted the Art + Medicine program in partnership with the Seattle Art Museum and the Virginia Mason Medical Center in Seattle. Dr. Dave will present on the program’s focus of utilizing the museum’s collection to support faculty learning objectives of building communication and cultural awareness in first and second year residents, and utilizing the close observation, group discussion, and art making sessions to address issues around patient care, wellness, and physician burnout.

(Please note, Dr. Dave has clinic on Thursdays, and can participate if the panel is held on Friday - Sunday)
                \emph{ (confirmed) }
              

              
                \subsubsection*{ Houghton Kinsman }
                Presenter\newline
                Adult Education Coordinator\newline
                Crocker Art Museum, Sacramento, CA
                \newline
                HKinsman@crockerartmuseum.org\newline
                
                916-808-1962\newline

                Kinsman oversees the program: ArtRx in partnership with the University of California, Davis School of Medicine. This public program is open to individuals living with chronic pain. As part of the partnership with the Integrative Pain Management Program at UC Davis, the goal of the program is to mitigate the isolation that often accompanies chronic pain and help people living with chronic pain experience the joy of art engagement.
                \emph{ (confirmed) }
              

              
                \subsubsection*{ Ian Koebner }
                Presenter\newline
                Director of Integrative Pain Management and an Assistant Professor in the Department of Anesthesiology and Pain Medicine at UC Davis\newline
                 UC Davis School of Medicine, Davis, CA
                \newline
                ikoebner@ucdavis.edu\newline
                
                

                Ian Koebner, PhD, is the Director of Integrative Pain Management and an Assistant Professor in the Department of Anesthesiology and Pain Medicine at UC Davis, as well as a Cultural Agents Fellow at Harvard University. As the founding director of an international arts-based conflict resolution organization he has over a decade of executive-level arts management experience and has curated over 50 exhibitions or performances. He is a past Rockefeller Foundation Bellagio Center Practitioner Resident and a current National Institute of Health Clinical and Translational Science Center KL2 Scholar at the University of California, Davis. His project - “A(n)esthetics and the Analgesic Museum” - explores the role of museums as public health partners to relieve pain 

(please note, Dr. Koebner has relocated to Europe and can participate via Zoom if panel is accepted)
                \emph{ (confirmed) }
              

              
                \subsubsection*{ Hallie  Scott }
                Presenter\newline
                Specialist: University Audiences\newline
                Hammer Museum, Los Angeles
                \newline
                hscott@hammer.ucla.edu\newline
                
                

                Hallie Scott leads the UCLA Community and Global Psychiatry program in partnership with UCLA Medical School where she facilitates discussions about works of art with current and future psychiatrists, followed by a debrief connecting the experience to psychiatric care. After deep analyses of the works, participants critically reflected on parallels between interpreting art and their medical practice, noting the importance of taking time to diagnose patients after careful observation and analysis.
                \emph{ (not confirmed) }
              
        
          \newpage
          \section{ Pulling the Past Into the Present }
            \begin{description}
              \item [ID:]
              WMA2022\_175

              \item [Assigned to:]
                \item [Track:]
              \end{description}

              Learn how Western Neighborhoods Project (WNP), a one-employee community history nonprofit in San Francisco, California, has expanded its relevance and reach by incorporating the arts into its history practice and providing historic context to difficult conversations, like the toppling of controversial monuments. We’ll discuss how this cuts across the board, from fundraising campaigns to programming and everything in between. There’s something for everyone at this session.

              \subsection*{Session Information}
                \begin{description}
                  \item [Format:] Regular session/panel (roundtable, single speaker, etc.)
							    
							    \item [Uniqueness:]Run by women under the age of 40, WNP is not your grandpa’s history group. See how we push the boundaries of History.
							    \item [Objectives:]With a new executive director at the helm in 2020, WNP dramatically expanded the scope of its vision while staying on mission by including local artists and taking on controversial subjects (sometimes to the detriment of her emotional well-being). This session will highlight initiatives that increased earned revenue and visibility by doing the unthinkable; created dynamic programming that speaks to audiences outside our traditional demographic, which is aging beyond our reach; and provided critical context to heated debates around the purpose of public history and the use of public space. 
We accomplished the first by partnering with a fine art conservator and contemporary art gallery to raise over \$150,000 in less than three weeks, allowing us to save historic art and artifacts from auction and then, with support from the National Park Service, open a temporary museum in the former Cliff House Restaurant, which closed in December 2020. The second is illustrated through a collaborative program with the Fort Point Beer Company and a graphic zine created by artist Thorsten Sideboard that engages young learners with illustrated activities centered on history. And, to explain the third, we’ll highlight a virtual program titled “Murals, Monuments, and Memorials” that brought a panel of artists together to discuss how we can rethink public art. 

In the (hopefully) post-pandemic era, cultural nonprofits can no longer afford to be siloed within their own genre if they want to be relevant and we hope this session will show you how.
							    \item [Engagement:]This will be a lively discussion between two overworked but eternally enthusiastic museum professionals: humorous, honest (it’s hard to run a nonprofit and make history interesting), and visually engaging. We’ll also hand out souvenirs and encourage our audience to be part of the conversation.
							    \item [Relationship to Theme:]This session is all about moving cultural nonprofits forward, beyond the ivory tower approach of an “infallible historian” and into a more democratic, fun, and relevant future for History that is centered in community.
							    
                \end{description}
              \subsection*{Audience}
                \begin{description}
                  \item [Audiences:]General Audience~
                  \item[Professional Level:]All levels~
                \item[Scalability:] This session is incredibly scalable. If we can do it with only one employee, you can do it too.

							
              \end{description}
            \subsection*{Participants}
              \subsubsection*{ Nicole Meldahl }
              Submitter, Moderator, Presenter\newline
              Executive Director\newline
              Western Neighborhoods Project, San Francisco, CA
              \newline
              nicole@outsidelands.org\newline
              wnpoffice@outsidelands.org\newline
              415-661-1000\newline

              I've provided the strategic vision for all the presented  initiatives  and, as the only employee of Western Neighborhoods Project, is responsible for project managing them all. Aside from that, I'm told I'm pretty funny and am known for my general enthusiasm.\newline


              

              
                \subsubsection*{ Nicole Meldahl }
                Presenter\newline
                Executive Director\newline
                Western Neighborhoods Project, San Francisco, CA
                \newline
                nicole@outsidelands.org\newline
                wnpoffice@outsidelands.org\newline
                415-661-1000\newline

                Nicole has provided the strategic vision for all the presented  initiatives  and, as the only employee of Western Neighborhoods Project, is responsible for project managing them all.
                \emph{ (confirmed) }
              

              
                \subsubsection*{ Eva Laflamme }
                Presenter\newline
                Development Committee Chair, Board of Directors\newline
                Western Neighborhoods Project, San Francisco, CA
                \newline
                eva@outsidelands.org\newline
                elaflamme@mail.sfsu.edu\newline
                

                Eva joined the WNP Board as part of the organization’s first open call for new board members in 2020 and plays a critical role in helping WNP push outside our comfort zone in fundraising and beyond.
                \emph{ (confirmed) }
              

              

              
        
          \newpage
          \section{ The Horror-egon Trail!: The Campfire Continues….. Even More true tales of the unexpected, unbelievable, and unmanageable.  }
            \begin{description}
              \item [ID:]
              WMA2022\_177

              \item [Assigned to:]
                \item [Track:]Other~
              \end{description}

              Grab a s’more and take a seat around the (artificial) campfire as we swap even more unimaginable but true tales of museum craft, such as:
||Ghoulish stories of guest services (The Mysterious handprint…)
||Hair raising tales of HR (They did what?)
||Truly terrifying tours ( I turned around and they were gone!)
|| Eerie archival practices (What’s in the box?)
*Learn from those who have lived to tell these tales?..and share your own if you dare!

              \subsection*{Session Information}
                \begin{description}
                  \item [Format:] Regular session/panel (roundtable, single speaker, etc.)
							    
							    \item [Uniqueness:]This session will share "extreme" museum professional experiences and lessons learned in a fun and creative way
							    \item [Objectives:]This session will show that challenges cross all disciplines, experience levels, museum types and come in all shapes and sizes but can be remedied; give support, hope and inspiration and positive healing to colleagues feeling challenged; flips the power dynamic to engage audience members as speakers; have a fun time while learning. This is needed now more than ever as we emerge from the living hell that was COVID
							    \item [Engagement:]By using a campfire approach to story swapping, this program will engage audience members in an innovative way and encourage audience participation. Story prompts will be used to help to guide the stories and allow for a range of topics. The prompts will be used/formatted in a creative way. The room will not be completely dark so participants will feel welcome and a welcome sign will be used too.
							    \item [Relationship to Theme:]This session will move participants forward in their careers, in the industry, and in their social/professional networks. The informal story swapping model encourages audience participation; it is breaks down barriers between people by being accessible and fun, and engages the audience directly with the topic and the speakers. The format is open to a diverse audience and allows all levels of experience and professional areas. Shared experiences are a great equalizer.
							    
                \end{description}
              \subsection*{Audience}
                \begin{description}
                  \item [Audiences:]General Audience~
                  \item[Professional Level:]All levels~
                \item[Scalability:] Personal stories that can make an impact on all. Open to all.

							
              \end{description}
            \subsection*{Participants}
              \subsubsection*{ Seth Margolis }
              Submitter, Moderator, Presenter\newline
              \newline
              The Museum of Flight, SEATTLE, WA
              \newline
              smargolis@museumofflight.org\newline
              smargolis@museumofflight.org\newline
              2067687116\newline

              Seth is a story collector and enjoys getting others to tell tales too. He has seen and done it all in his museum career and isn’t afraid to spill the beans. Seth recognizes the power of this type of session and will moderate it so that there is solid mix of topics and tales. He also has the props for the session\newline


              

              
                \subsubsection*{ Molly Wilmoth }
                Presenter\newline
                Bonnie and Oliver Steele III Curator of Education\newline
                High Desert Museum, Bend, OR
                \newline
                
                
                541.382.4754\newline

                Molly has worked in museums large and small and can share stories from all sorts of museum types. She is one of the most positive people in the museum industry and will share how to be upbeat and not beat up when facing challenges!
                \emph{ (confirmed) }
              

              
                \subsubsection*{ Katie Buckingham }
                Presenter\newline
                Curator\newline
                Museum of Glass, Tacoma, WA
                \newline
                kbuckingham@museumofglass.org\newline
                kbuckingham@museumofglass.org\newline
                253.284.4750\newline

                Katie is awesome and can share stories from the art world!
                \emph{ (confirmed) }
              

              

              
        
          \newpage
          \section{ Clowning Around: Creating Inclusive Collaborations with Carla Rossi }
            \begin{description}
              \item [ID:]
              WMA2022\_178

              \item [Assigned to:]
                \item [Track:]
              \end{description}

              Through exhibitions, programs, and content, the Portland Art Museum, Seattle Art Museum, and Museum of Natural and Cultural History have partnered with Anthony Hudson (Confederated Tribes of Grand Ronde)—the artist who transforms into Portland’s premier drag clown, Carla Rossi—to initiate intersecting conversations around identity, gender, capitalism, and white supremacy. Come explore ways in which museums can challenge and address these issues while simultaneously creating fun, welcoming, affirmative, and joyful spaces.

              \subsection*{Session Information}
                \begin{description}
                  \item [Format:] Regular session/panel (roundtable, single speaker, etc.)
							    
							    \item [Uniqueness:]An opportunity to hear from an extraordinary performance artist and their museum partners about innovative collaborations that created meaningful community connections.
							    \item [Objectives:]The audience will…

- Consider what artist collaborations can look like at varied organizations, and how artist-led collaborations can create welcoming, affirmative spaces within the museum
- Recognize that artist-led collaborations can provide an opportunity to explore heavy issues of identity, gender, white supremacy, and oppression in ways that are welcoming to new audiences and non-excluding to an organization’s existing audience—and can still create joy!
- Be inspired to look for ways their own organizations can partner with artists to facilitate greater connection, appreciation, and empathy across diverse communities
							    \item [Engagement:]- Anthony will facilitate a short discussion with the audience, where Anthony poses some questions based on the PAM LGBTQ+ youth program: Welcome to Clown Town: A Drag Workshop with Carla Rossi
- Ample time allotted for Q\&A with Anthony and museum staff
							    \item [Relationship to Theme:]This session will touch on numerous aspects of the 2022 theme, FORWARD, including:
    • Embracing DEIA
    • Acting as agents of social change, social justice, and activism through art and exploration
    • Joining into unconventional and innovative partnerships
    • Learning, from an institutional perspective, how to listen to marginalized communities and offer space to suit them on their own terms (Land Back!)
							    
                \end{description}
              \subsection*{Audience}
                \begin{description}
                  \item [Audiences:]Diversity and inclusion specialists~Educators~Events Planning~General Audience~Programs~
                  \item[Professional Level:]All professional levels~
                \item[Scalability:] Organizations of any type and size benefit from thinking about ways to be more welcoming, inclusive, and an agent of social change. The case studies presented are from both art museums and a cultural \& natural history museum, have different target audiences (i.e., teens, adults, university students), and represent different types of engagement/programs.

							
              \end{description}
            \subsection*{Participants}
              \subsubsection*{ Lauren Willis }
              Submitter, Presenter\newline
              Curator of Academic Programs\newline
              Museum of Natural and Cultural History, Eugene, OR
              \newline
              lmw@uoregon.edu\newline
              
              5413463030\newline

              To share the case study of the MNCH’s collaboration with Anthony Hudson, including an exhibit on display from 2019-2021, and related programs in winter 2020; brief descriptions of the programs; what the partnership meant for the museum and how it aligned with the museum’s strategic initiatives.\newline


              
                \subsubsection*{ Anthony Hudson }
                Moderator\newline
                Portland’s premier drag clown; Community Programmer\newline
                Hollywood Theatre (Community Programmer), Portland, OR
                \newline
                hudsonology@gmail.com\newline
                
                503-930-6618\newline

                To share an artist’s perspective on collaborations/partnerships with museums, and how these partnerships fit into Anthony’s practice as an artist and community activist.\newline
                \emph{ (confirmed) }
              

              
                \subsubsection*{ Hana Layson }
                Presenter\newline
                Head of Youth and Educator Programs\newline
                Portland Art Museum, Portland, OR
                \newline
                hana.layson@pam.org\newline
                
                5032764329\newline

                To discuss Portland Art Museum’s multi-year collaborations with Anthony / Carla during the annual LGBTQ+ youth event POWER UP.
                \emph{ (confirmed) }
              

              
                \subsubsection*{ Natali Wiseman }
                Presenter\newline
                Design Manager\newline
                Seattle Art Museum, Seattle, WA
                \newline
                NataliW@SeattleArtMuseum.org\newline
                
                8477274338\newline

                To discuss Seattle Art Museum’s collaboration with Anthony / Carla as related to the Jeffrey Gibson: Like a Hammer exhibition. This collaboration occurred in two separate capacities: an artistic response video created with  SAM’s Communications team, and a Dance and Drag pop-up event organized by SAM’s Education team.
                \emph{ (confirmed) }
              

              

              
        
          \newpage
          \section{ Archives Alive! Activating Archives for Engagement and Equity }
            \begin{description}
              \item [ID:]
              WMA2022\_179

              \item [Assigned to:]
                \item [Track:]Collections~
              \end{description}

              Exciting possibilities await us when we invest in museum archives! Through archives, relevant and engaging connections happen and a more inclusive, approachable space for community results. Learn how nascent collection and institutional archives expand accessibility, reach new audiences, and create unexpected discoveries that empower and shift narratives. This session will explore approaches to improving collections accessibility and holding space for diverse connections through archives.

              \subsection*{Session Information}
                \begin{description}
                  \item [Format:] Regular session/panel (roundtable, single speaker, etc.)
							    
								  \item [Fee:]n/a
							     
							    \item [Uniqueness:]As centers for community, museums can foster equity and inclusivity. Archives research in museums invite new perspectives and transparency into institutional and collections histories.
							    \item [Objectives:]1. Analyze the definition and classification of “archives” in museums and other collecting organizations, as well as examine how museum records are currently utilized and the potential they hold for developing new audiences. Specifically, how are institutional records (such as administrative or exhibition files) and collection records (accession files, catalog, and research) viewed through an archival lens?
2. Think critically about current systems of archives organization, both physically and intellectually, and their effectiveness in relevant, inclusive, and accurate information sharing and retrieval. What are the best approaches for museum staff who have other responsibilities in developing archives?
3. Identify methods for using the archives innovatively in fostering outreach to new interdisciplinary audiences, aligning the museum with social justice goals of shifting authoritative voice to develop more inclusive storytelling.
							    \item [Engagement:]As attendees arrive, project slide deck questions and provide same questions on 3x5” cards for engagement between speaker sections.
Questions might include: 
Are archives being used in your organization? If so, how?
How can you imagine using archives differently?
What does accessibility look like in museum archives?
How are collection records currently being used?
Who engages with your archives?
Do you have an archive accessibility plan?

Need computer projector with laptop hook up.
							    \item [Relationship to Theme:]A brave, FORWARD-looking, integrated approach to archive and collection accessibility is key to museum relevance and community voice. In the past, records oversight at museums kept records largely hidden. Activating the archives opens doors for an organization’s accountability, discovery of cultural responsibilities, and enrichment of education and interpretation. The community dialog that archival information informs can challenge institutional history and collecting practices, thereby empowering new perspectives.
							    
                    \item [Additional Comments: ]Thank you for the opportunity to present our session vision on the importance of museum archives in illuminating collections and supporting institutional relevance. We look forward to hearing any feedback.

                \end{description}
              \subsection*{Audience}
                \begin{description}
                  \item [Audiences:]Curators/Scientists/Historians~Diversity and inclusion specialists~Educators~Emerging Museum Professionals~Events Planning~General Audience~Library Staff~Marketing \& Communications (Including Social Media)~Programs~Registrars, Collections Managers~Technology~Visitor Services~
                  \item[Professional Level:]All professional levels~
                \item[Scalability:] Regardless of size, all museums, libraries, and even private sector organizations, have or hold institutional history and records (archives) which need to be referenced, whether for use in enriching programming and exhibits; supporting scholarly research; enabling operational, legal and financial activities of their organization; or for creating outreach opportunities with communities.

							
              \item[Other Comments:] Classification standards, nomenclatures, and lexicons are useful tools to describe and organize collections and archives for the purpose of physical and cultural preservation, and efficient, effective information retrieval. These tools require critical consideration and must never be used to define people, objects, or communities. Language matters. We should forever be inspecting classification standards, whose origins can be highly problematic due to the traditionally homogenous groups who originally developed them.

Further, in the framework of making physical records/archives accessible, databases are also useful tools. However, culturally sensitive content warnings, or removal of images of human remains, etc. are important to consider with community partners.
              \end{description}
            \subsection*{Participants}
              \subsubsection*{ Gina Caprari }
              Submitter, Moderator, Presenter\newline
              Registrar and Faculty Lecturer\newline
              Global Museum and Museum Studies program, San Francisco State University, San Francisco, CA
              \newline
              gcaprari@sfsu.edu\newline
              ginacaprari@gmail.com\newline
              

              Gina is cross-trained as museum professional, and librarian and archivist. With perspective working in a University museum holding global cultural collections, Gina’s work is paired with a Museum Studies academic program, where the primary goal of working with archives and collections is for use as a teaching tool for emerging museum professionals. Gina is currently activating the archives by establishing design and organization of the archives storage and research room following a comprehensive collections move, creating finding aids to improve accessibility for research, and investing in ongoing systematic research to enrich all areas of museum programs and operations.\newline


              
                \subsubsection*{ Linda Waterfield }
                Moderator, Presenter\newline
                Head of Registration\newline
                Phoebe A. Hearst Museum of Anthropology, University of California at Berkeley, Berkeley, CA
                \newline
                lwaterfield@berkeley.edu\newline
                
                510-643-6390\newline

                Linda holds degrees in art history and museum studies and has worked at art, history, and cultural museums. While juggling registrarial duties for approximately 3.8 million objects, Linda oversaw the creation of the museum’s archives as a result of a massive five-year collections move. Linda will provide perspective on how to create a public-facing archive to support museum collections accessibility and research. Interested in the intersection of collections, research, and cultural sovereignty, she is currently activating the archives by creating finding aids to improve accessibility and discoverability through the Online Archive of California (OAC).\newline
                \emph{ (confirmed) }
              

              
                \subsubsection*{ Peggy Tran-Le }
                Presenter\newline
                Research and Technical Services Managing Archivist\newline
                Archives and Special Collections at UCSF Library, University of California, San Francisco, San Francisco, CA
                \newline
                Peggy.Tran-Le@ucsf.edu\newline
                
                

                Peggy offers the experience of an archivist who has worked in both a university archives with medical focus and with dedicated exhibition space, as well as an art museum’s institutional archives.
At UCSF, her responsibilities include expanding access by increasing digitization efforts and accessioning of backlog content of faculty and department papers. 
As the archivist and records manager at SFMOMA, she collaborated with colleagues from other departments to activate the museum’s off-site collections facility in order to increase access and visibility. During the pandemic’s restrictions to on-site research, she increased digitization work to sustain researcher access to archival collections.
                \emph{ (confirmed) }
              

              

              

              
        
          \newpage
          \section{ Museum People: Exploring Museum Workforce Issues in 2022 }
            \begin{description}
              \item [ID:]
              WMA2022\_180

              \item [Assigned to:]
                \item [Track:]Leadership/Careerpath~
              \end{description}

              This session addresses museum workforce issues through the lens of an organization originally created in reaction to the pandemic. We will facilitate open discussion about issues – some of which are new and others of which have surfaced in light of the events of the past two years. The goal is to learn from each other by sharing what’s happening across institutions, identifying short- and long-term concerns and barriers to forward movement.

              \subsection*{Session Information}
                \begin{description}
                  \item [Format:] Regular session/panel (roundtable, single speaker, etc.)
							    
								  \item [Fee:][na]
							     
							    \item [Uniqueness:]The presenting organization is innovative and timely – created in direct response to the pandemic and continuously adapting in this time of uncertainty and rebuilding.
							    \item [Objectives:]Affect outcome: Participants will have increased sense of connection and community, and they will feel supported by peers.

Knowledge outcome: Participants will have increased understanding of contemporary workforce issues across different types or museums and potential resources for working through issues.
							    \item [Engagement:]Most of the session will be small group discussions using the World Café Method, which requires room set-up of round tables and seating for 4 - 10 people per table. After welcome and introduction – which will use PowerPoint \& video - participants will gather at tables with discussion prompts. After 15 minutes; participants will move to a different table/prompt; repeat 3 rounds. The session ends with full-group “harvesting” of insights.
							    \item [Relationship to Theme:]This session embodies the theme, FORWARD, as focused on the workforce, and it allows for sharing of stories of struggle and breakthroughs, all with an aim towards moving forward in a sense of community and with peer support.
							    
                    \item [Additional Comments: ]  The session presenter and facilitator is a seasoned museum person (recently an independent museum consultant) who will be representing MuseumExpert.org, an all-volunteer, not-for-profit organization that was originally created in reaction to the COVID-19 pandemic and has evolved to address museum workforce issues across all types of museums and with people at all levels and career stages. See MuseumExpert.org for more info.
 

                \end{description}
              \subsection*{Audience}
                \begin{description}
                  \item [Audiences:]General Audience~
                  \item[Professional Level:]General Audience~
                \item[Scalability:] Each museum, no matter the type or size, has staff of some sort, and each museum has been impacted in some way by the disruption of the past two years. Whether the staff is large or small, paid or volunteer, the mission of each organization is ultimately carried out by people. 

							
              \item[Other Comments:] Because this session addresses the museum workforce, writ large, it is relevant to anyone who identifies as a “museum person.” It may be of particular use for those in decision-making roles, and especially enlightening for those in leadership, both formal (i.e., by role) or informal (by influence).
              \end{description}
            \subsection*{Participants}
              \subsubsection*{ Rita Deedrick }
              Submitter, Moderator, Presenter\newline
              Volunteer\newline
              MuseumExpert.org, Columbus, Ohio
              \newline
              ritadeedrick@gmail.com\newline
              
              (614) 204-6717\newline

              Ms. Deedrick is part of MuseumExpert.org’s leadership team and is an experienced facilitator with over 25 years of museum experience.\newline


              
                \subsubsection*{ Rita Deedrick }
                Moderator, Presenter\newline
                Volunteer\newline
                MuseumExpert.org, Columbus, Ohio
                \newline
                ritadeedrick@gmail.com\newline
                
                (614) 204-6717\newline

                Ms. Deedrick is part of MuseumExpert.org’s leadership team and is an experienced facilitator with over 25 years of museum experience.\newline
                \emph{ (confirmed) }
              

              
                \subsubsection*{ Rita Deedrick }
                Presenter\newline
                Volunteer\newline
                MuseumExpert.org, Columbus, Ohio
                \newline
                ritadeedrick@gmail.com\newline
                
                (614) 204-6717\newline

                Ms. Deedrick is part of MuseumExpert.org’s leadership team and is an experienced facilitator with over 25 years of museum experience.
                \emph{ (confirmed) }
              

              

              

              
        
          \newpage
          \section{ From Intention to Action: Decolonial Perspectives from the Burke Museum }
            \begin{description}
              \item [ID:]
              WMA2022\_181

              \item [Assigned to:]
                \item [Track:]
              \end{description}

              Decolonization heals. Decolonization reconnects people to the natural environment. Decolonization finds common ground. 
At the Burke Museum, we are embedding Indigenous voices in all aspects of museum practice in ways that promote decolonization and contribute to meaningful systemic change.
Join us for a conversation about how we are creating an institutional culture that values and reflects the input of a broader range of voices and thus meets the needs of a broader range of audiences.

              \subsection*{Session Information}
                \begin{description}
                  \item [Format:] Regular session/panel (roundtable, single speaker, etc.)
							    
								  \item [Fee:]none
							     
							    \item [Uniqueness:]This session takes decolonization out of the abstract and provides concrete examples of how a museum advances Indigenous participation and perspectives through structural leadership changes.
							    \item [Objectives:]In 2019, the Burke Museum turned itself Inside/Out, establishing new levels of openness in all aspects of museum practice. This session will focus on one key aspect of this model, a deeper engagement with cultural stakeholders, acknowledging and addressing the issues in our museum’s colonial history and fully committing to creating opportunities and access for diverse communities.

Session attendees will hear from the Burke’s Tribal Liaison (TL), former Executive Director, and members of the Native American Advisory Board (NAAB), and will learn:

- How community voices—primarily via the NAAB—led to the creation of the TL role and defined its place within our organizational structure; 
- What decolonization looks like at the Burke, and how it affects not only cultural practices but also scientific research; and 
- What new initiatives the TL has put into place—including a yearlong Assessment and Healing Plan—and how this work has served to support the Burke’s mission to “care for and share natural and cultural collections so all people can learn, be inspired, generate knowledge, feel joy, and heal.”

Attendees will leave this session with a better understanding of how consultation with community advisory boards is essential to decolonization. At the Burke, our TL and NAAB collaborate with staff members in the development, communications, exhibits and visitors experience departments, as well as with our fundraising board and volunteers. These wide-ranging interactions deepen cohesion and collegiality within and among museum departments and stakeholder groups.
							    \item [Engagement:]The panel will engage the audience through storytelling and/or a story circle and an interactive format that provides time for reflection. Depending on the size of the audience, we will invoke a cultural practice that promotes engagement by strongly encouraging all participants to speak, even if only to verbally acknowledge their presence in the conversation and in the physical space.
							    \item [Relationship to Theme:]The museum’s sharper focus on decolonization emerged during the planning for the new Burke Museum; the need for this work only intensified through the COVID-19 pandemic and the renewed racial reckoning of 2020 and beyond. For the Burke’s staff and leadership, these initiatives represent a key step as we move FORWARD, into a new museum and a new, better partnership with the Indigenous communities we serve and whose objects we care for.
							    
                \end{description}
              \subsection*{Audience}
                \begin{description}
                  \item [Audiences:]General Audience~
                  \item[Professional Level:]All levels~
                \item[Scalability:] The Burke’s approach to decolonization, which embeds Indigenous voices early on in museum decision-making of all types, is of value to those organizations that serve Indigenous communities through collections, programming, education, and other museum activities. By focusing on the process by which we came to turn our intentions into actions, we hope to provide a model that other institutions--regardless of size--can adapt and use to best meet the needs of their own internal and external constituencies.

							
              \end{description}
            \subsection*{Participants}
              \subsubsection*{ Sumathi Raghavan }
              Submitter\newline
              Manager of Corporate and Foundation Relations\newline
              Burke Museum of Natural History and Culture, Seattle, WA
              \newline
              sumathi@uw.edu\newline
              
              

              


              
                \subsubsection*{ Polly Olsen (Yakama) }
                Moderator, Presenter\newline
                Tribal Liaison and Director of DEAI + Decolonization\newline
                Burke Museum, Seattle, WA
                \newline
                polly@uw.edu\newline
                
                206-543-5946\newline

                As Tribal Liaison and Director of DEAI + Decolonization at the Burke, Polly Olsen oversees the museum’s DEAI and Decol efforts and is the main liaison with the Native American Advisory Board. In these roles--and with both institutional and tribal affiliation--she can speak to the totality of the Burke's DEAI + Decol work.\newline
                \emph{ (confirmed) }
              

              
                \subsubsection*{ Dr. Julie Stein }
                Presenter\newline
                Former Executive Director\newline
                Burke Museum, Seattle, WA
                \newline
                jkstein@uw.edu\newline
                
                

                As ED of the Burke during the time period covered by this panel, Julie Stein can share the institutional leadership perspective on why this work was--and is--so crucial to museum and community well-being.
                \emph{ (confirmed) }
              

              
                \subsubsection*{ Yellowash Washines (Yakama) }
                Presenter\newline
                NAAB Executive Committee member\newline
                Yakama Nation, Dept. of Fisheries, Toppenish, WA
                \newline
                wasd@yakamafish.nsn.gov\newline
                
                

                As a member of the NAAB Executive Committee, Yakama Nation elder,  and former Executive Chairman of the Yakama Nation General Council, Washines can offer a unique perspective on the Burke’s historical relationship with the Tribes and the distance we have traveled together.
                \emph{ (confirmed) }
              

              
                \subsubsection*{ Josephine (JoJo) Jefferson }
                Presenter\newline
                NAAB Executive Committee member\newline
                Swinomish Tribal Preservation Office, La Conner, WA
                \newline
                jpeters@swinomish.nsn.us\newline
                
                

                As theTribal Historic Preservation Officer at Swinomish Indian Tribal Community and a younger member of the NAAB Executive Committee, Jefferson can bring her own professional and generational perspective to the conversation.
                \emph{ (confirmed) }
              

              
        
          \newpage
          \section{ Measuring Social Impact for Strategic Change }
            \begin{description}
              \item [ID:]
              WMA2022\_182

              \item [Assigned to:]
                \item [Track:]
              \end{description}

              A national research study, Measurement of Museum Social Impact (MOMSI), is working to create a survey to help museums measure their social impact. In this session, hear about the study, its history, and the forthcoming toolkit; preliminary social impact data from MOMSI host museums; and host museum perspectives on how to recruit participants through an equity lens and use social impact data for master and strategic planning, advocacy, and community engagement.

              \subsection*{Session Information}
                \begin{description}
                  \item [Format:] Regular session/panel (roundtable, single speaker, etc.)
							    
							    \item [Uniqueness:]Museums need a systematic way to measure social impact. This session shares how MOMSI is creating, testing, and sharing a social impact toolkit.
							    \item [Objectives:]Through this session, attendees will:

(1) Understand current developments in measuring museum social impact. Attendees will hear about our project’s history and the host museum selection process, briefly explore the social impact survey we’re testing, and see what social impact data looks like with preliminary findings from the study. 
(2) Explore how museums are using social impact to advance their work.  Attendees will have time to hear from three presenters on how their museums used a museum-wide equity-driven process to recruit study participants, are engaging communities in strategic planning through the study, and using social impact data for master planning and advocacy.
(3) Learn about a social impact toolkit that will be available summer 2023. Attendees will be provided the opportunity to offer feedback and comment on the forthcoming social impact toolkit. Their feedback and participation will guide the project team’s formation of the social impact toolkit, ensuring it and the resources included are useful and relevant to a broad range of museums.
							    \item [Engagement:]Using the Speed Geeking technique, each presenter will share their perspective and then facilitate a discussion with attendees around one of three topic areas: recruiting participants through an equity lens, community engagement/using social impact data for strategic planning, and using social impact data for advocacy. At the end of the session, attendees will share what aspects of a toolkit would be most beneficial to help them independently measure social impact in the future.
							    \item [Relationship to Theme:]Museums often talk about social impact, but how do they know it’s happening? This forward-thinking national study has made progress in creating a social impact survey that will help museums launch into the future. Knowing what impacts museums make helps inform choices in strategic planning, programming, hiring, and outreach. We will inspire museums to consider how to take ownership of learning about their social impact, and what changes they can make to their museum practices.
							    
                \end{description}
              \subsection*{Audience}
                \begin{description}
                  \item [Audiences:]Development and Membership Officers~Directors/Executive/C-Suite~Diversity and inclusion specialists~Evaluators~General Audience~
                  \item[Professional Level:]All levels~
                \item[Scalability:] MOMSI host museums were strategically chosen to represent every region of the US, museums of varying size (including staff, budget, and visitorship), and varying content areas (from art museums to zoos) all to help inform the social impact toolkit (coming 2023). We understand museum staff capacity is already at its limit, and carrying out such work, while important, needs to come with as many supporting documents and resources as possible. We hope that while our work will inspire attendees to understand and measure social impact, we also hope to gain a deeper understanding of what attendees will need to do this kind of work at their museum so we can create a relevant and usable toolkit.

							
              \end{description}
            \subsection*{Participants}
              \subsubsection*{ Michelle Mileham }
              Submitter, Moderator, Presenter\newline
              Project Manager, Measurement of Museum Social Impact\newline
              Utah Division of Arts \& Museums, Salt Lake City, UT
              \newline
              mmileham@utah.gov\newline
              
              

              As the Project Manager for MOMSI, Michelle is suited to provide an overarching background of the project and keep conversations focused on what has been relevant to date in this national study. Knowing what other host museums have experienced throughout this project also allows Michelle to share more details beyond the perspectives of the presenting host museums.\newline


              

              
                \subsubsection*{ Emily  Johnson }
                Presenter\newline
                Field Services Manager\newline
                Utah Division of Arts \& Museums, Salt Lake City, UT
                \newline
                emilyjohnson@utah.gov\newline
                
                

                Emily was the project manager for the Utah state social impact pilot study, on which MOMSI was based. She brings expertise and drive to learn how small museums can measure their social impact.
                \emph{ (confirmed) }
              

              
                \subsubsection*{ Dean Watanabe }
                Presenter\newline
                Chief Mission Officer\newline
                Fresno Chaffee Zoo, Fresno, CA
                \newline
                dwatanabe@fresnochaffeezoo.org\newline
                
                

                Brings experience as a MOMSI host museum. Fresno Chaffee Zoo focused on community engagement as a means to recruit visitors to the study, and will use the social impact data to inform their strategic plan.
                \emph{ (confirmed) }
              

              
                \subsubsection*{ Dan Keeffe }
                Presenter\newline
                Director of Learning \& Engagement\newline
                Los Angeles Zoo and Botanical Gardens, Los Angeles, CA
                \newline
                dan.keeffe@lacity.org\newline
                
                

                Brings experience as a MOMSI host museum. Los Angeles Zoo used an equity approach to recruit visitors to the study.
                \emph{ (confirmed) }
              

              
                \subsubsection*{ Lorie Millward }
                Presenter\newline
                VP of Possibilities\newline
                Thanksgiving Point Institute , Lehi, UT
                \newline
                lmillward@Thanksgivingpoint.org\newline
                
                

                Thanksgiving Point launched the first social impact study in the state of Utah. Lorie will share how they continue to use social impact data to drive institution-wide decisions, master planning, and advocacy efforts.
                \emph{ (confirmed) }
              
        
          \newpage
          \section{ Land Acknowledgments: Beyond the Writing on the Wall }
            \begin{description}
              \item [ID:]
              WMA2022\_184

              \item [Assigned to:]
                \item [Track:]
              \end{description}

              <strong>Are you considering developing a Land Acknowledgment? Join us for an honest conversation about the challenges and benefits of this initiative. We will share various institutional and tribal perspectives about proper processes and expected and realized outcomes. The session will offer reflection and inspiration on the appropriateness of such statements, possible uses, and best practices for planning and implementation, as well as provide practical advice about partnership building with tribal communities.</strong>

              \subsection*{Session Information}
                \begin{description}
                  \item [Format:] Regular session/panel (roundtable, single speaker, etc.)
							    
							    \item [Uniqueness:]Land acknowledgments are becoming de riguer, but may not reflect tribal communities’ sovereignty over, deep connectivity with, or modern stewardship of ancestral lands.
							    \item [Objectives:](1) Objective 1: to share examples of museums deepening tribal relations 
(2) Objective 2: to share examples of how organizations are stepping up service to the public through historical transparency and truth
(3) Objective 3: to provide honest dialogue and practical guidelines around the purpose, utility, and benefits/harms of land acknowledgement statement creation and implementation
 
Learning Outcomes: Attendees will learn from this session: 
 
Why it is critical to collaborate with local tribes to enter into relationship and how to build upon that relationship to explore land acknowledgement within a museum space.
 
How to develop and present a land acknowledgment in the most compelling and meaningful way, including employ of indigenous language.
 
Practical steps to take around tribal consultation, content collaboration, and related program development when exploring or developing a land acknowledgement statement.
							    \item [Engagement:]Session will require a bulletin board and post-its where audience can question as they arrive, to be answered during Q\&A .
Moderator will facilitate a conversation with the speakers and between the speakers. Each presenter will also present and  take some questions from the bulletin board. The moderator will act as a liaison between speakers, and will facilitate the flow of the session.
							    \item [Relationship to Theme:]Land acknowledgements, when properly developed and employed, can be an incredible example of FORWARD movement in museum practice. It is because of many museums’ commitment to authenticity, transparency, and social justice that other organizations have been inspired to do truth-telling. This session will reflect on existing work and champion more spaces, places, and people to acknowledge the presence and contribution of indigenous people, as well as their role as rightful stewards over their self-representation.
							    
                    \item [Additional Comments: ]I have myself (as moderator) and one presenter (our tribal partner) confirmed, as well as a non-indigenous cultural museum, but would welcome input/referrals to others that have or are in the process of developing land acknowledgments - Ideally an art or science museum/org --- I would also be happy to merge with another session.
As moderator, I will be asking some in depth questions, since the SBCM and SMBMI worked on the land acknowledgment together. We thought it important to have the tribal representative have more time and space to present from their point of view. 
 
  

                \end{description}
              \subsection*{Audience}
                \begin{description}
                  \item [Audiences:]Board Members~Curators/Scientists/Historians~Directors/Executive/C-Suite~Diversity and inclusion specialists~Educators~Emerging Museum Professionals~
                  \item[Professional Level:]All professional levels~
                \item[Scalability:]   This session will include organizations of different sizes and different missions to show the variety of applications of land acknowledgments. The session will also include practical advice from government-led institutions, tribal leadership, and non-profits, in order to scale the topic to various types of organizations. 
  

							
              \item[Other Comments:] The intended audience for this session is any museum staff person who can affect change within their organization. This conversation will highlight that anyone can begin to change where he/she/they are!
              \end{description}
            \subsection*{Participants}
              \subsubsection*{ Tamara  Serrao-Leiva }
              Submitter, Moderator\newline
              Curator of Anthropology\newline
              San Bernardino County Museum, Redlands
              \newline
              tserrao-leiva@sbcm.sbcounty.gov\newline
              tamserlei@gmail.com\newline
              19097988623\newline

              Tamara Serrao-Leiva has been instrumental in creating strong relationships between the county and tribal members, residents, and leadership. As an archaeologist, Tamara has understood the importance of having transparent conversations with indigenous community members. In her career at the museum, she has been entrusted with liaising with various tribal governments for repatriations, exhibits, and programming. At the beginning on 2020, she began conversations about introducing a land acknowledgment at the SBCM and with the incredible support of leadership and staff members, saw a land acknowledgment installed in the front entrance, in Serrano language...a FIRST in the County’s history.\newline


              
                \subsubsection*{ Tamara Serrao-Leiva }
                Moderator\newline
                Curator of Anthropology\newline
                San Bernardino County Museum, Redlands
                \newline
                tserrao-leiva@sbcm.sbcounty.gov\newline
                tamserlei@gmail.com\newline
                19097988623\newline

                Tamara Serrao-Leiva has been instrumental in creating strong relationships between the county and tribal members, residents, and leadership. As an archaeologist, Tamara has understood the importance of having transparent conversations with indigenous community members. In her career at the museum, she has been entrusted with liaising with various tribal governments for repatriations, exhibits, and programming. At the beginning on 2020, she began conversations about introducing a land acknowledgment at the SBCM and with the incredible support of leadership and staff members, saw a land acknowledgment installed in the front entrance, in Serrano language...a FIRST in the County’s history.\newline
                \emph{ (confirmed) }
              

              
                \subsubsection*{ Lee Clauss }
                Presenter\newline
                Vice President of Cultural and Natural Resources\newline
                San Manuel Band of Mission Indians, Highland, CA
                \newline
                lee.clauss@sanmanuel-nsn.gov\newline
                
                909-633-5851\newline

                Lee Clauss has been instrumental in the partnership between SBCM and SMBMI. Having worked for tribes for two decades, she has an understanding of tribal histories as well as contemporary realities. She is an archaeologist and historic preservationist, with experience in museums and archives. Lee is both pragmatic and aspirational in her approach to partnership building and tribal sovereignty expansion. Her position as Tribal Historic Preservation Officer highlights the forward movement of the Tribe internally within the realm of heritage stewardship, but also externally as a thought-leader and partner in the improvement of the region the tribe calls home.
                \emph{ (confirmed) }
              

              
                \subsubsection*{ Leslie  Anderson }
                Presenter\newline
                Director of Collections, Exhibitions \& Programs\newline
                National Nordic Heritage Museum, Seattle, WA
                \newline
                lesliea@nordicmuseum.org\newline
                
                

                Leslie Anderson is the director of collections, exhibitions, and programs. Her specialization in Danish arts, prior experience, and work with Denmark’s National Gallery and University of Copenhagen give a fascinating perspective on using a land acknowledgment of ancestral land in a non-indigenous cultural institution. Leslie is perfectly positioned to support and expand this session by discussing her outreach, exhibitions, and programs.
                \emph{ (confirmed) }
              

              

              
        
          \newpage
          \section{ Looking Ahead: Museums and Youth Collaborating in Climate Justice }
            \begin{description}
              \item [ID:]
              WMA2022\_185

              \item [Assigned to:]
                \item [Track:]
              \end{description}

              The future of museums as meaningful institutions lies in shifting away from neutrality, addressing social topics of interest. OMSI led an exploratory, symbiotic\_ Youth Advisory Research Board\_ focusing on climate change education. 
The goal of this session is to share lessons learned, best practices, and challenges. This session will be useful to attendees interested in youth programming and social action. Key aspects explored are program design and development, youth as researchers/advisors, and youth-created programming.

              \subsection*{Session Information}
                \begin{description}
                  \item [Format:] Regular session/panel (roundtable, single speaker, etc.)
							    
								  \item [Fee:]0
							     
							    \item [Uniqueness:]Participants will interact directly with youth in open, frank discussions.
							    \item [Objectives:]The worsening climate crisis has pushed museums to rethink their role in climate awareness, turning that awareness into action. Inspired by the lead that youth have taken in climate action, the Oregon Museum of Science and Industry (OMSI) has explored a symbiotic collaboration model for museums and youth regarding climate change and potentially broader themes of socioeconomic interest.

The innovative “YARB,” or Youth Advisory Research Board model, combines Youth Advisory Boards (YABs), which support youth voice implementation in museums, and Youth Participatory Action Research (YPAR), which supports youth in designing and conducting research of their interest. Thus, YARBs present an opportunity to learn from youth activism while supporting social movements in ways not normally accessible for youth. While a YARB can address a number of socioeconomic themes, OMSI focused its first YARB program on climate change.

The objectives of this session are:
1. Attendees and facilitators will have a better understanding of the opportunities and potential challenges implementing a YARB in their institutions.
2. Attendees will understand how museums can position themselves in climate action efforts, seeking opportunities that resonate with youth 
3. Attendees will broaden their professional network and build relationships with others interested in engaging with youth researchers.
4. Attendees will discuss and brainstorm ideas of engaging and working with youth in a museum setting
							    \item [Engagement:]Participants will be grouped in sub-groups and be encouraged to ask questions and share any concerns about implementing a YARB in their institutions.
Resources needed are ample space and chairs arranged in groups of 6-8.
							    \item [Relationship to Theme:]In line with the theme FORWARD, this session will be especially useful for institutions interested in becoming more active, adaptable, and responsive to the myriad events that shape the way we think and the themes that move society.
							    
                    \item [Additional Comments: ]In order to accommodate youth who have to attend school, this session would benefit greatly if it was held on the weekend.

                \end{description}
              \subsection*{Audience}
                \begin{description}
                  \item [Audiences:]Educators~Emerging Museum Professionals~Evaluators~Programs~Volunteer Managers~
                  \item[Professional Level:]All levels~
                \item[Scalability:] The YARB model can be adapted to different types of institutions, and YARBs can have from 3 to 15 youth  members. Themes addressed do not have to revolve around climate change, but can address a number of social phenomena that are of interest to youtube such as equity, or freedom of speech. The outcomes of the session will provide learning for professionals that work in different kinds of museums.

							
              \end{description}
            \subsection*{Participants}
              \subsubsection*{ Cyrus Lyday }
              Submitter, Presenter\newline
              Teen Engagement Educator\newline
              Oregon Museum of Science and Industry, Portland, OR
              \newline
              clyday@omsi.edu\newline
              
              5039629615\newline

              Submitter is the direct supervisor for youth climate ambassadors in this project and can speak directly to the program\newline


              

              
                \subsubsection*{ Rebecca Reilly }
                Presenter\newline
                Teen and Adult Engagement Assistant Manager\newline
                Oregon Museum of Science and Industry, Portland, OR
                \newline
                rreilly@omsi.edu\newline
                
                5037974675\newline

                Rebecca is the PI on this NSF funded project and is a secondary supervisor for the youth.
                \emph{ (confirmed) }
              

              
                \subsubsection*{ Fabiana Barrientos }
                Presenter\newline
                Youth Climate Ambassador\newline
                Oregon Museum of Science and Industry, Portland, OR
                \newline
                463591@bsd48.org\newline
                
                

                Fabiana is a member of the youth advisory research board for this project.
                \emph{ (not confirmed) }
              

              

              
        
          \newpage
          \section{ Working with Students and Artists at the University Art Museums }
            \begin{description}
              \item [ID:]
              WMA2022\_186

              \item [Assigned to:]
                \item [Track:]
              \end{description}

              In this session, Aimee Shapiro, Director of Programming and Engagement at the Anderson Collection at Stanford University will present her experience working with artists and students in a collaborative space. University students worked closely with artists to help produce exhibitions and public programs. She will discuss specific examples from working with Nick Cave, The Undocumented Migration Project, Eamon Ore-Giron, Kiyan Williams, and Wendy Red Star.

              \subsection*{Session Information}
                \begin{description}
                  \item [Format:] Regular session/panel (roundtable, single speaker, etc.)
							    
								  \item [Fee:]no
							     
							    \item [Uniqueness:]My session will go through the steps and experiences working with different artists as collaborators with college students. This unique opportunity has given college students experience in the art world, access to living artists, and opportunities to actively participate in performative and art-making workshops.
							    \item [Objectives:]I hope people will leave the session with new ideas for collaboration among artists, institutions, and different audiences, specifically college students. I hope to have an engaging discussion that is more of a conversation and sharing of experiences and questions rather than delivering a traditional lecture. People will discover strategies for working directly with artists, what successful outreach to different audiences looks like, and how working across institutions/departments can set the stage for dynamic exhibitions and programs.
							    \item [Engagement:]I plan to present the Anderson Collection's programs and exhibitions that were  presented as collaborations with students. I will pose questions that will help get people thinking creatively about how they can form programs that bring audiences together with living artists. We will workshop an outline for a potential collaborative program or exhibition for each person's institution.
							    \item [Relationship to Theme:]The projects I will focus on involve all POC artists who work with diverse audiences and address important social and political issues of contemporary society. In addition, work by these artist directly addresses issues of colonialism, racism, and what museums choose to show and not to show. We engage with different student organizations and departments on campus in order to see how we can move forward as a museum and a University, into a more responsible, aware, inclusive space for visitors.
							    
                \end{description}
              \subsection*{Audience}
                \begin{description}
                  \item [Audiences:]Diversity and inclusion specialists~Educators~Emerging Museum Professionals~General Audience~Programs~
                  \item[Professional Level:]All professional levels~
                \item[Scalability:] People can switch out an artist for a historian, scientist or other guest. The main idea is connecting your featured invited guest to college students or other audiences.

							
              \end{description}
            \subsection*{Participants}
              \subsubsection*{ Aimee Shapiro }
              Submitter, Moderator, Presenter\newline
              Director of Programming and Engagement\newline
              Anderson Collection at Stanford University, Stanford, CA
              \newline
              aimees@stanford.edu\newline
              aimees@stanford.edu\newline
              6507216105\newline

              I'm the submitter and the presenter.\newline


              

              
                \subsubsection*{ Aimee Shapiro }
                Presenter\newline
                Director of Programming and Engagement\newline
                Anderson Collection at Stanford University, Stanford, CA
                \newline
                aimees@stanford.edu\newline
                aimees@stanford.edu\newline
                6507216105\newline

                I will be speaking about the projects I have done.
                \emph{ (confirmed) }
              

              

              

              
        
          \newpage
          \section{ Museum on the Move! School Outreach Implementation }
            \begin{description}
              \item [ID:]
              WMA2022\_187

              \item [Assigned to:]
                \item [Track:]
              \end{description}

              School outreach is an important part of engaging students who may never visit your institution, but can be fraught with logistical challenges. Get inspired to update, expand, or implement your school outreach program by hearing from a panel of museum educators with local, regional, and statewide outreach programs.	 Topics of discussion will include scheduling, scalability, accessibility, addressing school standards, and staffing, as well as Q\&amp;A time.

              \subsection*{Session Information}
                \begin{description}
                  \item [Format:] Regular session/panel (roundtable, single speaker, etc.)
							    
								  \item [Fee:]N/A
							     
							    \item [Uniqueness:]I've never seen a school outreach session at a conference, and I'm always interested in more museum education sessions!
							    \item [Objectives:]Key takeaways will include inspiration and new ideas related to the logistics of a school outreach program, how to design an efficient and successful program, and the importance of accessibility in an outreach program.
							    \item [Engagement:]The main opportunity for session participation will be in question and answer time with the panelists. Participants will receive resources related to program logistics.
							    \item [Relationship to Theme:]School outreach moves museums forward by providing a connection between an institution and students who may not ever visit the building. When museum educators visit a school, they remove the barriers of distance and cost and meet the students where they are, expanding engagement and increasing accessibility.
							    
                    \item [Additional Comments: ]I hope to include panelists from a range of museums with school outreach programs that vary in scale and longevity. 

                \end{description}
              \subsection*{Audience}
                \begin{description}
                  \item [Audiences:]Educators~
                  \item[Professional Level:]All professional levels~Emerging Professional~Mid-Career~
                \item[Scalability:] This workshop is intended to be helpful for organizations both with existing school outreach programs and with yet-to-be-implemented programs, as well as plans/resources for local, regional, or statewide outreach.

							
              \end{description}
            \subsection*{Participants}
              \subsubsection*{ Glynis Bawden }
              Submitter, Moderator, Presenter\newline
              School Outreach and Teacher PD Manager\newline
              Natural History Museum of Utah, Utah
              \newline
              gbawden@nhmu.utah.edu\newline
              gbawden@nhmu.utah.edu\newline
              14252753266\newline

              I will contribute my knowledge and experience from seven years of experience as a museum school outreach educator and two years of experience as school outreach manager.\newline


              

              
                \subsubsection*{ Glynis Bawden }
                Presenter\newline
                School Outreach and Teacher PD Manager\newline
                Natural History Museum of Utah, Utah
                \newline
                gbawden@nhmu.utah.edu\newline
                gmbawden@gmail.com\newline
                425-275-3266\newline

                I will contribute my knowledge and experience from seven years of experience as a museum school outreach educator and two years of experience as school outreach manager.
                \emph{ (confirmed) }
              

              

              

              
        
          \newpage
          \section{ Putting the AANHPI into DEAI }
            \begin{description}
              \item [ID:]
              WMA2022\_188

              \item [Assigned to:]
                \item [Track:]
              \end{description}

              The diverse histories, cultures, perspectives, and experiences of the Asian American, Native Hawaiian, and Pacific Islander (AANHPI) diaspora are often missing from or kept to the edges of diversity, equity, accessibility, and inclusion (DEAI) efforts in the museum sector. This session will provide participants an opportunity to consider and reflect upon the narratives that continue to impact AANHPI communities and potential steps that individual and organizational allies can take to improve meaningful inclusion and access.

              \subsection*{Session Information}
                \begin{description}
                  \item [Format:] Regular session/panel (roundtable, single speaker, etc.)
							    
								  \item [Fee:]n/a
							     
							    \item [Uniqueness:]This session is unique in the museum sector in that it focuses on AANHPI communities. It is timely given the increasing hate and violence targeted at AANHPI communities.
							    \item [Objectives:]By the end of this session, attendees will have
1)	Increased awareness of the diversity and experiences of AANHPI communities
2)	Increased understanding of potential steps that individuals can take to support AANHPI communities
3)	Increased understanding of potential steps that museums can take to be more inclusive of AANHPI communities

By the end of this session, attendees potential learning outcomes will include capacity to
1a) Appreciate that the AANHPI diaspora consists of over 50 different ethnic groups that speak over 100 different languages and dialects 
1b) Describe the 4 predominant (and inaccurate) narratives that impact how AANHPIs are perceived
2a) Identify at least one step they can personally take to better support AANHPI communities
3a) Identify at least one step their department or museum can take to improve DEAI efforts for AANHPI communities
							    \item [Engagement:]To ensure maximum audience engagement, presenters will utilize a combination of participant polling, knowledge café, and paired discussions to provide opportunities for participants to surface their own reflections and learn from other participants’ perspectives. 

Resources needed include LCD projector, internet access, and sufficient chairs and space to allow for flexible seating configurations
							    \item [Relationship to Theme:]Over the past year, there have been over 10,000 reported acts of hate and violence against Asian Americans, particularly against women and the elderly. Nearly half of the Asian Americans in the US live in the West. The AANHPI population is estimated to triple in size by 2060. FORWARD-thinking institutions must more meaningfully embrace the past, current, and future of AANHPI communities.
							    
                    \item [Additional Comments: ]Additional details about Session Format
Below is planned session for 1.25 hour session and can be adapted for 1/2 day workshop.
1 - The session will begin with brief didactic overview of key AANHPI community facts, figures, and histories. 
2 - Participants will demonstrate their knowledge and perspectives through Slido (www.sli.do) polling.
3 - Participants will then participate in a knowledge café to discuss what they collectively know and understand about the 4 predominant AANHPI narratives (i.e. model minority myth; treatment as perpetual foreigners; othering and invisibility; hyper-sexualization and de-masculinization).
4 - Presenters will then engage in a series of mini-Ignite talks highlighting experiences and strategies from different AANHPI-focused museums and AANHPI-identified museum leaders.
5 - Participants will then have the opportunity to engage in pairs conversations to synthesize their learnings and identify at least two potential action steps they might take post-workshop.

Additional Presenters
We also hope to add an NHPI presenter. 




                \end{description}
              \subsection*{Audience}
                \begin{description}
                  \item [Audiences:]General Audience~
                  \item[Professional Level:]All levels~
                \item[Scalability:] This session aims to foster a greater understanding of a marginalized and often overlooked population. The outcomes are applicable to cultural organizations of any type and size that serve a public audience. 

							
              \item[Other Comments:] The intended audience for this session is anyone in the museum sector, regardless of level, role, and identity.
              \end{description}
            \subsection*{Participants}
              \subsubsection*{ Edward Tepporn }
              Submitter, Moderator, Presenter\newline
              Executive Director\newline
              AIISF, Oakland, CA
              \newline
              etepporn@aiisf.org\newline
              
              4158461749\newline

              Edward is the Executive Director at AIISF. He has over 25 years of experience in the nonprofit sector focusing on a public health, advocacy, and racial equity with an emphasis on Asian American, Native Hawaiian, and Pacific Islander communities. In April 2021, he authored “It Is Time to Include AANHPIs In Museum Diversity, Equity, Accessibility, and Inclusion Efforts” for the American Alliance of Museums.\newline


              
                \subsubsection*{ Edward Tepporn }
                Moderator, Presenter\newline
                Executive Director\newline
                AIISF, Oakland, CA
                \newline
                etepporn@aiisf.org\newline
                
                415 658 7691\newline

                Edward is the Executive Director at AIISF. He has over 25 years of experience in the nonprofit sector focusing on a public health, advocacy, and racial equity with an emphasis on Asian American, Native Hawaiian, and Pacific Islander communities. In April 2021, he authored “It Is Time to Include AANHPIs In Museum Diversity, Equity, Accessibility, and Inclusion Efforts” for the American Alliance of Museums.\newline
                \emph{ (confirmed) }
              

              
                \subsubsection*{ Edward Tepporn }
                Presenter\newline
                Executive Director\newline
                AIISF, Oakland, CA
                \newline
                etepporn@aiisf.org\newline
                
                4158461749\newline

                Edward is the Executive Director at AIISF. He has over 25 years of experience in the nonprofit sector focusing on a public health, advocacy, and racial equity with an emphasis on Asian American, Native Hawaiian, and Pacific Islander communities. In April 2021, he authored “It Is Time to Include AANHPIs In Museum Diversity, Equity, Accessibility, and Inclusion Efforts” for the American Alliance of Museums.
                \emph{ (confirmed) }
              

              
                \subsubsection*{ Jennifer  Fang }
                Presenter\newline
                Director of Interpretation and Community Engagement\newline
                Pittock Mansion, Portland, OR
                \newline
                jfang@pittockmansion.org\newline
                
                503 998 5108\newline

                Jennifer Fang earned a Ph.D. in Asian American history and has held leadership positions at Chinese and Japanese American museums in Portland. She was recently the guest co-editor of a special issue of the Oregon Historical Quarterly about Oregon’s early Chinese diaspora.
                \emph{ (confirmed) }
              

              
                \subsubsection*{ Joël  Barraquiel Tan }
                Presenter\newline
                Executive Director\newline
                Wing Luke Museum of the Asian Pacific American Experience , Seattle, WA
                \newline
                jtan@wingluke.org\newline
                
                (206) 623-5124\newline

                Joël Barraquiel Tan (siya/he/him) is the Executive Director at the Wing Luke Museum of the Asian Pacific American Experience. He is the author of “Type O Negative” (Red Hen) and various works on identity, AIDS, \& queer politics appear in academic and commercial venues. Joël is a cofounder of LA's Asian Pacific AIDS Intervention Team Health Center and served as the Director of Community Engagement at Yerba Buena Center for the Arts from 2004-2015. Prior to coming to the Wing Luke Museum, he lived in Hawai'i Island and served as Touching the Earth's Director of Social Impact.
                \emph{ (confirmed) }
              

              
        
          \newpage
          \section{ Where to Begin: Q&A Collections Management }
            \begin{description}
              \item [ID:]
              WMA2022\_189

              \item [Assigned to:]
                \item [Track:]Collections~
              \end{description}

              Have a project stumping you? New to collections management/registration and need a sounding board? Then this session is for you! Lead by a panel of well-seasoned collections professionals, this session will share tips and lessons learned, as well as provide an open forum for questions and discussion. Join us for our version of AAM’s \_Collections Conundrums \_to find your way forward.

              \subsection*{Session Information}
                \begin{description}
                  \item [Format:] Regular session/panel (roundtable, single speaker, etc.)
							    
								  \item [Fee:]n/a
							     
							    \item [Uniqueness:]This session provides inexpensive/easy solutions developed by seasoned professionals, while also providing a space for discussion and problem solving. Several minds are greater than one.
							    \item [Objectives:]As the field finds itself on the other side of a global pandemic, we must grapple with how it is reflected in our staff and our everyday roles (loss of staff, professionals in new roles, new professionals in the field, even tighter budgets). This session is meant to provide insights to those starting anew, to those looking for time-saving and cost effective methodologies, and for those who just need other professionals to help develop a solution.
							    \item [Engagement:]As stated, this session is meant to provide insights, time saving and cost effective methodologies, and an opportunity to discuss with others in similar situations. The audience will be expected to ask questions, but the panel will also be ready with additional information and prompting questions, if needed. Resources (handouts and/or links) will also be shared.
							    \item [Relationship to Theme:]The field has been greatly impacted by the pandemic. This session will help collections professionals find a path Forward in their everyday work. This might be reflected in time or cost saving methods or by hashing out/seeking insight on an issue that has halted them.
							    
                    \item [Additional Comments: ]I need help finding at least two other presenters. I have reached out to a few, but have yet to hear word.

                \end{description}
              \subsection*{Audience}
                \begin{description}
                  \item [Audiences:]Curators/Scientists/Historians~Emerging Museum Professionals~General Audience~Registrars, Collections Managers~
                  \item[Professional Level:]Emerging Professional~Mid-Career~Student~
                \item[Scalability:] The panel will have representation from professionals who can discuss all levels of collections management and care (documentation/administration, collections moves, shipping, artwork, weapons, textiles, vehicles, etc.). The focus will be on smaller museums or those with small budgets, but ideas can easily be scaled upward for other institutions.

							
              \end{description}
            \subsection*{Participants}
              \subsubsection*{ Kathleen Sligar }
              Submitter, Moderator, Presenter\newline
              Museum Director and Curator\newline
              Oregon Military Museum, Clackamas, Oregon
              \newline
              kathleen.m.sligar.civ@army.mil\newline
              dalyka@yahoo.com\newline
              541-760-8228\newline

              I will provide a portion of the presentation and serve as a panelist for Q\&A. If I am confused on what a moderator is for/why one is needed, I would be happy to discuss this further to remedy.\newline


              

              
                \subsubsection*{ Kathleen Sligar }
                Presenter\newline
                Museum Director and Curator\newline
                Oregon Military Museum, Clackamas, Oregon
                \newline
                
                
                

                With nearly 20 years of museum collections, preventive conservation and professional training/presentation experience, I am able to share my breadth and depth knowledge with the field.
                \emph{ (confirmed) }
              

              

              

              
        
          \newpage
          \section{ Creative Attention: Art and Community Restoration }
            \begin{description}
              \item [ID:]
              WMA2022\_190

              \item [Assigned to:]
                \item [Track:]
              \end{description}

              How can museums support individual and community wellness, belonging, and resilience? Hear a case study from the Palo Alto Art Center about <em>Creative Attention,</em> an initiative that included an exhibition, artist residencies, an art therapy residency, public programs, and wellness programs. As part of the session, participate in a virtual meditation with our wellness program provider and use one the prompts created by our art therapist in an artmaking session.

              \subsection*{Session Information}
                \begin{description}
                  \item [Format:] Regular session/panel (roundtable, single speaker, etc.)
							    
								  \item [Fee:]n/a
							     
							    \item [Uniqueness:]Even before COVID, Americans have been living in an age of anxiety. Now with COVID, we are faced with even greater challenges surrounding our physical and psychological health and well-being. Museums have an opportunity and responsibility to respond.
							    \item [Objectives:]Participants will:
 
1)      Learn about one example of how a museum created an initiative around healing and resilience.
2)      Consider ways to incorporate wellness programs and art therapy into a museum context. 
3)    Leave refreshed, revived, and engaged after meditating and making art!
							    \item [Engagement:]This session will include a virtual meditation to start, a presentation of the program, and then a hands-on artmaking session (materials provided) inspired by a prompt from an art therapist, with opportunities at the end for share out.
							    \item [Relationship to Theme:]Museums have a responsibility and an opportunity to help our entire community move forward into recovery. This session will showcase how one museum accomplished this, through an exhibition, artist residency, art therapist residency, wellness and other public programs.
							    
                \end{description}
              \subsection*{Audience}
                \begin{description}
                  \item [Audiences:]General Audience~
                  \item[Professional Level:]All professional levels~
                \item[Scalability:]    
 
Applicable to all types of organizations. 
  

							
              \end{description}
            \subsection*{Participants}
              \subsubsection*{ Karen Kienzle }
              Submitter, Moderator, Presenter\newline
              Director\newline
              Palo Alto Art Center, Palo Alto
              \newline
              karen.kienzle@cityofpaloalto.org\newline
              karen.kienzle@cityofpaloalto.org\newline
              6506173535\newline

              Project Director for the Creative Attention project.\newline


              

              
                \subsubsection*{ Julie  Forbes }
                Presenter\newline
                Stress Management Consultant\newline
                n/a, San Luis Obispo, CA
                \newline
                
                
                

                To provide a virtual meditation session
                \emph{ (not confirmed) }
              

              

              

              
        
          \newpage
          \section{ Commemorating the 250th in the West }
            \begin{description}
              \item [ID:]
              WMA2022\_191

              \item [Assigned to:]
                \item [Track:]Leadership/Careerpath~
              \end{description}

              **The United States will commemorate its 250th anniversary in 2026. How do history organizations participate in this important anniversary when they are not on the East Coast? How do the perspectives of indigenous people and immigrants fit into this commemoration? This session will allow attendees to discuss how museums in the west can connect to this national anniversary using the five themes outlined in AASLH’s <strong>\_<a href="http://download.aaslh.org/Making+History+at+250+Field+Guide.pdf">Making History at 250: The Field Guide for the Semiquincentennial</a>.</strong> **\_

              \subsection*{Session Information}
                \begin{description}
                  \item [Format:] Regular session/panel (roundtable, single speaker, etc.)
							    
							    \item [Uniqueness:]This session is relevant as it is the time for museums to start planning for America250. The session will be innovative by guiding conversation about the commemoration using the Field Guide.
							    \item [Objectives:]Start a regional discussion about how history organizations in the west can connect to America250.
Attendees will understand the five themes created for exploring the 250th and know how to use the Field Guide.
Attendees will feel that their organization can be a part of America250 and be better prepared to plan for 2026.
							    \item [Engagement:]Our intended audience is people responsible for long range organizational and/or program planning who can set the direction for their plans for America250. Attendees will be expected to brainstorm ideas in small groups about how an organization like theirs can connect to a commemoration about the founding of the United States.
							    \item [Relationship to Theme:]This session is made for the theme “Forward”. It encourages future planning and programming for an anniversary that will hopefully move the entire history museum field forward as the Bicentennial did.
							    
                    \item [Additional Comments: ] **We have speakers/facilitators: Jennifer Ortiz, Director,** **Utah Division of State History; Jennifer Kilmer, Director, Washington State Historical Society, Liz Hobson, Museum Director, Idaho State Museum, and Bethany Hawkins, Chief of Operations, American Association for State and Local History**
** **
  

                \end{description}
              \subsection*{Audience}
                \begin{description}
                  \item [Audiences:]Directors/Executive/C-Suite~
                  \item[Professional Level:]Senior Level~
                \item[Scalability:] The information we will share will be centered around five historical themes that provide a structure for attendees to think about the 250th. The interpretation and delivery of those themes will depend on the resources and mission of the individual organizations.

							
              \end{description}
            \subsection*{Participants}
              \subsubsection*{ Bethany Hawkins }
              Submitter, Moderator, Presenter\newline
              Chief of Operations\newline
              American Association for State and Local History, Nashville, TN
              \newline
              hawkins@aaslh.org\newline
              
              615-320-3203\newline

              Bethany will provide the introduction to the Field Guide and a brief overview of the state of planning for the 250th nationwide.\newline


              

              
                \subsubsection*{ Jennifer  Ortiz }
                Presenter\newline
                Director\newline
                Utah Division of State History, Salt Lake City, UT
                \newline
                jennifer.ortiz@utah.gov\newline
                
                

                Jennifer is a member of the AASLH Council and has been passionate about expanding the discussion of America250 to the western states.
                \emph{ (confirmed) }
              

              
                \subsubsection*{ Liz Hobson }
                Presenter\newline
                Museum Director\newline
                Idaho State Museum, Boise, ID
                \newline
                liz.hobson@ishs.idaho.gov\newline
                
                

                Liz leads a state museum and will be in charge of planning 250 exhibits and programs for her organization.
                \emph{ (confirmed) }
              

              
                \subsubsection*{ Jennifer Kilmer }
                Presenter\newline
                Director\newline
                Washington State Historical Society, Tacoma, WA
                \newline
                jennifer.kilmer@wshs.wa.gov\newline
                
                

                Jennifer is part of the AASLH 250 Coordinating Committee and worked to help develop the five themes for the Field Guide.
                \emph{ (not confirmed) }
              

              
        
          \newpage
          \section{ Empathetic Guest Service: Acknowledging the Wholeness of Staff and Visitors }
            \begin{description}
              \item [ID:]
              WMA2022\_192

              \item [Assigned to:]
                \item [Track:]
              \end{description}

              In this session we’ll outline an approach to guest service that centers the wholeness of individuals. Empathetic guest service is proactive, transparent, consistent, and compassionate. It does not mean always agreeing with visitors, but rather that you understand that their perspectives and feelings are valid. We will share lessons from a tumultuous period of heightened social stress, and how this approach has impacted how we resolve conflict and influenced an employee-centered decision-making process.

              \subsection*{Session Information}
                \begin{description}
                  \item [Format:] Regular session/panel (roundtable, single speaker, etc.)
							    
							    \item [Uniqueness:]This approach outlines a different modality of “guest service,” especially in its recognition of staff as guests, and by recognizing emotional impacts of guest interactions.
							    \item [Objectives:]1.  Learn the basics of the empathetic guest service approach and what the possible impacts on organizational culture and guest experience may be.
2. Learn about tips, strategy, and resources that can be used to implement and tailor this approach at you organization.
3. Learn from peers on examples of when this approach has helped or could have improved the outcome of a guest interaction.
							    \item [Engagement:]The intended audience is other frontline museum staff, as well as those who manage, train, and make policy for guest service in museums.  We plan on offering open discussion, sharing of case studies, training materials, and examples of resources we have used to engage guest services staff in this approach!
							    \item [Relationship to Theme:]This approach to guest services aims to empower front-line staff, recognize the wholeness of guests as individuals, open pathways to healing in the museum space, and hopes to offer a more sustainable pathway forward for guest services staff and for how we engage with visitors in museum spaces.
							    
                    \item [Additional Comments: ]We'd be willing to merge with a similar session!

                \end{description}
              \subsection*{Audience}
                \begin{description}
                  \item [Audiences:]Development and Membership Officers~Emerging Museum Professionals~General Audience~HR Personnel~Visitor Services~Volunteer Managers~
                  \item[Professional Level:]All professional levels~
                \item[Scalability:] This approach is about how we treat ourselves and all museum guests. It is scalable to organizations of any size given people are interested and willing to try and reflect on how it’s working! 

							
              \end{description}
            \subsection*{Participants}
              \subsubsection*{ Sarah Winkowski }
              Submitter, Presenter\newline
              Visitor Services Supervisor\newline
              Burke Museum of Natural History and Culture, Seattle, Wa
              \newline
              swinko@uw.edu\newline
              
              17573233815\newline

              I worked with Valerie Roberts and Visitor Services staff to co-create and implement the concept of Empathetic Guest Service at the Burke Museum. I also advocated for a departmental focus on employee experience, empowerment and professional and personal development.\newline


              
                \subsubsection*{ Kate Fernandez }
                Moderator, Presenter\newline
                Former Director of Interpretation + Visitor Experience\newline
                Burke Museum of Natural History and Culture, Seattle, WA
                \newline
                seekaterun@gmail.com\newline
                
                206.255.6305\newline

                As the director of the department Kate oversaw the creation of the visitor experience department and was an advocate for systemic change at the leadership level.\newline
                \emph{ (confirmed) }
              

              
                \subsubsection*{ Valerie Roberts }
                Presenter\newline
                Former Visitor Services Manager\newline
                Burke Museum of Natural History and Culture, Seattle, Wa
                \newline
                valerie.e.roberts17@gmail.com\newline
                
                541-292-3003\newline

                Valerie and Sarah built the concept of Empathetic Guest Service at the Burke Museum, as well as created a Visitor Services department focused on employee experience, empowerment and professional and personal development.
                \emph{ (confirmed) }
              

              

              

              
        
          \newpage
          \section{ Using empathy to create meaningful outreach programs for underserved communities }
            \begin{description}
              \item [ID:]
              WMA2022\_193

              \item [Assigned to:]
                \item [Track:]
              \end{description}

              Museum programs should be accessible to all, yet they are treated like a luxury. Barriers to participation have been exacerbated in the last year. To reach out to underserved or marginalized communities, it is important to co-create programs with organizations already serving these populations, and create authentic partnerships that are founded in humility and service. Our organizations learned to embrace their vulnerability as we created spaces for transformative learning experiences without prescribing their outcomes. 

              \subsection*{Session Information}
                \begin{description}
                  \item [Format:] Regular session/panel (roundtable, single speaker, etc.)
							    
								  \item [Fee:]n/a
							     
							    \item [Uniqueness:]This session is about exercising organizational empathy, co-creating learning, challenging orthodoxies and honoring diverse learners, at a time when inequalities are larger than ever.
							    \item [Objectives:]-    Participants will understand the existing and new barriers to visitation for underserved and marginalized audiences
-    Participants will apply lessons learned by the presenting organizations to their own institutions
-    Participants will analyze the priorities and needs of their own community partners to create programs that are uniquely relevant to them
							    \item [Engagement:]In the spirit of empathy toward the intended audience, which is the topic of this session, it will have a high level of audience input. Instant surveys will allow participants to collectively decide where the presentation goes, and to "choose their own adventure". Presenting organizations will share case studies, their challenges, and lessons learned with participants, and challenge them to apply them to their own organizations; collect participants' experiences to collectively decide on best practices.
							    \item [Relationship to Theme:]We can only grow our mission by reaching out to more underserved audiences, not only the traditional elites. Offering meaningful programs to underserved audiences adds social value and well-being, empowers participants, increases self-efficacy, and can help overcome social gaps between communities.
							    
                    \item [Additional Comments: ]Session format: A session based on principles of collective intelligence, where the presenting organizations serve as a catalyst for reflection and expression, and collect information from all audience members. Ideal formats for this session are questionstorming, Spectrogram World Café.

                \end{description}
              \subsection*{Audience}
                \begin{description}
                  \item [Audiences:]Diversity and inclusion specialists~Educators~General Audience~Programs~
                  \item[Professional Level:]All professional levels~
                \item[Scalability:] This session is about challenging colonial constructs, embracing vulnerability and using empathy to be authentic within each organization’s community, regardless of what it may be. The case studies presented serve as a catalyst for each organization to conduct its own reflection about its practices.

							
              \end{description}
            \subsection*{Participants}
              \subsubsection*{ Axel Estable }
              Submitter, Moderator, Presenter\newline
              Curator of Curiosity\newline
              Thanksgiving Point, Lehi, UT
              \newline
              aestable@thanksgivingpoint.org\newline
              
              801.768.7431\newline

              Thanksgiving Point has engaged in a variety of community outreach programs, often funded by grants. We have learned from our mistakes, challenged our orthodoxies, and designed processes to reach out to communities with vulnerability and curiosity. Axel has been an advocate for removing barriers and increasing accessibility to programs since he started four years ago.\newline


              

              
                \subsubsection*{ Hillary Spencer }
                Presenter\newline
                Chief Executive Officer\newline
                The Bishop Museum of Science and Nature, Bradenton, FL
                \newline
                HSpencer@bishopscience.org\newline
                
                941.216.3464\newline

                Hillary uses organizational risk-taking as a management doctrine, challenges conventions and is a champion of disruptive innovation.
                \emph{ (confirmed) }
              

              
                \subsubsection*{ Tracey  Collins }
                Presenter\newline
                Director of Education and Community Engagement\newline
                Natural History Museum of Utah, Salt Lake City, UT
                \newline
                tcollins@nhmu.utah.edu\newline
                
                801.587.5713\newline

                Tracey Collins oversees the Natural History Museum of Utah’s education services and state-wide community outreach. With 19 years of experience working in museum education, she has focused on informal and formal curriculum development, teacher professional development, establishing community/school partnerships, and building a teaching and learning culture for educational equity and inclusion. In her current role, she provides administrative support to the museum’s education and community engagement programs, including school programs, adult and gallery programs, child and family programs, volunteer programs, community outreach, digital education initiatives, and teen programs. She also supports NHMU’s Indigenous Advisory Committee and Community Relations Committee. In 2016 she became an iPAGE Leader for Equity in Informal STEM Institutions from the Science Museum of Minnesota and in 2020 was awarded the Online Excellence Staff Award from the University of Utah.
                \emph{ (confirmed) }
              

              
                \subsubsection*{ Annie  Burbidge Ream }
                Presenter\newline
                Co-Director of Learning and Engagement\newline
                Utah Museum of Fine Arts, Salt Lake City, UT
                \newline
                annie.burbidge.ream@umfa.utah.edu\newline
                
                801.585.5168\newline

                Annie Burbidge Ream oversees learning and engagement in a shared leadership model that facilitates experiential programs for families, K-12 teachers and their students, adult learners, university students, and general museum visitors. Throughout her fifteen years of museum education experience, Annie's focus has moved from being art and object-centered, to utilizing art to create programs that strive to be community-centered. Annie actively participates in developing meaningful partnerships and collaborations, curricula development, visioning and strategic planning, action- and value-centered budgeting, institutional leadership, anti-racism work, and breaking down barriers to create access to the dynamic world of art, expression, and creativity.
                \emph{ (confirmed) }
              

              
        
          \newpage
          \section{ Getting Started with Disaster Planning: A Framework to Build On }
            \begin{description}
              \item [ID:]
              WMA2022\_194

              \item [Assigned to:]
                \item [Track:]
              \end{description}

              Threats to cultural institutions and their collections can be unpredictable and preparing for any eventuality is an important part of collections stewardship. Step one is to come up with a flexible plan that provides clear instructions in the event of a disaster. This session will walk participants through the construction of an emergency response plan and provide them with supporting resources and a framework to build a robust and flexible disaster plan for their institution.

              \subsection*{Session Information}
                \begin{description}
                  \item [Format:] Regular session/panel (roundtable, single speaker, etc.)
							    
							    \item [Uniqueness:]Participants explore a framework for building a disaster plan and should be able to get the work started, plus have the tools to continue onwards.
							    \item [Objectives:]This session empowers participants to complete the critical stages of developing an emergency preparedness and response plan to become more prepared for the unexpected.
·      We will touch on the prerequisite stages of writing an emergency response plan, such as conducting a risk assessment that identifies and evaluates potential threats based on an organization’s region, building, collections, and history. 
·      The session walks participants through the elements of writing an emergency response plan, including forming a response team, creating salvage priorities, putting together a supplies kit, drafting salvage procedures, and making floor plans. 
·      Participants finish with an emergency response plan template and having the knowledge necessary to move forward with the planning process.
							    \item [Engagement:]The audience should be prepared to interact with the presenter and each other. The presenter will pose questions to the audience to prompt active discussion. Participants will also be provided with a worksheet for outlining an emergency response plan that they can start to fill out during the session.
							    \item [Relationship to Theme:]As we start to see a possible end to the pandemic, it is important to learn from our experience and prepare for other disasters that will come in the future. Having a flexible and sustainable emergency response plan ready will help institutions to weather what may come, as well as provide a path forward to recovery.
							    
                    \item [Additional Comments: ]This session will be participatory in style, with the presenter walking participants through the construction of an emergency response plan. Participant will be encouraged to share their own experiences as well as ask for guidance from the presenter and fellow participants.

This session could be merged with another session if needed. It is currently programmed for one presenter, but can be adapted to another format. I had a conversation with Cory Gooch and we determined that our proposal plans were different enough that they could be separate sessions.

                \end{description}
              \subsection*{Audience}
                \begin{description}
                  \item [Audiences:]Facilities Management Personnel~General Audience~Other~Registrars, Collections Managers~
                  \item[Professional Level:]All levels~All professional levels~
                \item[Scalability:] Scalable to all types of institutions of all sizes.

							
              \item[Other Comments:] This session is geared to anybody who wishes to form an emergency response plan for their organization. An understanding of their organizational structure and available resources coming into the session is helpful, but not essential. Participants should be willing to dig into their institution’s inner workings to come up with a realistic and sustainable plan.
              \end{description}
            \subsection*{Participants}
              \subsubsection*{ Tara Puyat }
              Submitter, Moderator, Presenter\newline
              Preservation Specialist\newline
              NEDCC | Northeast Document Conservation Center, Springfield. OR
              \newline
              tespuyat@gmail.com\newline
              tpuyat@nedcc.org\newline
              (541)232-4992\newline

              Tara Puyat has been an instructor for Disaster Preparedness programs developed and presented by NEDCC. As a Preservation Specialist based on the west coast, she is cognizant of issues specific to the region. Her prior positions were in museum collections managements, which gives her insight into the needs of small to midsize cultural institutions.\newline


              

              
                \subsubsection*{ Tara Puyat }
                Presenter\newline
                Preservation Specialist\newline
                NEDCC | Northeast Document Conservation Center, Springfield, OR
                \newline
                tespuyat@gmail.com\newline
                tpuyat@nedcc.org\newline
                (541)232-4992\newline

                Tara Puyat has been an instructor for Disaster Preparedness programs developed and presented by NEDCC. As a Preservation Specialist based on the west coast, she is cognizant of issues specific to the region. Her prior positions were in museum collections managements, which gives her insight into the needs of small to midsize cultural institutions.
                \emph{ (confirmed) }
              

              

              

              
        
          \newpage
          \section{ Collaboration with Native Museums--Moving Forward and Beyond Land Acknowledgements and Decolonization  }
            \begin{description}
              \item [ID:]
              WMA2022\_195

              \item [Assigned to:]
                \item [Track:]Indigenous~
              \end{description}

              It\&rsquo;s easy to get bogged down in land acknowledgements and decolonization discussions.  For some, these are unnecessary hurdles preventing amazing collaborations!  Mingei International Museum, The New Children\&rsquo;s Museum, and Barona Cultural Center \&amp; Museum moved forward creating amazing collaborations benefitting all.  Collaborations with Native communities include difficult discussions, educational journeys, lots of listening, and lots of coffee, but end up in success.  Pick up some ideas from us to implement at home.

              \subsection*{Session Information}
                \begin{description}
                  \item [Format:] Regular session/panel (roundtable, single speaker, etc.)
							    
								  \item [Fee:]N/A
							     
							    \item [Uniqueness:]We moved past political hurdles and focused on how the Mingei and NCM can support Barona Museum and offering space for the Native perspective.
							    \item [Objectives:]1. Land Acknowledgements and Decolonizing are not prerequisites for working with a Native community/Native Museums.  Panelists intend to discuss culture changes and progressive mindsets in their museums that led up to these collaborations.  Participants will assess how "ready" they might be to do this kind of work and their museum's strengths and weaknesses.
2. In this case, collaborators are museum educators who have the support of administration but it is a personal journey as well.  Panelists will discuss personal growth and participants will undertake a thoughtful self-evaluation.
3. Collaborations are definitely challenging--panelists will demonstrate the skills and recognize the benefits of working collaboratively especially with regard to issues of significance to Native communities.  Participants will develop an awareness and responsibility towards a shared humanity and issues of significance to the Native community.
							    \item [Engagement:]Participants will be asked to jot down in a word cloud, adjectives they think describe working with a Native museum/community.  We expect words like, "scary" or "intimidating" or "difficult".  At the end of the presentation, we'll ask to see if any of the adjectives have changed (change in biases) like, "interesting" or "possible" or "amazing experience".  We will also field questions from the participants.
							    \item [Relationship to Theme:]FORWARD--museums have to keep moving forward.  Forward progression is necessary but also scary.  If we can remove the hurdles that might make it scary, moving forward in collaborative ways is exciting and beneficial.  Our three museums set politics aside and moved forward with collaborations that benefitted the People--it's about the People and their voice.
							    
                    \item [Additional Comments: ]Happy to merge if needed but we have 3 presenters lined up.

                \end{description}
              \subsection*{Audience}
                \begin{description}
                  \item [Audiences:]Curators/Scientists/Historians~Educators~Programs~
                  \item[Professional Level:]All levels~
                \item[Scalability:] Any size/type of museum can work with a Native museum/community.  Common ground can be found.  Barona Cultural Center \& Museum is very small but partnered with two large and very different art museums--neither of which have any Kumeyaay collections, but want to include the Kumeyaay voice.    

							
              \end{description}
            \subsection*{Participants}
              \subsubsection*{ Laurie Egan-Hedley }
              Submitter, Moderator, Presenter\newline
              Director/Curator\newline
              Barona Cultural Center \& Museum, Lakeside, CA
              \newline
              lhedley@barona-nsn.gov\newline
              
              619-443-7003\newline

              I will moderate and present on behalf of Barona Cultural Center \& Museum's perspective on the collaborations with Mingei International Museum and The New Children's Museum.\newline


              
                \subsubsection*{ Laurie Egan-Hedley }
                Moderator, Presenter\newline
                Director/Curator\newline
                Barona Cultural Center \& Museum, Lakeside, CA
                \newline
                lhedley@barona-nsn.gov\newline
                
                619-443-7003\newline

                Ms. Egan-Hedley is the link between the two presenters.  As Director/Curator of Barona Cultural Center \& Museum, she worked with each of the other 2 co-presenters and knows the session material very well.\newline
                \emph{ (confirmed) }
              

              
                \subsubsection*{ Laurie Egan-Hedley }
                Presenter\newline
                Director/Curator\newline
                Barona Cultural Center \& Museum, Lakeside, CA
                \newline
                lhedley@barona-nsn.gov\newline
                
                619-443-7003\newline

                Ms. Egan-Hedley led the collaborations with each co-presenter
                \emph{ (confirmed) }
              

              
                \subsubsection*{ Charles Thunyakij }
                Presenter\newline
                Education and Community Programs Specialist\newline
                Mingei International Museum, San Diego, CA
                \newline
                cthunyakij@mingei.org\newline
                
                619-704-7527\newline

                led the collaboration on behalf of the MIngei International Museum
                \emph{ (confirmed) }
              

              
                \subsubsection*{ Kurosh Yahyai }
                Presenter\newline
                Studios Manager\newline
                The New Children's Museum, San Diego, CA
                \newline
                kyahyai@thinkplaycreate.org\newline
                
                619-795-1570\newline

                led the collaboration on behalf of The New Children's Museum
                \emph{ (confirmed) }
              

              
        
          \newpage
          \section{ Envision a Thematic Teen Science Café  }
            \begin{description}
              \item [ID:]
              WMA2022\_196

              \item [Assigned to:]
                \item [Track:]
              \end{description}

              Teen Science Café programs are free, fun ways for teens to engage in lively conversations and participate in STEM activities. Individual café sites have explored everything from astronomy to zoology, but some sites have created highly focused cafes. Learn about how to start a Teen Science Café focused on your museum’s mission, create a vision board of your ideal Teen Science Café, and walk away with a draft implementation plan to start your own thematic café.

              \subsection*{Session Information}
                \begin{description}
                  \item [Format:] Regular session/panel (roundtable, single speaker, etc.)
							    
							    \item [Uniqueness:]The vision board will serve as a guide point for launching their café. It is a model to use with youth leaders.
							    \item [Objectives:]Objective 1: Understand the Teen Science Café model and how it can support a museum’s mission.
Objective 2: Envision what a Teen Science Café program looks like in your institution and region.
Objective 3: Draft an Implementation Plan for starting a Teen Science Café program.
							    \item [Engagement:]Following a brief overview of the Teen Science Café program, attendees will have time to visually brainstorm what their own café would look like at their institution. They will do this by flipping through magazines and other printed materials to find words and pictures that describe their ideal café. They will also leave with instructions on how to do this activity with their new youth leaders.
							    \item [Relationship to Theme:]The Teen Science Café programs typically highlight new, innovative science by asking presenters to focus on their research. Since students select the topics they are interested in, the presentations tend to be focused on current events or contemporary issues.
							    
                \end{description}
              \subsection*{Audience}
                \begin{description}
                  \item [Audiences:]Educators~Programs~
                  \item[Professional Level:]All professional levels~
                \item[Scalability:] The Teen Science Café Network has supported a variety of organizations in starting new Teen Science Cafes. From small clubs in rural Missouri to regional, multi-site organizations, the Teen Science Café model can be scaled to all organization types and sizes. 

							
              \end{description}
            \subsection*{Participants}
              \subsubsection*{ Carolyn Noe }
              Submitter, Presenter\newline
              Director\newline
              Institute for Health Innovation, Northern Kentucky University, Highland Heights
              \newline
              noec2@nku.edu\newline
              carolyn.noe@gmail.com\newline
              9548152457\newline

              As a Resource Guide for the Teen Science Café Network, Carolyn can speak directly to launch of a new Teen Science Café. Carolyn also leads a health-focused Teen Science Cafés in urban and rural northern Kentucky and can help guide the conversation toward establishing thematic cafes.\newline


              
                \subsubsection*{ Michelle Hall }
                Moderator\newline
                President\newline
                Science Education Solutions, Los Alamos, NM
                \newline
                hall@scieds.com\newline
                
                859-572-6332\newline

                Michelle is the co-founder of the Teen Science Café Network.\newline
                \emph{ (confirmed) }
              

              
                \subsubsection*{ Bree Oatman }
                Presenter\newline
                Education Director\newline
                South Dakota Discovery Center, Pierre, SD
                \newline
                breeoatman@sd-discovery.org\newline
                
                605-580-1710\newline

                Bree is also a Resource Guide for the Teen Science Café Network and built the professional development for all new Teen Science Café sites.
                \emph{ (confirmed) }
              

              

              

              
        
          \newpage
          \section{ Museums and the Fight Against Hate (working title) }
            \begin{description}
              \item [ID:]
              WMA2022\_197

              \item [Assigned to:]
                \item [Track:]
              \end{description}

              Now, more than ever, museums have the opportunity to lead efforts in the fight against enmity and injustice. Whether through neighborhood interactions, programs, exhibitions, collecting, or educational efforts, our organizations are public centers for deterring hate and promoting diversity, understanding, and respect. Panelists from four culturally specific institutions will consider new insights and approaches for facing the immediate future, drawing upon their institutional approaches to the abovenamed issues.
** **

              \subsection*{Session Information}
                \begin{description}
                  \item [Format:] Regular session/panel (roundtable, single speaker, etc.)
							    
							    \item [Uniqueness:]Our fraught political world compels us to confront intolerance with heightened urgency and new insight. Panelists will discuss why and how our museums must be prepared to be a lead voice.
							    \item [Objectives:]1.    Participants learn how museums can match their subjects with today’s societal issues
2.    Participants learn approaches that stretch boundaries and are complex and impactful.
Participants reconsider the responsibilities of their own institutions -- is it our responsibility to promote the standards of a community and recognized approaches, or should we be offering exhibitions and programs that challenge assumptions and engage visitors in conversations on various topics of controversy?
							    \item [Engagement:]If I understand the question correctly, panelists will be conversing amongst themselves and with the audience; no PowerPoint or anything other than people willing to engage in a difficult conversation.
							    \item [Relationship to Theme:]This session looks forward. It helps museum workers to understand how consideration of new insights and approaches for facing the immediate future will help them to stay relevant for their audiences.
							    
                \end{description}
              \subsection*{Audience}
                \begin{description}
                  \item [Audiences:]General Audience~
                  \item[Professional Level:]All levels~
                \item[Scalability:] The subject is universal and should be relevant to all museums, regardless of type and size. Museum workers who do not explicitly work in institutions with antiracist missions or programs that examine hate and discrimination will gain understanding that could be of benefit in conversations with their boards, co-workers and visitors. 

							
              \item[Other Comments:] This session will appeal to a broad range of museum professionals, including museum leadership, educators, curators, event planners, collection managers and museum guest services.
              \end{description}
            \subsection*{Participants}
              \subsubsection*{ Judy Margles }
              Submitter, Moderator, Presenter\newline
              Director\newline
              Oregon Jewish Museum and Center for Holocaust Education, Portland, Oregon
              \newline
              jmargles@ojmche.org\newline
              
              503-226-3600 ext 103\newline

              The panelists for this session are all neighbors in Portland’s Old Town neighborhood. We’ve been working together for the past two years to confront issues of antisemitism and anti-Asian discrimination as well as crises in our neighborhood. I will facilitate the conversation.\newline


              
                \subsubsection*{ Judy Margles }
                Moderator, Presenter\newline
                Director\newline
                OJMCHE, Portland Oregon
                \newline
                jmargles@ojmche.org\newline
                
                503-913-5560 ext 103\newline

                This cohort have been working together for more than two years to determine strategies to help visitors navigate the issues of discrimination and hate. At the core of OJMCHE's mission is to help our audiences talk about difficult subjects. My familiarity with the subject and with the other panelists should contribute to a lively and relevant discussion.\newline
                \emph{ (confirmed) }
              

              
                \subsubsection*{ Anna  Truxes }
                Presenter\newline
                Interim Executive Director\newline
                Portland Chinatown Museum, Portland OR
                \newline
                annatruxes@gmail.com\newline
                
                

                As mentioned above, Anna is part of a cohort that we formed about two years ago to confront the myriad of crises that arose with the dual pandemics of racial violence and Covid. Anna will speak from the perspective of the Chinese American community.
                \emph{ (confirmed) }
              

              
                \subsubsection*{ Elizabeth  Nye }
                Presenter\newline
                Executive Director\newline
                Lan Su Chinese Garden, Portland OR
                \newline
                elizabeth@lansugarden.org\newline
                
                

                As mentioned above, Elizabeth is part of a cohort that we formed two years ago to confront the myriad of crises that arose with the dual pandemics of racial violence and Covid. Elizabeth has become the de facto leader of our group, especially with the media, as we work together to demand change in the neighborhood. Elizabeth represents a garden and can offer an interesting perspective about the way in which her institution contributes to the conversation about pressing societal issues.
                \emph{ (confirmed) }
              

              
                \subsubsection*{ Mark Takiguchi }
                Presenter\newline
                Interim Director\newline
                Japanese American Museum of Oregon, Portland OR
                \newline
                mark@jamo.org\newline
                
                

                As mentioned above, Mark is part of a cohort that we formed about two years ago to confront the myriad of crises that arose with the dual pandemics of racial violence and Covid. Mark will speak from the perspective of the Japanese American community and the trauma his museum has endured since opening in a new location in May 2021.
                \emph{ (confirmed) }
              

              
        
          \newpage
          \section{ Labeling Legacies – Applying Today’s Language & Identities to Historical Figures  }
            \begin{description}
              \item [ID:]
              WMA2022\_198

              \item [Assigned to:]
                \item [Track:]Other~
              \end{description}

              How do museums and historians shape a person’s legacy, for better or for worse? Join us for small group conversations as we investigate the stories of aviation pioneers Pancho Barnes and Amelia Earhart through an LBGTQ+ perspective. We’ll consider the wider challenges of representing the diverse ethnicities or sexual/gender identities of historical individuals with today’s language, and examine how what museums say (and don’t say) about a person’s life has a profound impact on visitors.

              \subsection*{Session Information}
                \begin{description}
                  \item [Format:] Regular session/panel (roundtable, single speaker, etc.)
							    
							    \item [Uniqueness:]We’ll explore how to effectively represent historical diversity in museums by posing tough questions about applying today’s identities, labels, language & assumptions to historical figures.
							    \item [Objectives:]Get participants thinking about the importance \& challenges of representing historical diversity while being mindful of both modern identity labels \& historical (and possibly outdated) identities.
Encourage participants to recognize and challenge their internal biases, \& build awareness of how their museum \& others go about diversity representation. Participants should be left thinking about their organization’s intentions and methods toward diversity representation \& the impacts it can have on visitors.
							    \item [Engagement:]This session starts with a brief presentation to introduce some historical figures as discussion examples, and then presenters pose some difficult questions for the audience to discuss in small groups. After several minutes, the groups rejoin to share their discussions. We’re planning for a few rounds of group discussions that build on the previous conversations. Presenters will also provide a handout sheet to all participants with more examples to discuss that can be taken home.
							    \item [Relationship to Theme:]Forward-thinking museums are chomping at the bit to bring diversity to their organizations, \& this session explores the benefits \& challenges of putting that diversity in the public eye through museum displays, education programs, publications, social media, and more. In order to look forward, it’s important that we also take the time to critically observe the past, and decide how to present it to visitors both respectfully \& accurately.
							    
                \end{description}
              \subsection*{Audience}
                \begin{description}
                  \item [Audiences:]Board Members~Curators/Scientists/Historians~Directors/Executive/C-Suite~Diversity and inclusion specialists~Educators~Emerging Museum Professionals~Exhibit Designers, Installers, Fabricators~General Audience~HR Personnel~Marketing \& Communications (Including Social Media)~Other~Programs~Registrars, Collections Managers~
                  \item[Professional Level:]All levels~
                \item[Scalability:] These session outcomes can be applied to absolutely any size of organization \& any organization that displays history in any capacity. It can also apply across a multitude of museum departments. This session can even be useful to individuals to consider their own intentions, biases, \& beliefs.

							
              \item[Other Comments:] We believe educators, historians, exhibit design, \& collections staff will all leave with important takeaways. Senior leadership of museums who want to learn more about how their museums can grow in representation within their spaces are also welcome. Anyone interested in history or diversity will come away with new ideas \& concepts to consider for their own museum.
              \end{description}
            \subsection*{Participants}
              \subsubsection*{ Shae Skager }
              Submitter, Moderator, Presenter\newline
              Administrative Coordinator, Education\newline
              The Museum of Flight, Seattle, WA
              \newline
              sskager@nmof.org\newline
              sskager@nmof.org\newline
              206-768-7214\newline

              From high school to today, Shae Skager (they/them) has enthusiastically participated in the museum industry. They have worked with various museums over the years, and in a multitude of departments and positions. They, with co-presenter Sean Mobley, recently presented on this same session topic for their colleagues at The Museum of Flight. Shae identifies as queer, nonbinary \& transmasculine, and have committed themself to learning and educating others on LGBTQ+ identities, language, and other topics.\newline


              

              
                \subsubsection*{ Shae Skager }
                Presenter\newline
                Administrative Coordinator, Education\newline
                The Museum of Flight, Seattle, WA
                \newline
                sskager@museumofflight.org\newline
                shae.skager@gmail.com\newline
                206-768-7214\newline

                From high school to today, Shae Skager (they/them) has enthusiastically participated in the museum industry. They have worked with various museums over the years, and in a multitude of departments and positions. They, with co-presenter Sean Mobley, recently presented on this same session topic for their colleagues at The Museum of Flight. Shae identifies as queer, nonbinary \& transmasculine, and have committed themself to learning and educating others on LGBTQ+ identities, language, and other topics.
                \emph{ (confirmed) }
              

              
                \subsubsection*{ Sean Mobley }
                Presenter\newline
                Social Media and Content Marketing Specialist\newline
                The Museum of Flight, Seattle, WA
                \newline
                smobley@museumofflight.org\newline
                
                206-768-7201\newline

                Since 2019, Sean Mobley (he/him) has been researching and presenting on LGBTQ+ history in museum contexts, including presentations for The Museum of Flight, the GLBT Historical Society, and the Smithsonian National Air and Space Museum. He has previously presented at WMA, co-leading a workshop on techniques for building tours for blind or low-vision visitors.
                \emph{ (confirmed) }
              

              

              
        
          \newpage
          \section{ Collaboration Tools for Exhibition Projects }
            \begin{description}
              \item [ID:]
              WMA2022\_199

              \item [Assigned to:]
                \item [Track:]
              \end{description}

              Collaboration can be sticky. Lots of ideas, feedback, and opinions contribute to making exhibitions great, but can also be challenging to manage. We’ll be sharing nuts and bolts tools for delivering and asking for feedback, decision-making and feasibility, and improving transparent accountability. In this session, we will also ask participants to share areas of collaborative work they find challenging and begin to match that with other participants who might have tools and solutions to share.

              \subsection*{Session Information}
                \begin{description}
                  \item [Format:] Regular session/panel (roundtable, single speaker, etc.)
							    
								  \item [Fee:]n/a
							     
							    \item [Uniqueness:]This session provides proven tools and processes for collaborative development of exhibition projects that can be used in many areas of museum work.
							    \item [Objectives:]1. Learn about hands on tools that can be used to streamline collaboration in museum work and exhibitions.
2. Share problems to be solved, and solutions, with other participants.
3. Create a community of folks doing this work in the western region who might be able to support each other and share resources after the conference.
							    \item [Engagement:]This session will be split up into two sections. The first is a sharing of the collaborative tools in a typical panel format. This is then followed by a more “Knowledge Café” style activity that includes working together in small groups to share questions and ideas.
							    \item [Relationship to Theme:]Moving museums forward means being more efficient with our own work. It means valuing team member contributions in a way that is equitable and shared. The tools we are presenting take a long-standing tradition of collaborative exhibition-making and add a layer of team transparency and shared understanding.
							    
                    \item [Additional Comments: ]We would like to add a third presenter from a different institution. We are waiting to confirm availability from someone additional from NAME, and are also open to suggestions from WMA.

                \end{description}
              \subsection*{Audience}
                \begin{description}
                  \item [Audiences:]Emerging Museum Professionals~Events Planning~Exhibit Designers, Installers, Fabricators~General Audience~
                  \item[Professional Level:]All levels~Emerging Professional~Mid-Career~Senior Level~
                \item[Scalability:] These tools can be used in any sized organization, and any sized project team.

							
              \item[Other Comments:] While the presenters are from museum exhibition background, and representative of NAME (National Association of Museum Exhibition), the content of this session is applicable to any collaborative development process.
              \end{description}
            \subsection*{Participants}
              \subsubsection*{ Emily Saich }
              Submitter, Moderator, Presenter\newline
              Director of Exhibition Projects\newline
              Monterey Bay Aquarium (and Western Regional Representative of NAME), Monterey, CA
              \newline
              esaich@mbayaq.org\newline
              ebsaich@gmail.com\newline
              831-644-5612\newline

              Emily has over 15 years of experience managing museum exhibition projects, including children’s museums, nature centers, science centers, historic sites, performing art centers, and local regional museums. In her current role, she leads a team of project managers and exhibitions staff to develop collaborative processes, and design, prototype, and build major exhibitions at the Monterey Bay Aquarium.\newline


              

              
                \subsubsection*{ Joey Scott }
                Presenter\newline
                Exhibitions Project Manager\newline
                Monterey Bay Aquarium, Monterey, CA
                \newline
                jscott@mbayaq.org\newline
                
                831-648-6882\newline

                Joey helped to develop and refine the tools presented in this session and actively uses them. Most recently when acting as the project manager for a \$15,000,000 11,000 sq ft exhibition on the Deep Sea.
                \emph{ (confirmed) }
              

              

              

              
        
          \newpage
          \section{ Lessons Learned for New Leaders }
            \begin{description}
              \item [ID:]
              WMA2022\_201

              \item [Assigned to:]
                \item [Track:]Leadership/Careerpath~
              \end{description}

              <strong>Join three Executive Directors as they share lessons learned and strategies for success for new leaders. The session will begin with a fish bowl conversation between the presenters and then open up to a Q\&amp;A.</strong>

              \subsection*{Session Information}
                \begin{description}
                  \item [Format:] Regular session/panel (roundtable, single speaker, etc.)
							    
							    \item [Uniqueness:]This session offered a unique opportunity for new leaders to hear from established executive directors and ask questions.
							    \item [Objectives:]Share lessons learned and strategies for success for new leaders
							    \item [Engagement:]Fishbowl followed by Q\&A
							    \item [Relationship to Theme:]People helping people move the field FORWARD
							    
                    \item [Additional Comments: ]Only 3 presenters now. Looking for a 4th from a rural area.

                \end{description}
              \subsection*{Audience}
                \begin{description}
                  \item [Audiences:]Directors/Executive/C-Suite~Emerging Museum Professionals~General Audience~
                  \item[Professional Level:]All professional levels~
                \item[Scalability:] Panel will represent museums of different sizes and locations

							
              \end{description}
            \subsection*{Participants}
              \subsubsection*{ Jason  Jones }
              Submitter\newline
              Executive Director\newline
              40, Tulsa
              \newline
              jason@westmuse.org\newline
              
              7074334701\newline

              


              

              
                \subsubsection*{ Micah Parzen }
                Presenter\newline
                CEO\newline
                Museum of Us, San Diego, CA
                \newline
                mparzen@museumofus.org\newline
                
                

                Micah is great!
                \emph{ (confirmed) }
              

              
                \subsubsection*{ Felicia  Shaw }
                Presenter\newline
                Executive Director\newline
                Women's Museum of California, San Diego, CA
                \newline
                director@womensmuseumca.org\newline
                
                

                Felicia is great!
                \emph{ (confirmed) }
              

              
                \subsubsection*{ Nicole  Meldahl }
                Presenter\newline
                Executive Director\newline
                Western Neighborhoods Project, San Francisco, CA
                \newline
                nicole@outsidelands.org\newline
                
                

                Nicole is great!
                \emph{ (confirmed) }
              

              
        
          \newpage
          \section{ Woosh.Jee.Een Pulling Together - The Xunaa Cultural Heritage Center and Museum }
            \begin{description}
              \item [ID:]
              WMA2022\_202

              \item [Assigned to:]
                \item [Track:]Indigenous~
              \end{description}

              The Hoonah Indian Association (HIA) is creating a new Xunaa Cultural Heritage Center and Museum (XCHCM) for community cultural activities, museum and interpretive experiences and tribal administration offices, in Hoonah Alaska. The development of the Museum is connected to evolving relationships with the University of Pennsylvania Museum’s Tlingit Louis Shotridge Collection, and the National Parks Service (NPS) stewardship of Glacier Bay National Park.
NPS and the University of Pennsylvania both have ongoing dialogues with the HIA on collaboration, research, reclaiming rights and the repatriation of artifacts.  These partnerships are a forward-thinking approach to the creation of a new museum environment for site, building and exhibits to tell the Xunaa Tlingit story. 

              \subsection*{Session Information}
                \begin{description}
                  \item [Format:] Regular session/panel (roundtable, single speaker, etc.)
							    
								  \item [Fee:]N/A
							     
							    \item [Uniqueness:]Seeking common ground through an empowered tribal perspective brings a new dimension to sharing, collaboration and design for the Hoonah Indian Association.
							    \item [Objectives:]1. Understanding positive processes to empower local tribal sovereignty while building mutual bridges for research, collaboration and sharing. 
2. Creating a multi-dimensional cultural center with full museum functions, yet prioritizing community and cultural gathering. The center will also be a major draw to the cruise tourist industry for 4 months of the year and accommodate the influx of 500,000 visitors to the town of Hoonah, pop. 700.
3. Strategies for embodying traditional stories in the choreography of experience at the new museum through the design of architecture, site, and exhibits.
							    \item [Engagement:]Using films (recently edited), powerpoint, and discussion we will engage the audience in open discussion about ‘finding new pathways of collaboration’ at your institution.
							    \item [Relationship to Theme:]The Hoonah Indian Association could have chosen a more confrontational path to access objects and rights with multiple institutions. They chose a positive collaborative path that resulted in attaining the rights and control of cultural objects and land, while also building strong relationships and sharing potential to move forward with the University of Pennsylvania and the National Park Service.
							    
                \end{description}
              \subsection*{Audience}
                \begin{description}
                  \item [Audiences:]Curators/Scientists/Historians~Diversity and inclusion specialists~Educators~Exhibit Designers, Installers, Fabricators~General Audience~Other~Programs~Registrars, Collections Managers~
                  \item[Professional Level:]All professional levels~
                \item[Scalability:] Collaboration, empowerment and story-telling are common to multiple scales of institutions from regional to large urban museums. The integration of a ‘destination’ site for cruise tourism mixed with an authentic year-round community building is a current issue at multiple sites internationally. 

							
              \item[Other Comments:] Focus on high level organizational collaboration and reconciliation mixed with design strategies for achieving these goals. Diverse audience from Indigenous community members to organizations and museum specialists seeking to partner and find allies in cultural work.
              \end{description}
            \subsection*{Participants}
              \subsubsection*{ Richard (Rich) Franko }
              Submitter, Presenter\newline
              Partner\newline
              Mithun, Inc., Seattle, WA
              \newline
              RichardF@Mithun.com\newline
              
              206-971-5627\newline

              See presenter #3 justification below.\newline


              
                \subsubsection*{ Robert (Bob)  Starbard }
                Moderator, Presenter\newline
                CEO and Tribal Administrator\newline
                Hoonah Indian Association, Hoonah, AK
                \newline
                robert.starbard@hiatribe.org\newline
                
                907-723-7791\newline

                As the leader of the Hoonah Indian Association (HIA) and multiple economic development initiatives, Bob leads the engagement with the University of Pennsylvania Museum, the National Park Service, and the new Xunaa museum design team at Mithun. Bob is leading the XCHC\&M project for HIA, a federally recognized tribe, with over 1200 registered members. 
 He has been directly involved with negotiations with the National Park Service over long standing issues ranging from resource gathering rights to the creation of a new cultural house in Glacier National Park. Bob is a member of the Takdeintaan Tsal Xaan Hit-Mt.Fairweather House.\newline
                \emph{ (confirmed) }
              

              
                \subsubsection*{ Lucy  Fowler Williams  }
                Presenter\newline
                Associate Curator in Charge and Senior Keeper of American Collections of the University of Pennsylvania Museum of Archaeology and Anthropology\newline
                Penn Museum, Philadelphia, PA
                \newline
                wfowler@upenn.edu\newline
                
                215-898-4048\newline

                As the Curator of the Shotridge Collection at Penn Museum, Lucy Fowler Williams has been in discussions with HIA since 2002 regarding Tlingit artifacts acquired by Louis Shotridge during the early 20th Century. A joint partnership between Penn Museum and the tribe resulted in repatriation protocols for Huna artifacts within the Shotridge collection, joint educational and research endeavors, and a collaborative spirit based on balance and respect. Louis Shotridge (Stoowukaa, 1879-1937) was a Tlingit scholar and curator who worked for Penn Museum from 1912-1932 and made collections for Penn. Some of these artifacts took on special significance when the town of Hoonah burned to the ground in June of 1944 and every house and artifacts were destroyed. Member of the XCHC\&M advisory committee, and is an adopted member of the Takdeintaan clan.
                \emph{ (confirmed) }
              

              
                \subsubsection*{ Mary Beth Moss }
                Presenter\newline
                Cultural Anthropologist\newline
                National Park Service , Juneau, AK
                \newline
                Mary_Beth_Moss@nps.gov\newline
                
                

                Mary Beth Moss brings the perspective of a career as a National Park Service Historian and Cultural Anthropologist, balanced with an interim career as a Tribal Administrator for the Hoonah Indian Association. She was a leader in the 2006 process of programming to engage tribal members in the collection of stories and a vision for the first design iteration of the tribal center and museum. She also participated in the discussions around reconciliation and collaboration at Glacier Bay National Park.    With a background in Wildlife Ecology and Forest Science she also has a passion for Traditional Ecological knowledge and lifeways.  Member of the XCHC\&M advisory committee the Chookaneidi clan.
                \emph{ (confirmed) }
              

              
                \subsubsection*{ Richard Franko }
                Presenter\newline
                Partner and FAIA Member\newline
                Mithun, Inc., Seattle, WA
                \newline
                richardf@mithun.com\newline
                
                206-971-5627\newline

                Rich Franko is leading the design of the new XCHC\&M project, and brings a background in the design of museums from the National Nordic Museum to the Wanapum Heritage Center. The Mithun team, includes AldrichPears for exhibit design, and are working to bring to life the Xunaa Tlingit story on a site in the town of Hoonah. The Mithun perspective is based on a thorough analysis of site, story and future goals to create a center that can provide a tourist destination for the cruise industry while emphasizing and prioritizing the year round tribal community functions.
                \emph{ (confirmed) }
              

              
        
          \newpage
          \section{ Wink:  Life of an Object }
            \begin{description}
              \item [ID:]
              WMA2022\_203

              \item [Assigned to:]
                \item [Track:]Indigenous~
              \end{description}

              <strong>Our Indigenous cultural materials have a life and a story to tell.  This is what our ancestors have taught us across time.  As part of the vision for the First Americans Museum, this exhibition realized a special space that could bring our cultural materials back in relationship with our tribal communities.  Working with the Smithsonian’s National Museum of the American Indian, we welcome objects back home to our communities to hear their voices.</strong>

              \subsection*{Session Information}
                \begin{description}
                  \item [Format:] Regular session/panel (roundtable, single speaker, etc.)
							    
							    \item [Uniqueness:]This reunion meets a significant goal to create a museum to educate the broader public about our relationship to our cultures and lifeways.
							    \item [Objectives:]1.      To Learn about the Diverse Lifeways carried across generations since the beginning through our languages and cultural materials.  We remain cultural people because we have not lost our relationships to these conduits of knowledge.
2.     To Learn about Collections Practices in relation to Indigenous People in American history particularly during the time of Removal.
3.     To Learn about the Continuum of Cultures – the ways in which our special cultural materials still relate to us today and continue to walk with us on our journey.
							    \item [Engagement:]Intended Audience is any museum/museum professional that seeks to engage the broader public in learning about Indigenous Cultures through their Objects/Collections or Public Programs; Indigenous Cultural Centers \& Museums.  Participation Strategy – sharing our experience in working with the tribes and tribal communities and engaging in dialogue with the audience.
							    \item [Relationship to Theme:]FORWARD:  Proposed Session explores diverse ways of viewing collections/objects in relation to the communities that create them - forward-thinking exploration to view the creation/collection/continuum of Objects and the potential to reunite them to Indigenous cultures.
							    
                \end{description}
              \subsection*{Audience}
                \begin{description}
                  \item [Audiences:]General Audience~
                  \item[Professional Level:]All levels~
                \item[Scalability:] The session outcomes apply to diverse organization types and sizes because the session presents a unique approach to the relationship between objects and the communities.

							
              \item[Other Comments:] Intended Audience is any museum/museum professional that seeks to engage the broader public in learning about Indigenous Cultures through their Objects/Collections or Public Programs; Indigenous Cultural Centers \& Museums.
              \end{description}
            \subsection*{Participants}
              \subsubsection*{ Gena Timberman }
              Submitter, Moderator, Presenter\newline
              Principal, Luksi Group\newline
              First Americans Museum, Oklahoma City, OK
              \newline
              gena@luksigroup.com\newline
              
              4054202375\newline

              Project experience in developing FAM and working with Indigenous Communities on the exhibition\newline


              

              
                \subsubsection*{ James Pepper Henry }
                Presenter\newline
                Director/CEO \newline
                First Americans Museum, Oklahoma City, OK
                \newline
                jph@famok.org\newline
                
                4055942105\newline

                He is the Director of the First Americans Museum
                \emph{ (confirmed) }
              

              

              

              
        
          \newpage
          \section{ Class-in-Residence: Modeling Participatory Practices in a University Museum }
            \begin{description}
              \item [ID:]
              WMA2022\_204

              \item [Assigned to:]
                \item [Track:]
              \end{description}

              This session focuses on a student-driven exhibition at the Center for Creative Photography (CCP), University of Arizona. In Fall 2022, a graduate-level art education course will inhabit the CCP\&rsquo;s new interdisciplinary gallery to design and implement an exhibition from the CCP’s collection of color photography and focused on community-centered practices. The session discusses the class-residency collaboration from the perspective of the student, faculty, and museum professional as it is in progress.

              \subsection*{Session Information}
                \begin{description}
                  \item [Format:] Regular session/panel (roundtable, single speaker, etc.)
							    
								  \item [Fee:]N/A
							     
							    \item [Uniqueness:]​​This session offers a new type of collaboration between students and academic museums that foregrounds experiential learning opportunities and multivocal exhibition narratives.
							    \item [Objectives:]By the end of the session, participants will:
- Understand the opportunities and strategies for advocating for such collaborations to leadership and stakeholders in their academic community and museums/cultural institutions.
- Consider how best to structure similar, sustainable projects in their institutional contexts in ways that benefit all participants.
- Identify benefits and potential challenges in working with students, faculty, and others on university and museum timelines.
							    \item [Engagement:]While this presentation will be positioned midway through the project’s implementation and progress, we plan to talk with audience members about their institutions and how this collaboration could work in different settings and across disciplines and multigenerational courses. We will employ several brief activities using dialogue techniques (such as pair and share, and small group discussion) to encourage participants to think creatively about applying potential collaborations and new engagement models within their institutions.
							    \item [Relationship to Theme:]This session focuses on engaging university students through an experiential learning opportunity that moves the field forward with the next generation of museum professionals. We will share a project in-process that incorporates new ways of engaging students with fine art and archival collection as they develop a transdisciplinary exhibition over the course of a semester.
							    
                    \item [Additional Comments: ]We have all required presenters already involved in this session and we plan this to be a separate session of its own.  

                \end{description}
              \subsection*{Audience}
                \begin{description}
                  \item [Audiences:]Curators/Scientists/Historians~Educators~Emerging Museum Professionals~Other~
                  \item[Professional Level:]All professional levels~
                \item[Scalability:] Although our collaboration was through an academic museum, any community or cultural institution interested in working with students/multigenerational learners would benefit from knowledge generated through this project. It could also extend to different types of museums which have flexible exhibition spaces for community or student programming.

							
              \item[Other Comments:] N/A
              \end{description}
            \subsection*{Participants}
              \subsubsection*{ Nupur Sachdeva }
              Submitter, Moderator, Presenter\newline
              PhD Student / Teaching Associate\newline
              University of Organization, Tucson, AZ
              \newline
              nupur@email.arizona.edu\newline
              nups15@gmail.com\newline
              404.903.8620\newline

              Nupur Sachdeva is a PhD student in the Art and Visual Culture Education department at University of Arizona. She is enrolled in the class-in-residence that is discussed in this session. She has previous experience as a museum educator, and her research interest focuses on Edu-curation and participatory art-based practices.\newline


              

              
                \subsubsection*{ Carissa DiCindio }
                Presenter\newline
                Assistant Professor, Art and Visual Culture Education\newline
                University of Arizona, Tucson, AZ
                \newline
                cdicindio@arizona.edu\newline
                
                706.338.0626\newline

                Dr. DiCindio is faculty in Art and Visual Culture Education. She is teaching a graduate-level course on edu-curation and participatory practices that is serving as a class-in-residence at the Center for Creative Photography. Her background is in art museum education.
                \emph{ (confirmed) }
              

              
                \subsubsection*{ Meg Jackson Fox }
                Presenter\newline
                Curator of Interdisciplinary \& Community Practices\newline
                Center for Creative Photography, University of Arizona, Tucson, AZ
                \newline
                JacksonFoxM@ccp.arizona.edu\newline
                
                520.621.0447\newline

                Dr. Jackson Fox is the collaborating partner from the Center for Creative Photography (CCP) for the class residency, helping to build the syllabus, coordinate the museum staff’s participation, and joining the course weekly. Her role at CCP is to lead learning and community initiatives, as well as to curate interdisciplinary exhibitions and programming.
                \emph{ (confirmed) }
              

              

              
        
          \newpage
          \section{ Nevertheless, WE Persisted: Using Creative Content to Foster Connection }
            \begin{description}
              \item [ID:]
              WMA2022\_205

              \item [Assigned to:]
                \item [Track:]
              \end{description}

              The Oregon Historical Society’s (OHS) original exhibition, <em><a href="https://www.ohs.org/museum/exhibits/nevertheless-they-persisted.cfm">Nevertheless, They Persisted: Women’s Voting Rights and the 19th Amendment</a></em>, was slated to open the day OHS closed its doors due to COVID-19. In this panel discussion, staff and community partners will share how flexible educational curriculum, valuable community partnerships, and a shift in marketing priorities from visitor conversion to mission delivery made <em>Nevertheless, They Persisted</em> arguably one of OHS’s most successful exhibits in terms of broad access. 

              \subsection*{Session Information}
                \begin{description}
                  \item [Format:] Regular session/panel (roundtable, single speaker, etc.)
							    
							    \item [Uniqueness:]This session presents a successful exhibition community-engagement strategy amid COVID-19 from the perspectives of public historians, educators, and marketing professionals.
							    \item [Objectives:]1) Gain tips for how to build a robust and nimble digital content strategy that is fueled by an organization’s knowledge-holders (museum, archival, education staff, etc.) to connect with members and stakeholders located anywhere in the world.
2) Learn how to design standards-aligned exhibition-inspired curriculum that can be used with or without a museum visit that is appropriate for distance learning as well as how to utilize virtual professional-development workshops to show educators how to implement this curriculum in their classrooms.
3) Discover ways to design a multi-faceted community engagement strategy that takes the learning objectives of an exhibition outside the museum walls and involves diverse community partners.
							    \item [Engagement:]While the panelists will each provide context in support of the stated learning outcomes, we intend for this to be a dynamic presentation with a focus on audience engagement and Q\&A. The panelists will be prepared to workshop ideas with the session attendees and answer questions on the topics of digital content strategy (online exhibits, blogs, social media), curriculum design and educator professional development, and how to develop and maintain strong community partnerships.
							    \item [Relationship to Theme:]FORWARD-thinking institutions need to find ways to foster conversations, inspire connection, and empower everyone to share ideas and learn. The global pandemic, America’s ongoing reckoning with racism, and unprecedented political upheaval have increased the urgency of these goals, even when museum doors are shuttered. By creatively employing digital strategies and community partnerships for mission delivery, institutions of any size can achieve these objectives by broadening and deepening points of connection, while eliminating barriers to access.
							    
                \end{description}
              \subsection*{Audience}
                \begin{description}
                  \item [Audiences:]Educators~Emerging Museum Professionals~Marketing \& Communications (Including Social Media)~Programs~
                  \item[Professional Level:]All levels~
                \item[Scalability:] At the core of this presentation are lessons about meeting visitors where they are — in this case, it was at home and socially distanced — and making use of often hidden talents already within an organization to leverage content appropriately across multiple channels and increase reach and impact. These lessons can be applied to all organizations and in a variety of areas, from marketing to education to community engagement.

							
              \end{description}
            \subsection*{Participants}
              \subsubsection*{ Rachel Randles }
              Submitter\newline
              Director of Marketing \& Communications\newline
              Oregon Historical Society, Portland, Oregon
              \newline
              rachel.randles@ohs.org\newline
              
              503-306-5221\newline

              


              
                \subsubsection*{ Eliza Canty-Jones }
                Moderator, Presenter\newline
                Chief Programs Officer\newline
                Oregon Historical Society, Portland, Oregon
                \newline
                eliza.canty-jones@ohs.org\newline
                rachel.randles@ohs.org\newline
                503-306-5236\newline

                Eliza Canty-Jones oversees the Oregon Historical Society’s education and public programs and is the editor of the society’s scholarly journal, the Oregon Historical Quarterly. She produces scholarship, public programs, and organizational partnerships that advance complex perspectives on Oregon’s past. Canty-Jones was also involved in the content development process for OHS’s Nevertheless, They Persisted exhibition, and fostered partnerships with Sarah Anderson and Jan Dilg in support of OHS’s exhibit-related education and community engagement efforts.\newline
                \emph{ (confirmed) }
              

              
                \subsubsection*{ Erin Brasell }
                Presenter\newline
                Manager Editor, Oregon Historical Quarterly\newline
                Oregon Historical Society, Portland, Oregon
                \newline
                erin.brasell@ohs.org\newline
                
                503-306-5238\newline

                Erin Brasell’s editorial and publication management experience lies at the heart of the success of OHS’s digital content strategy. Brasell spearheaded the internal effort to form the OHS blog, Dear Oregon, and the editorial standards, content strategy, and production processes that enabled OHS to radically accelerate the scale of its digital content production when its doors were closed in March of 2020. Producing an avalanche of compelling, innovative content, written by a broad spectrum of OHS staff, enabled OHS to grow audiences, reach underserved groups, increase fundraising, and apply a mission-driven perspective to current events.
                \emph{ (confirmed) }
              

              
                \subsubsection*{ Sarah Anderson }
                Presenter\newline
                Educator and Author\newline
                Cottonwood School, Portland, Oregon
                \newline
                bringingschooltolife@gmail.com\newline
                
                

                Sarah Anderson is an educator and author specializing in place-based education and curriculum design. Previously a middle school humanities teacher, Anderson is the Fieldwork and Place-Based Education Coordinator at the Cottonwood School. Anderson wrote the Oregon Historical Society’s Nevertheless, They Persisted curriculum and also led a series of virtual professional development workshops for educators in support of OHS’s educational outreach efforts.
                \emph{ (confirmed) }
              

              
                \subsubsection*{ Jan Dilg }
                Presenter\newline
                Historian\newline
                HistoryBuilt, Oregon Women’s History Consortium, Portland, Oregon
                \newline
                jan@historybuilt.com\newline
                
                

                Jan Dilg is a founding member of the Oregon Women’s History Consortium (OWHC), which provides leadership to support research and education related to the history of women in Oregon. In summer 2020, OWHC and OHS created ChalkTheVoteOR, a community engagement campaign aimed to honor, complicate, and be in conversation with the many changes to voting rights in our state’s and nation’s history. This campaign involved many local institutions in public chalking activities as well as produced a series of blog posts that provided activities so that families and community members could participate in the campaign from anywhere.
                \emph{ (confirmed) }
              

              
        
          \newpage
          \section{ Facilitating Critical Conversations around Exhibitions  }
            \begin{description}
              \item [ID:]
              WMA2022\_206

              \item [Assigned to:]
                \item [Track:]
              \end{description}

              **Museums provide space for people to engage in critical conversations. In this session, participants will hear from four museums on their relationship between the curation/exhibitions and education/community programs departments, how educators navigate complex and sometimes controversial topics with visitors, and how program organizers create public discussions on critical topics. <strong>Participants</strong> will also have the <strong>opportunity</strong> to speak with other museum professionals **on how they address critical topics and **foster dialogue and civil <strong>discourse.</strong>   **

              \subsection*{Session Information}
                \begin{description}
                  \item [Format:] Regular session/panel (roundtable, single speaker, etc.)
							    
								  \item [Fee:]none
							     
							    \item [Uniqueness:]We want to connect collections/curatorial and education/community programs departments to enhance collaboration that will improve the visitor experience.
							    \item [Objectives:]Reflect and discuss with peers on the role of museum to engage with and select exhibitions that address critical topics that challenge visitors

Considerations when selecting or curating exhibition content and how education teams frame the conversations and programs

Learn how to foster  deeper understanding of context and the multitude of historical truths that are relevant to contemporary debates.
							    \item [Engagement:]Audience members will discuss prompts with each other at tables. For the main activity, we’ll jigsaw the audience. Participants will begin in their “expert/museum role” groups to discuss a prompt. Afterwards they will be placed in mixed groups where one person from each “expert/museum role” group will be represented. 
No resources (aside from a projector) will be needed.
							    \item [Relationship to Theme:]Our session relates to the theme as it stretches museum professionals to consider and prepare their institutions role or ability to select exhibitions that challenge their visitors in an increasingly polarized society.
							    
                    \item [Additional Comments: ]none

                \end{description}
              \subsection*{Audience}
                \begin{description}
                  \item [Audiences:]Curators/Scientists/Historians~Educators~Exhibit Designers, Installers, Fabricators~Programs~Visitor Services~Volunteer Managers~
                  \item[Professional Level:]All levels~
                \item[Scalability:] Various size and types of organizations are represented in those who are presenting. We range from small (Coos History Museum) to large (Oregon Historical Society) and represent urban and rural communities. The discussion and participation will encourage museums to listen and learn from each other in order to meet the needs of a variety of audiences.

							
              \item[Other Comments:] We really would like to encourage curators and community program officers to attend and hear from educators on how we interpret and engage visitors and how we need to/can work more collaboratively.
              \end{description}
            \subsection*{Participants}
              \subsubsection*{ Amanda  Coven }
              Submitter, Moderator, Presenter\newline
              Director of Education \newline
              Oregon Jewish Museum and Center for Holocaust Education , Portland, OR
              \newline
              acoven@ojmche.org\newline
              
              503-226-3600\newline

              I represent one of the museums who will be sharing their experiences with navigation critical conversations.\newline


              

              
                \subsubsection*{ Amanda  Coven }
                Presenter\newline
                Director of Education\newline
                Oregon Jewish Museum and Center for Holocaust Education, Portland, OR
                \newline
                acoven@ojmche.org\newline
                
                503-226-3600\newline

                Amanda has facilitated educator professional developments on navigating critical conversations about history previously. In her role as Director of Education, she is well versed in lead discussions with students and adult visitors and respond to their questions and comments on sensitive, difficult, and sometimes controversial topics.
                \emph{ (confirmed) }
              

              
                \subsubsection*{ Molly  Wilmoth }
                Presenter\newline
                Bonnie Lee and Oliver P. Steele III Curator of Education \& Engagement\newline
                High Desert Museum , Bend, OR
                \newline
                mwilmoth@highdesertmuseum.org\newline
                
                541.382.4754 x255\newline

                In her currently role at High Desert Museum, Molly oversees the Education \& Engagement team creating programming and reaching out to the community to encourage dialogue and excitement about the High Desert. She has worked in multiple museums of varying sizes and content areas gaining experience working directly with community, co-creating curriculum, and assisting in creating exhibition outcomes focused on challenging traditional interpretations of history.
                \emph{ (confirmed) }
              

              
                \subsubsection*{ Eliza Canty-Jones }
                Presenter\newline
                Chief Program Officer and Editor, Oregon Historical Quarterly\newline
                Oregon Historical Society, Portland, OR
                \newline
                Eliza.canty-jones@ohs.org\newline
                
                503.306.5236\newline

                Eliza produces scholarship and public discussions that bring together academic and community knowledge to advance complex, multi-perspectives understandings of the past. At OHS, she oversees public programming, K-12 education, and publication of a scholarly, public history journal.
                \emph{ (confirmed) }
              

              
                \subsubsection*{ Ariel  Peasley }
                Presenter\newline
                Education and Community Engagement Coordinator\newline
                Coos History Museum, Coos Bay, OR
                \newline
                Education@cooshistory.org\newline
                
                503-951-3981\newline

                Ariel is an emerging museum professional who manages educational programs and community engagement at a small rural history museum on the Southern Oregon Coast. She has worked in exhibits as well as education and is passionate about bringing together collections/exhibits professionals, education professionals, and the community to work together on projects. In her current position this has included the implementation of exhibits, programs, and events that cover diverse and critical topics while involving various individuals, community groups, partner organizations, and more.
                \emph{ (confirmed) }
              
        
          \newpage
          \section{ Managing Successful Museum Projects: Exhibits, Expansions, Renovations and New Construction }
            \begin{description}
              \item [ID:]
              WMA2022\_207

              \item [Assigned to:]
                \item [Track:]
              \end{description}

              Excited about an addition, renovation or new building? Regardless of your museum\&rsquo;s topic, size or age, significant capital projects are heavy lifts! With effective strategies discussed in this session, you\&rsquo;ll use your planning, design and construction dollars wisely and well. Gather the tools to translate your great ideas into effective, innovative, sustainable and responsive built space. Engage great internal and external teams. Get a head start on your project!
** **

              \subsection*{Session Information}
                \begin{description}
                  \item [Format:] Regular session/panel (roundtable, single speaker, etc.)
							    
							    \item [Uniqueness:]The design and construction environment – for buildings, sites and exhibits – changes quickly, and the members of this panel have been on all sides of the space. We have been clients (museum staff), consultants, fabricators and owner’s representatives, so we are advocates for museum professionals being informed, prepared clients for design and construction services.
							    \item [Objectives:]Learning outcomes Include:
·      Develop and answer the basic Implementation questions about your project - budget, schedule, team and goals.
·      Discover techniques for selecting designers, fabricators and contractors, and for procuring services in a fair and effective way.
·      Understand the phases of a design and construction project, from Concept Design to Closeout.
 
Topics will include:
·      Implementation plans - budgets, schedule, programs and project calendars 
·      The relationships between exhibits, building and site in design and construction 
·      Phases of design
·      Care and feeding of consultant teams
·      Internal team roles and responsibilities 
·      Running a project while your museum is open. How does that work? 
·      Project cash flow
							    \item [Engagement:]We will bring partially populated flow charts showing how capital projects proceed, and encourage people to fill in details of their own project, either known or estimated. The format will be open to discussion - the facilitators know each other and can balance each other's expertise and experience. 
    In a round of "lightening presentations," the speakers will cover the basics of getting projects off the ground and seeing them built. Together we will travel through the entire project cycle, and create a picture that attendees will be able to fill in as their project proceeds. We will follow that with directed Q/A, depending on the number of attendees. With a large group, we'll divide the room into four topic areas, with each facilitator leading a specific topic.
							    \item [Relationship to Theme:]Capital projects are, by definition, forward thinking! Many museums have re-assessed their spaces for the post-2020 reality, and many may wonder if physical updates are in order.
							    
                    \item [Additional Comments: ]  The proposed panel has many years, roles and projects under their belts, and a wide range of complimentary and overlapping skills. While several of us have roles as consultants, we are not selling services. We understand what it means for museums to embark on significant projects, and how far outside the comfort zone that can be for museum professionals. Our goal is to help our museum allies be great clients. Happy to talk more about this! 
  

                \end{description}
              \subsection*{Audience}
                \begin{description}
                  \item [Audiences:]Board Members~Directors/Executive/C-Suite~Facilities Management Personnel~
                  \item[Professional Level:]Museum board members~Senior Level~
                \item[Scalability:]   Museums of all sizes (sf or visitor numbers) plan renovations, expansions and updates. The ideas in this presentation apply to anyone who wants to do a capital project with an outside design and construction team.
  

							
              \item[Other Comments:] Intended for museum leadership or project managers. 
Anyone contemplating a capital project can benefit from an introduction, or a refresher, about the design and construction process. Speakers will also address continuous operations plans - or how you do a big project while your museum is up and running.
              \end{description}
            \subsection*{Participants}
              \subsubsection*{ Alissa Rupp }
              Submitter, Moderator, Presenter\newline
              Principal      and        Acting Director\newline
              FRAME | Integrative Design Strategies    and    Seattle Children's Museum, Seattle
              \newline
              alissa@FrameDesignStrategies.com\newline
              alissa@thechildrensmuseum.org\newline
              2062347217\newline

              Alissa proposed that this group gather for the session, and will organize the team, collect slides and images, and orchestrate rehearsals of the group. She is an experienced presenter, and has a long career as a planner, architect and exhibit designer. She is also currently serving as Acting Director of Seattle Children’s Museum, so she sees this topic from many angles! Visit the FRAME website at www.FrameDesignStrategies.com\newline


              

              
                \subsubsection*{ Jill Randerson }
                Presenter\newline
                Principal\newline
                Jill Randerson Project Management, Seattle WA
                \newline
                Jill@jillranderson.com\newline
                
                612-327-7353\newline

                Jill has worked with dozens of museums and science centers, as planner, owner’s representative, cost estimator, project manager and fabricator. She has hands-on experience with getting exhibits designed, built and installed at all scales, across the US and around the world. Her work is detailed at www.jremco.com
                \emph{ (confirmed) }
              

              
                \subsubsection*{ Peter Olson }
                Presenter\newline
                Principal                              and                      Interim Director\newline
                Peter Olson Museum Planning LLC                     Wonder Trek Children’s Museum, Minneapolis, MN
                \newline
                polsonmuseum@gmail.com\newline
                
                507-995-2242\newline

                Peter is an advocate and excellent consultant to small and emerging museums, and believes in the amazing impact a museum can have on a community. He is a planner and has been a full time and interim museum director, so he brings a strong understanding of project management from several perspectives.  Peter’s credentials are detailed at www.startingachildrensmuseum.com
                \emph{ (confirmed) }
              

              

              
        
          \newpage
          \section{ Focusing Resources to Strengthen Resilience: Strategies for Museums }
            \begin{description}
              \item [ID:]
              WMA2022\_209

              \item [Assigned to:]
                \item [Track:]Leadership/Careerpath~
              \end{description}

              Now more than ever, museums must build the capacities needed to continually evolve, operate efficiently, and serve their communities. Presenters from the Institute of Museum and Library Services (IMLS) and TCC Group will discuss findings from their March 2021 Market Analysis and Opportunity Assessment of Museum Capacity Building Programs report. Session participants will explore how museum leaders can overcome the barriers that impede organizational strengthening and invest their limited resources to support real organizational change.

              \subsection*{Session Information}
                \begin{description}
                  \item [Format:] Regular session/panel (roundtable, single speaker, etc.)
							    
								  \item [Fee:]N/A
							     
							    \item [Uniqueness:]Organizational strengthening within the museum sector is critical due, in part, to the impact of the COVID pandemic and nation-wide racial reckoning, and this timely Market Analysis/ Opportunity Assessment can serve as a guidepost for museum leaders and session attendees.
							    \item [Objectives:]Attendees will:
- Increase their knowledge and understanding of the current state of capacity building in the museum sector. 
- Identify and explore recommendations for how they can focus and strengthen capacity building efforts at their own institutions.
- Learn about opportunities offered by IMLS to support museums’ capacity building efforts.
							    \item [Engagement:]All attendees will be provided with copies of the March 2021 report, which is also available online, and will be encouraged to share their insights, reflections, and questions. The goal is to spark ideas and take-aways for how they may most effectively invest in today’s rapidly changing museum sector through capacity building offerings at their institutions.
							    \item [Relationship to Theme:]This session directly supports “FORWARD” in presenting the field’s most recent data analysis regarding capacity building strategies for museums – i.e., to build their potential to be effective, meaningful organizations and to sustain themselves and their communities in today’s ever-changing society.
							    
                    \item [Additional Comments: ]IMLS is open to reconfiguring this session based on WMA’s needs. Speakers/presenters are not yet confirmed. Travel and attendance is dependent upon latest Federal guidance regarding COVID-19 pandemic, Federal budget, and IMLS leadership discretion.

                \end{description}
              \subsection*{Audience}
                \begin{description}
                  \item [Audiences:]Board Members~Development and Membership Officers~Directors/Executive/C-Suite~Diversity and inclusion specialists~General Audience~Other~
                  \item[Professional Level:]General Audience~Mid-Career~Museum board members~Senior Level~
                \item[Scalability:] Findings and recommendations from the Market Analysis and Opportunity Assessment of Museum Capacity Building Programs report, while primarily focused on and drawn from small-to-mid-sized museums, are applicable to cultural institutions of all types and sizes, as well as museum associations and funding organizations.

							
              \item[Other Comments:] A general audience is welcome and encouraged to attend, but primary intended audience are museum leaders and decision-makers, particularly those at small and medium-sized museums, museum associations, and funding organizations.
              \end{description}
            \subsection*{Participants}
              \subsubsection*{ Sarah Glass }
              Submitter, Moderator, Presenter\newline
              Senior Program Officer\newline
              Institute of Museum and Library Services, Washington, DC
              \newline
              sglass@imls.gov\newline
              
              202-653-4668\newline

              An IMLS Senior Program Officer (or higher) in the Office of Museum Services will introduce the presenters and topics to be covered, as well as provide background for the Agency’s engagement in this capacity building report. This person will also provide the latest information on upcoming (FY23) funding opportunities relevant to these topics. (Sarah Glass is point of contact).\newline


              

              
                \subsubsection*{ Tim  Hausmann }
                Presenter\newline
                Consultant\newline
                TCC Group (www.tccgrp.com), New York, NY
                \newline
                thausmann@tccgrp.com\newline
                
                212-949-0990\newline

                Tim Hausmann is co-author of the Market Analysis and Opportunity Assessment of Museum Capacity Building Programs report. As an experienced Consultant with TCC Group, he conducts and supports activities that strengthen the governance mechanisms of nonprofit organizations and foster collaboration and inclusive impact programming across and within the sector. Tim has presented on this topic with IMLS staff at previous conferences (AAAM 2021, Kansas Museums Association, 2021) with very positive feedback.
                \emph{ (not confirmed) }
              

              
                \subsubsection*{ Samantha Hackney }
                Presenter\newline
                Senior Consultant\newline
                TCC Group (www.tccgrp.com), Portland, ME
                \newline
                shackney@tccgrp.com\newline
                
                212-949-0990\newline

                Samantha Hackney is co-author of the Market Analysis and Opportunity Assessment of Museum Capacity Building Programs report. As an experienced Senior Consultant with TCC Group, she conducts and supports activities that strengthen the governance mechanisms of nonprofit organizations and foster collaboration and inclusive impact programming across and within the sector. Samantha has presented on this topic with IMLS staff at previous conferences (AAAM 2021, Kansas Museums Association, 2021) with very positive feedback.
                \emph{ (not confirmed) }
              

              

              
        
          \newpage
          \section{ Tribal Objects in Small Museums: Building Partnership }
            \begin{description}
              \item [ID:]
              WMA2022\_210

              \item [Assigned to:]
                \item [Track:]Indigenous~
              \end{description}

              The Utah Divisions of Indian Affairs, Arts \&amp; Museums, and State History with Utah Humanities have begun partnering to assist Utah museums in becoming better stewards of tribal objects in collections. We are creating resources to help identify objects (prehistoric through contemporary) they may have in their collections. We want to hear how museums and designated tribal representatives can be supported to begin a process of community consultation and practice of shared museum authority.

              \subsection*{Session Information}
                \begin{description}
                  \item [Format:] Regular session/panel (roundtable, single speaker, etc.)
							    
							    \item [Uniqueness:]This is our state’s first coordinated attempt to address this issue and to provide guidance and support to small museums and their tribal neighbors.
							    \item [Objectives:]We have spent our efforts over the last year collecting feedback from tribal members, government agency representatives, and small museum professionals. We are in the process of developing and refining tools and processes that provide broad access and utility to help guide the long-term work of appropriate community collaboration and consultation in the small museum context. Our session objectives are to collect region-wide information about what methods or structures work in other areas, as well as to hear new ideas we may not have thought of. This is not a NAGPRA session, nor is it strictly indigenous, as the underpinnings of good community collaboration can be applied in many different situations and in varying types of organizations. Good ideas and illuminating experiences can come from anywhere. Our hope is that others can contribute or borrow strategies for good community collaboration to create better exhibits, more meaningful and truthful interpretation, and more appropriate storage solutions for their collections after hearing what others in the room have learned.
							    \item [Engagement:]The session will centralize participants' experiences - good and bad - with this topic. We want to crowd-source lessons learned and tools/tips of greatest efficacy to inform development of other resources and guidelines. The strength of our project is collaboration, and we are interested in hearing about relationships or partnerships others have developed to the best impact in their communities. Those with little experience will hear an array of ideas that may present a starting place.
							    \item [Relationship to Theme:]It is our thinking that all museums need strategies and resources in place to address their relationships with their communities and neighbors. This is the central role of museums, and the best way for us to put our best foot forward.
							    
                \end{description}
              \subsection*{Audience}
                \begin{description}
                  \item [Audiences:]Curators/Scientists/Historians~Directors/Executive/C-Suite~Diversity and inclusion specialists~Educators~General Audience~Registrars, Collections Managers~
                  \item[Professional Level:]All professional levels~
                \item[Scalability:]  The session is geared toward small museums, who tend to be more reliant on fewer staff and smaller budgets. The ideas shared will necessitate an ability to scale specifically to small museum contexts. 

							
              \end{description}
            \subsection*{Participants}
              \subsubsection*{ Emily Johnson }
              Submitter, Moderator, Presenter\newline
              Field Services Manager\newline
              Utah Arts and Museums, Salt Lake City, UT
              \newline
              emilyjohnson@utah.gov\newline
              
              

              The submitter is a primary participant in the collaborative effort described in Utah. Also, this is a weird question given the available session formats - it's kind of facilitating/moderating/presenting all at once.\newline


              

              
                \subsubsection*{ James Toledo }
                Presenter\newline
                Program Manager\newline
                Utah Division of Indian Affairs, Salt Lake City, UT
                \newline
                jtoledo@utah.gov\newline
                
                

                The presenter is a key collaborator in the partnership described.
                \emph{ (confirmed) }
              

              
                \subsubsection*{  TBD }
                Presenter\newline
                \newline
                , Salt Lake City, UT
                \newline
                emilyjohnson@utah.gov\newline
                
                

                
                \emph{ (not confirmed) }
              

              

              
        
          \newpage
          \section{ High Risk or High Stakes? Embracing risk-taking in museums }
            \begin{description}
              \item [ID:]
              WMA2022\_211

              \item [Assigned to:]
                \item [Track:]
              \end{description}

              Risk-taking is an important part of human development, and can lead to significant learning opportunities in museums.  Risks are also important to consider in leadership, capital campaigns and community engagement. Risks available to visitors may be physical, emotional or intellectual, and perception of that risk depends on context. In this session, museum staff, board members and designers will examine the role of risk in the museum environment. We’ll consider the power – and pitfalls – of integrating active play, controversial topics, scientific discoveries, and issues like gender, social justice and inclusion into exhibit design and programs.

              \subsection*{Session Information}
                \begin{description}
                  \item [Format:] Regular session/panel (roundtable, single speaker, etc.)
							    
							    \item [Uniqueness:]Risk is only rarely discussed in museum exhibit and program design, yet it is inextricable from frequently mentioned concepts of citizenship, learning and critical thought. We'd like to have a lively discussion about what risk means in different contexts, to different people.
							    \item [Objectives:]This session is meant to raise an important topic that can touch on many of the most significant social and cultural issues of our time.  What for some visitors may seem high-stakes or high-risk, may for others be a welcome challenge. What does it mean to encourage museum visitors, especially kids, to take risks in museums? Should we create exhibits and encourage interactions that help visitors develop decision-making and risk-assessment skills?
 
Attendees will have access to these concrete takeaways:
·      We will do a series of exercises showing how to discuss risk with museum staff, and discuss the facilitation of the discussion. Attendees then can take these exercises back and facilitate them with their teams. 
·      Each panel member will bring an example of an exhibit where anticipated (or unanticipated) risks played out in the design, programming or ongoing use. We will provide a handout summarizing those exhibits. 
·      Table discussions will include a wide range of considerations including Defining Risk, as Risk looks very different to different people and Museums
o  Risk and Danger
o  Risk in discussion of politics and social issues 
o  Ricky vs scary – be clear about the difference
o  The book “50 Dangerous Things…” and its origin
o  Risk for adults vs for kids, and family risk taking
o  Risk to donors, or to the donor relationship
							    \item [Engagement:]This session will engage a diverse panel of museum professionals, and will allow a significant amount of interaction among participants. Panelists will ask a series of interactive questions, allowing real-time on-screen voting. 
 
Attendees will be encouraged to think about how play, emotional engagement, parenting, controversial topics, scientific discoveries, and issues of gender, social justice and access might provide the context for experiences that visitors may see as posing a physical, emotional or intellectual challenge. We will discuss what makes an exhibit seem “risky” and what the “stakes” might be for visitors engaging with these exhibits.
							    \item [Relationship to Theme:]The session is about how we move FORWARD with courage, clarity and conviction to bring challenges, new ideas, and engagement to visitors. Museums (and museum visitors and staff) will need to take some risks to remain relevant and compelling in the future, and also to remain competitive with with new trends in immersive and interactive commercial spaces. There are also always-evolving expectations with regard to the "tough topics" with which society is grappling.
							    
                    \item [Additional Comments: ]  We are open to developing this topic in a less conventional format than the traditional panel. We have ideas about how to create a highly interactive session, bringing the idea of risk and comfort to the session itself. This session may be of most interest to those planning exhibits, programs, installations or events for their museums, but it will also be a powerful topic for others.
 Also, if the WMA program committee is interested in this topic, we will invite 2 more presenters to be part of the main fishbowl panel, and then have an "open chair" format so that others who are exploring this issue in their museums can join. 

                \end{description}
              \subsection*{Audience}
                \begin{description}
                  \item [Audiences:]Board Members~Directors/Executive/C-Suite~Diversity and inclusion specialists~Educators~Exhibit Designers, Installers, Fabricators~General Audience~
                  \item[Professional Level:]All levels~
                \item[Scalability:] The ideas presented here will be flexible with regard to museum type or size. Every museum, whether a children’s museum, art museum, science center or historic house can think about the role of risk in their learning environments, programs or events.

							
              \item[Other Comments:] Museum professionals interested in how visitors engage with challenging topics, activities or spaces are encouraged to attend and participate. The panelists consider this an ongoing discussion, not a settled question! After brief presentations that include images and case studies to get the discussion going, session attendees will:
1.     Consider the idea of risk in museum environments, and think about how their organizations or facilities might engage with risk.
2.     Assess decisions related to how other museums grapple with, and ultimately take on, the idea of risk.
3.     Be inspired by examples of exhibits, programs and installations that allow visitors (and staff) to take risks in the museum environment.
              \end{description}
            \subsection*{Participants}
              \subsubsection*{ Alissa Rupp }
              Submitter, Moderator, Presenter\newline
              Principal                                                Acting Director\newline
              FRAME Integrative Design Strategies         Seattle Children’s Museum, Seattle
              \newline
              alissa@FrameDesignStrategies.com\newline
              alissa@thechildrensmuseum.org\newline
              2062347217\newline

              Alissa Rupp, is a leader in the design of places for community building, informal education and lifelong learning. She works with the conviction that we can improve public life through the serious work of creating spaces where art, nature, culture and play intersect. Her values include interdisciplinary thinking, collaboration, co-creation and clarity. On the projects she is most passionate about, the visitor experience is key; her wide definition of “client,” includes an organization’s staff, visitors, stakeholders, volunteers, collections, and even the flora and fauna of the project site.\newline


              

              
                \subsubsection*{ Putter Bert }
                Presenter\newline
                President and CEO\newline
                KidsQuest Children’s Museum, Bellevue, WA
                \newline
                putter@kidsquestmuseum.org\newline
                
                425-637-8100\newline

                Putter Bert joined KidsQuest Children’s Museum (KQCM) in 1999 as Executive Director and lead the plan to create a new children’s museum in Bellevue. KQCM opened in Factoria in 2005. Built to serve 70,000 visitors, KQCM served almost double the projected patrons in its first year. Putter then led a \$12.7M capital campaign to secure a downtown Bellevue location for an expanded facility that opened in January 2017. Ms. Bert has over 30 years of experience serving in museum leadership roles. As KidsQuest President \& CEO she has also been a leader in the Museum’s community including the Association of Children’s Museums and Northwest Association of Youth Museums.
                \emph{ (confirmed) }
              

              

              

              
        
          \newpage
          \section{ Helping Communities Heal in the Wake of Local Crisis }
            \begin{description}
              \item [ID:]
              WMA2022\_212

              \item [Assigned to:]
                \item [Track:]
              \end{description}

              As natural disasters and crises become prevalent, hear how four museums responded to wildfires and the recent pandemic. Learn innovative ways to help your community heal. Each museum will share how they addressed local crises in thoughtful and meaningful ways while staying true to their missions and protecting their collections. Through partnerships, interactive social media platforms, creative artmaking, reflective exhibitions, collecting oral histories, and developing programs, each museum became a place of gathering, engagement, connection, reflection, and support.

              \subsection*{Session Information}
                \begin{description}
                  \item [Format:] Regular session/panel (roundtable, single speaker, etc.)
							    
							    \item [Uniqueness:]Crisis is personal, communal, and global. Learn innovative and practical ways for surviving and supporting visitors in extreme need by becoming a place of healing.
							    \item [Objectives:]Upon completion of this presentation, 1) participants will gain new ideas for ways to prepare for and respond to local crises that are appropriate to their budget, size, and mission including creating meaningful local partnerships, developing responsive public programming and social media platforms, curating reflective exhibitions, and actively collecting and preserving history as it’s happening.
2) Participants will learn best practices for meeting community needs in the wake of devastating personal loss and disaster and will benefit from lessons learned about responses that were not beneficial and how best intentions are not always best practices. 
3) Participants will learn about and receive hand-outs listing local and national organizations that respond to natural disasters and are available for partnerships, recommended natural disaster preparedness plans and kits, and direct contact information for museum staff willing to share their customized natural disaster plans and responses.
							    \item [Engagement:]After sharing how four museums lived through different wildfires at different times and met visitor needs in mission-driven, meaningful, relevant, and deeply personal ways, we will invite questions and discussion from participants, so each participant will be able to create a site-specific blueprint for how to respond if the need arises in their own community. Participants will see images, details of specific programs, and take away hand-outs with information and resources.
							    \item [Relationship to Theme:]The panel will explore moving forward after natural disasters and local crises, with museums harnessing the power of collaboration across varied and diverse organizations to respond to immediate community needs. The presenting museums will share how they promoted access and inclusion for visitors; delivered workable and relevant content through educational programming, exhibitions, and online forums that furthered the goals of the museums and the needs of the community; and how each museum became a center for community engagement when the community needed it the most.
							    
                \end{description}
              \subsection*{Audience}
                \begin{description}
                  \item [Audiences:]Curators/Scientists/Historians~Diversity and inclusion specialists~Educators~Events Planning~General Audience~Programs~Visitor Services~
                  \item[Professional Level:]All levels~
                \item[Scalability:]   The museums presenting range from very small to mid-sized, include art and history collections, and responded to different fires of differing magnitudes and the Covid pandemic in different parts of Northern California. They will demonstrate how any museum of any size can respond to community needs in ways that meet their mission, preserve their collection, and respect staff losses and experiences as well as those of members, volunteers, and the community.
Museums of Lake County has only one full-time employee and serves a population of around 64,000 through three history-focused sites and will share their ongoing response to the 2015 Valley Fire that destroyed almost 2,000 structures and the Mendocino Complex Fire of 2018. The Charles M. Schulz Museum, a mid-sized single-artist museum, and Museum of Sonoma County, a mid-size Smithsonian Affiliate art and history museum that serves Sonoma County (with a population of about 480,000) and led community-wide responses to multiple North Bay Wildfires as well as creating robust virtual program content during the pandemic.
California Indian Museum and Cultural Center has ten staff members and serves a local population of twenty-four tribes throughout Sonoma, Mendocino, and Lake counties. This region includes nearly 8,000 tribal community members, in addition to the local Santa Rosa population of 175,000.

  

							
              \end{description}
            \subsection*{Participants}
              \subsubsection*{ Jessica Ruskin }
              Submitter, Presenter\newline
              Education Director\newline
              Charles M. Schulz Museum, Santa Rosa, California
              \newline
              jessica@schulzmuseum.org\newline
              
              707-284-1265\newline

              As the Education Director and primary audience advocate at the Charles M. Schulz Museum, Jessica Ruskin will highlight the ways the education team worked with local organizations during the 2017-2019 wildfires to develop immediate and relevant programs for families who lost their homes and to raise funds and awareness and how they revisited and honored the fires one year later. Ruskin will also speak to the Museum's pandemic response as they shifted all programming online.\newline


              
                \subsubsection*{ Jeff Nathanson }
                Moderator, Presenter\newline
                Executive Director\newline
                Museum of Sonoma County, Santa Rosa, California
                \newline
                jnathanson@museumsc.org\newline
                
                (707)579-1500 ext 102 \newline

                Jeff Nathanson is an experienced presenter who has moderated numerous panel discussions in his roles as museum director, community arts advocate, educator, and facilitator. He was instrumental in coordinating a community-wide creative response to the 2017 wildfires and has continued to be an active organizer and collaborator, working with multiple museums and other organizations to promote healing and renewal. He also led the effort during the pandemic to quickly shift from in-person to virtual programs and to provide the museum’s audiences with meaningful content while sheltering in place and after welcoming back visitors.\newline
                \emph{ (confirmed) }
              

              
                \subsubsection*{ Jesse  Clark McAbee }
                Presenter\newline
                Curator of Museums\newline
                Museums of Lake County, Lakeport, California
                \newline
                Jclark.mcabee@lakecountyca.gov\newline
                
                707-530-1598\newline

                Lake County has been the location of several of the largest and deadliest wildfires in California in the past 7 years. During this period all of the three County museums have been emergency evacuated including staff and objects under immediate threat of wildland fire. The Museum helped host and conduct oral video interviews for a professionally produced documentary of the devastating Valley Fire.
                \emph{ (confirmed) }
              

              
                \subsubsection*{ Nicole Lim }
                Presenter\newline
                Executive Director\newline
                  The California Indian Museum and Cultural Center, Santa Rosa, California
                \newline
                nikkimyers@aol.com\newline
                
                  (707) 579-3004\newline

                Nicole Lim has been working towards creating a resiliency center at the museum, implementing a Resilient Native Generations project to increase green infrastructure and institutional capacities to facilitate equitable responses and resources to tribal communities during a crisis. Additionally, she has directed strengths based cultural programming to address healing and processing of grief and trauma related to disaster response.
                \emph{ (confirmed) }
              

              

              
        
          \newpage
          \section{ Tools for Meaningful and Engaging Internship Programs }
            \begin{description}
              \item [ID:]
              WMA2022\_213

              \item [Assigned to:]
                \item [Track:]
              \end{description}

              Interns can be (and are!) more than just a morning coffee run. Inviting interns onto your team is a great way to introduce support, generate diverse ideas, and create valuable learning opportunities for all involved. But not all internship programs are created equal. This session explores tools and engagement strategies to provide more meaningful experiences to interns and to the institutions they contribute.

              \subsection*{Session Information}
                \begin{description}
                  \item [Format:] Regular session/panel (roundtable, single speaker, etc.)
							    
							    \item [Uniqueness:]This session is focused on intern-centered engagement and the mutually beneficial working relationships that come from engaging interns in meaningful ways.
							    \item [Objectives:]During this session, participants will explore key aspects of creating a meaningful internship program specifically for their institution, including: 
1. Brainstorming institutional common goals, roles for interns, and potential partnerships with diverse community organizations  
2. Workshopping tools, programs, and/or communities for meaningful intern engagement 
3. Identifying ways to learn from intern input and feedback to grow their institution

These objectives will be met through collectively examining an example process for creating tools and communities to improve an existing internship program, brainstorming meaningful internship goals and outcomes with other session participants for a variety of institution types and disciplines, and discussing other potential tools or sharing out successful strategies to address these ideas and develop valuable internship experiences.

In the spirit of looking Forward, a large emphasis will be placed on engaging diverse students in internship programs. Participants will be encouraged to break away from traditional internship recruitment and work with community organizations to invite a wide range of student backgrounds, experiences, and skills to their organization.
							    \item [Engagement:]Participants will explore the process of developing meaningful tools and engagement strategies for their interns, have the opportunity to examine examples developed for existing internships, discuss potential needs and goals for programs at their own institutions, and workshop their strategic ideas with colleagues.
							    \item [Relationship to Theme:]Creating more meaningful experiences for interns can generate greater support and unique directions for museum staff while providing new opportunities for emerging professionals to learn and grow through internship positions. Building a purposeful, intern-centered program can also open the door to greater diversity of individuals, innovations, change, and technology across museum institutions.
							    
                    \item [Additional Comments: ]This session would be a mix of a Standard panel and Knowledge Cafe format.

If selected, we would only be available to present the session on Friday, April 7 or in the afternoon of Saturday, April 8 due to an exhibit opening at Jennifer’s museum.

                \end{description}
              \subsection*{Audience}
                \begin{description}
                  \item [Audiences:]General Audience~
                  \item[Professional Level:]All professional levels~
                \item[Scalability:] This intern-centered engagement process explored in this session can be applied to new or existing programs of any size in order to create more meaningful and intentional experiences for interns.

							
              \item[Other Comments:] Any participant who oversees interns or is interested in creating an internship program in a meaningful and engaging way.
              \end{description}
            \subsection*{Participants}
              \subsubsection*{ Peter Kukla }
              Submitter, Moderator, Presenter\newline
              Planetarium Manager\newline
              Eugene Science Center, Eugene, OR
              \newline
              peterj.kukla@gmail.com\newline
              pkukla@eugenesciencecenter.org\newline
              503-863-8588\newline

              Peter has nearly 4 years of experience in museums, supervising interns and volunteers in a wide variety of roles over that time. He ultimately entered the field after interning for his co-presenter, Jennifer Powers, over 3 years. Peter has also co-developed engagement tools and his own internship positions, designed to improve experiences for future participants by centering their needs and goals.\newline


              

              
                \subsubsection*{ Jennifer Powers }
                Presenter\newline
                Featured Hall Assistant Manager\newline
                Oregon Museum of Science and Industry, Portland, OR
                \newline
                jpowers@omsi.edu\newline
                jenlynnrichards@gmail.com\newline
                503-797-4687\newline

                Jennifer has more than 7 years of experience managing museum internship, work-study and volunteer programs. She has co-developed, tested, and improved a number of engagement tools and programs to facilitate student-centered experiences for diverse students and organizations.
                \emph{ (confirmed) }
              

              

              

              
        
          \newpage
          \section{ More Than a Gold Star: The Power of State, Regional, and National Awards }
            \begin{description}
              \item [ID:]
              WMA2022\_216

              \item [Assigned to:]
                \item [Track:]Other~
              \end{description}

              It’s no secret that museums of all sizes are doing great work around the country, and quantifying and sharing wins is more important than ever. Awards can help sites increase visitation, improve publicity, secure funding, boost morale, and encourage innovation. Join our discussion of Oregon Heritage, WMA, and AASLH awards programs as we share tips, demystify the nomination process, and explain how awards can be powerful marketing tools and catalysts for growth.

              \subsection*{Session Information}
                \begin{description}
                  \item [Format:] Regular session/panel (roundtable, single speaker, etc.)
							    
							    \item [Uniqueness:]A chance to directly engage with awards administrators to learn best tips for nominations and why they should apply.
							    \item [Objectives:]Attendees will learn about the processes for state, regional, and national awards they are eligible for; feel supported and empowered to participate in an accessible process where they now know the administrators; and understand what awards can do for an organization beyond just being a pat on the back.
							    \item [Engagement:]Attendees will hear brief presentations on the 3 awards programs (and respond to informal polls about their awards knowledge), their similarities and differences in requirements, and how awards can be catalysts for organizational growth. Then we'll have audience Q\&A for specific questions about each program, as well as going through a mock nomination so they can get hands-on experience turning a great exhibit or program into a great application.
							    \item [Relationship to Theme:]Awards can help museums move forward by encouraging innovation, rewarding risk-taking, normalizing professionalization and best practices,
							    
                    \item [Additional Comments: ]Willing to add additional presenters or merge if needed! Anyone involved with their organization's awards program would be welcome.

                \end{description}
              \subsection*{Audience}
                \begin{description}
                  \item [Audiences:]Board Members~Curators/Scientists/Historians~Directors/Executive/C-Suite~Diversity and inclusion specialists~Educators~Emerging Museum Professionals~Exhibit Designers, Installers, Fabricators~Marketing \& Communications (Including Social Media)~Programs~Registrars, Collections Managers~
                  \item[Professional Level:]All levels~
                \item[Scalability:] Awards are for organizations of all kinds and budgets, and a goal of this session is to democratize the application process. Small orgs should not be at a disadvantage because they haven't done a nomination before, or don't personally know the people or organizations doing the judging. We want to make awards more visible, accessible, and achievable for all kinds of organization. Any org will know how to put together a great application with the tips and guidance we will share.

							
              \item[Other Comments:] Each program will have their specific requirements on who can submit or what type of work can be submitted, of course, but between the 3 speaker orgs there is an opportunity for anyone to spotlight their org's great work in an award nomination. So anyone interested in awards in the field would be the intended audience.
              \end{description}
            \subsection*{Participants}
              \subsubsection*{ Aja Bain }
              Submitter, Moderator, Presenter\newline
              Program and Publications Manager\newline
              AASLH, Nashville TN
              \newline
              abain@aaslh.org\newline
              
              6159441588\newline

              I will talk about AASLH's national awards program, as well as facilitate questions and activities. I have done versions of this session before at other state and regional conferences.\newline


              
                \subsubsection*{ Aja Bain }
                Moderator, Presenter\newline
                Program and Publications Manager\newline
                AASLH, Nashville TN
                \newline
                abain@aaslh.org\newline
                
                6159441588\newline

                I will talk about AASLH's national awards program, as well as facilitate questions and activities. I have done versions of this session before at other state and regional conferences.\newline
                \emph{ (confirmed) }
              

              
                \subsubsection*{ Katie Henry }
                Presenter\newline
                Oregon Heritage Commission Coordinator\newline
                Oregon Heritage/State Historic Preservation Office, Eugene OR
                \newline
                katie.henry@oprd.oregon.gov\newline
                
                

                Katie will discuss Oregon Heritage's awards program.
                \emph{ (confirmed) }
              

              
                \subsubsection*{ WMA awards person }
                Presenter\newline
                \newline
                WMA, Tulsa, OK
                \newline
                proposals@westmuse.org\newline
                
                

                We would love for someone involved with WMA awards to join us!
                \emph{ (not confirmed) }
              

              

              
        
          \newpage
          \section{ NatureScape Playground: Fun and Learning in an Earthly Setting }
            \begin{description}
              \item [ID:]
              WMA2022\_217

              \item [Assigned to:]
                \item [Track:]
              \end{description}

              In 2022, the Ute Indian Museum built a natural playground to provide a space for children to learn and play. The NatureScape is a Scandinavian concept with a goal to reconnect children to nature and creative play. The museum has combined that ideal with the Native American belief of working with and respecting Mother Nature. This presentation will share project information so this concept can be replicated at other museums.

              \subsection*{Session Information}
                \begin{description}
                  \item [Format:] Regular session/panel (roundtable, single speaker, etc.)
							    
								  \item [Fee:]$0
							     
							    \item [Uniqueness:]This presentation will demonstrate how museum educational opportunities can expand outside of the institution walls onto the museum grounds.
							    \item [Objectives:]The objective of this presentation is to inspire attendees to strive for outside-the-box educational opportunities and providing them with knowledge they can take back to their organizations to create similar projects that fit organically with their particular organization.  Practical knowledge will be shared to provide attendees with real world examples of things to do and things not to do when undertaking large-scale projects like this. This session will hopefully inspire museums to take a chance on innovating how their museum grounds are used.
							    \item [Engagement:]The Ignite format will be an ideal method for sharing the NatureScape project details with project photographs and architecture plans.
							    \item [Relationship to Theme:]The Ute Indian Museum is moving Forward by expanding the educational opportunities at the museum by combining a play and learning space. Museums needs to innovate to engage communities and make learning more approachable. This presentation will also discuss how the project embraces diversity, equity, inclusivity, \& accessibility.
							    
                \end{description}
              \subsection*{Audience}
                \begin{description}
                  \item [Audiences:]Educators~General Audience~Programs~
                  \item[Professional Level:]All levels~All professional levels~
                \item[Scalability:] Organizations of all sizes can learn from this presentation, as the Ute Indian Museum is a small organization under the umbrella of a larger state organization. Funding is an issue for organizations of all sizes and this presentation will address different funding sources for this somewhat large-scale project.

							
              \end{description}
            \subsection*{Participants}
              \subsubsection*{ Dena Sedar }
              Submitter, Presenter\newline
              Education Director\newline
              Ute Indian Museum, Montrose
              \newline
              dena.sedar@state.co.us\newline
              denar63@yahoo.com\newline
              970-249-3098\newline

              The submitter and presenter are the same person.\newline


              

              
                \subsubsection*{ Dena Sedar }
                Presenter\newline
                Education Director\newline
                Ute Indian Museum, Montrose
                \newline
                dena.sedar@state.co.us\newline
                dena.sedar@state.co.us\newline
                970-249-3098\newline

                The presenter is the Education Director at the Ute Indian Museum and has been with the museum during the construction and unveiling stages of the project.
                \emph{ (confirmed) }
              

              

              

              
        
          \newpage
          \section{ Lighting for Artifacts: Protecting Collections in an LED Age }
            \begin{description}
              \item [ID:]
              WMA2022\_218

              \item [Assigned to:]
                \item [Track:]
              \end{description}

              Should you use LEDs or Fiber Optic Lighting for artifact lighting? Should we spot measure or calculate exposure over time? We will explore a variety of lighting techniques including light sources, occupancy sensors, and lighting control systems. We will also present two case studies detailing lighting conservation issues for the new Academy Museum of Motion Pictures in Los Angeles and George Washington\&rsquo;s Tent at the Museum of the American Revolution in Philadelphia.

              \subsection*{Session Information}
                \begin{description}
                  \item [Format:] Regular session/panel (roundtable, single speaker, etc.)
							    
							    \item [Uniqueness:]A review and demonstration of state-of-the art artifact lighting technology. From high efficacy to superior color rendering. Separate fact from fiction.
							    \item [Objectives:]Learning Objectives:
1)      Gain a greater understanding and appreciation of artifact conservation lighting in museums which includes deeply informative two case studies.
2)      Learn key critical components that go into the development of a lighting design.
3)      Learn about luminaire/light source options and control system considerations (pros and cons of different choices)
 
There are several key “chapters” to this presentation:
1)      Introduction by the Lighting Manager of the Academy Museum of Motion Pictures. A discussion of why conservation lighting need not be boring lighting. Personal reflections on the day-to-day demands of maintaining systems. A brief tour/case study of the new museum.
2)      An gentle-touch introduction to lighting design considerations—aesthetic and technical (i.e., color temperature, color rendering rating system, contrast, revealing form, etc.)
3)      A deeper dive into case lighting including choosing LED vs. Fiber Optic, mounting details, etc.
4)      Transition into a hardware demonstration where attendees are able to see and manipulate equipment in their own hands.
5)      An over overview of lighting design specifically systems addressing conservation issues.
6)      A case study in using the conservation metric of exposure over time. George Washington’s Tent multimedia presentation at the Museum of the American Revolution in Philadelphia.
7)      Question and answer period
							    \item [Engagement:]We have a three-pronged approach to audience participation:
1)      Standard session presentation with slide deck
2)      Hardware hands-on demo
3)      Q\&A section
							    \item [Relationship to Theme:]With the theme "Forward," we posit that to look forward, you must first look back for perspective. If one does not protect and honor historic objects that inform our present and future, we are limited by an act of self-inflicted destruction.

This presentation is an example presenting best practices in museums.
							    
                    \item [Additional Comments: ]**We are happy to receive your suggestions.**
**Our panel consists of three lighting experts:**
**1)      Our Moderator is the lighting manager of the Academy Museum of Motion Pictures. Omar Madkour has been on staff since long before the museum opening and was party to every lighting design choice.**
**2)      Kate Furst has recently opened the Los Angeles studio of Available Light, a lighting design firm specializing in museum exhibition and architecture. Ms. Furst has over a decade’s experience working in the field.**
**3)      Steven Rosen is the president and founder of Available Light. He has been storytelling with light in museums for over 30 years. He has presented on the topic of artifact lighting at SEMC, NEMA, and several conservator meetings including an ICOM/ICAMT conference in Europe.**

                \end{description}
              \subsection*{Audience}
                \begin{description}
                  \item [Audiences:]Curators/Scientists/Historians~Educators~Emerging Museum Professionals~Exhibit Designers, Installers, Fabricators~Facilities Management Personnel~Registrars, Collections Managers~Technology~
                  \item[Professional Level:]All professional levels~Mid-Career~Senior Level~Student~
                \item[Scalability:] This is pertinent information of institutions of all sizes.   **If your institution displays delicate objects, this session is for you.**

							
              \end{description}
            \subsection*{Participants}
              \subsubsection*{ Kate Furst }
              Submitter, Presenter\newline
              Managing Associate Principal, Los Angeles\newline
              Available Light, Los Angeles, California
              \newline
              kate@availablelight.com\newline
              steven@availablelight.com\newline
              513-720-5323\newline

              Kate Furst has recently opened the Los Angeles studio of Available Light, a lighting design firm specializing in museum exhibition and architecture. Ms Furst has over a decade’s experience working in the field.\newline


              
                \subsubsection*{ Omar Madkour }
                Moderator, Presenter\newline
                Manager, Lighting Design\newline
                Academy Museum of Motion Pictures, Los Angeles, California
                \newline
                omadkour@oscars.org\newline
                
                310-247-3000 x4205\newline

                Our Moderator is the lighting manager of the Academy Museum of Motion Pictures. Omar Madkour has been on staff since long before the museum opening and was party to every lighting design choice.\newline
                \emph{ (confirmed) }
              

              
                \subsubsection*{ Steven Rosen }
                Presenter\newline
                President and Creative Director\newline
                Available Light, Boston, Massachusetts
                \newline
                Steven@availablelight.com\newline
                
                617-944-6800 x123\newline

                Steven Rosen is the president and founder of Available Light. He has been storytelling with light in museums for over 30 years. He has presented on the topic of artifact lighting at SEMC, NEMA, and several conservator meetings including an ICOM/ICAMT conference in Europe.
                \emph{ (confirmed) }
              

              

              

              
        
          \newpage
          \section{ Vessels of Memory: Enhancing Engagement for Regional, Community-centered Museums     }
            \begin{description}
              \item [ID:]
              WMA2022\_219

              \item [Assigned to:]
                \item [Track:]
              \end{description}

              <strong>This panel will discuss the evolving role of regional and community-centered institutions as vessels of memory and spaces to convene, as the panel questions and reflects on the power of place.</strong>
** **
**Sharing the perspectives from museum directors, arts council leaders, and designers who support their missions, the discussion will center on what new architecture and innovative design thinking at all scales can bring to these institutions to elevate their voice and fulfill their purpose. **

              \subsection*{Session Information}
                \begin{description}
                  \item [Format:] Regular session/panel (roundtable, single speaker, etc.)
							    
								  \item [Fee:]No fee
							     
							    \item [Uniqueness:]The panel will share insights from the leadership responsible for the award-winning Corvallis Museum—directors, fundraisers, architects, exhibit designers—with additional regional case studies.
							    \item [Objectives:]1— Context and community response—approaches to making institutions more resonant and relevant to the places and people they serve
 
2—Understanding design and fundraising strategies for smaller-scale institutions—learning to maximize value and innovate within constraints
 
3— The session will offer lessons in tailoring museum spaces and exhibit installations to the needs of a collection and institutional mission, and working collaboratively to achieve a shared vision
							    \item [Engagement:]We will encourage an open discussion by including a range of perspectives on the planning, design, and development process, and inviting participants to share their own experiences and insights.  

Additionally, as many of the institutions / case studies are proximate to the conference location in Portland, we will also work to arrange special visits or opportunities for attendees to explore the region—the Corvallis Museum in particular.
							    \item [Relationship to Theme:]The session embraces the theme FORWARD by focusing on transformation—the impact that new architecture and exhibition approaches have on an institution’s capabilities, development strategies, identity and relationship to their audience and region. 

We will describe the transformation of a cloistered, remote  collection into a vibrant exhibition, education, community, and cultural destination: the new Corvallis Museum. The vision and leadership at all levels contributed to the creation of a successful and vibrant new cultural center.
							    
                    \item [Additional Comments: ]**We would be willing to merge or hybridize our presentation if there are other proposers / presenters covering similar themes. This proposal is really only a starting point for further development with our colleagues and with WMA.**

                \end{description}
              \subsection*{Audience}
                \begin{description}
                  \item [Audiences:]Board Members~Curators/Scientists/Historians~Development and Membership Officers~Directors/Executive/C-Suite~Emerging Museum Professionals~Exhibit Designers, Installers, Fabricators~General Audience~Programs~Registrars, Collections Managers~Technology~Visitor Services~
                  \item[Professional Level:]All professional levels~Emerging Professional~General Audience~Mid-Career~Museum board members~Senior Level~Student~
                \item[Scalability:] **While centering on smaller case studies, the session will offer valuable and scalable lessons on the commitment and coordination required to plan, fund, design, and curate community focused institutions.**

							
              \item[Other Comments:] Museum leaders that are interested in reimagining, refreshing, or expanding their institution. The panel brings a variety of experience and perspectives on ways to design and program museums for approachability to attract new audiences and encourage engagement with collections.

In addition, for Sessio Format - We will explore ways to break the wall between presenters and participants in the Standard Panel format—the “Fishbowl” model might work well after an initial visual presentation / prompts on the subject matter. This first series of prompts could adopt a “PechaKucha” style format, with multiple presenters highlighting their own perspective before opening up the discussion.
              \end{description}
            \subsection*{Participants}
              \subsubsection*{ Chelsea Grassinger }
              Submitter, Presenter\newline
              Principal\newline
              Allied Works, Portland, Oregon
              \newline
              chelsea@alliedworks.com\newline
              
              (503) 888-2509\newline

              Chelsea is the Allied Works Principal-in-Charge on a variety of arts and cultural projects such as the Penn State University, Palmer Museum of Art; Corvallis Museum, Benton County Historical Society; National Veterans Memorial Museum; and the University of Michigan Museum of Art. She is an expert on what new architecture and innovative design thinking at all scales can bring to museum institutions to elevate their voice and mission.\newline


              
                \subsubsection*{ Chris D'Arcy }
                Moderator, Presenter\newline
                Former Executive Director\newline
                Oregon Arts Commission, Salem, Oregon
                \newline
                darcychris777@gmail.com\newline
                
                (503) 510-3633\newline

                Chris D’Arcy brings over 25 years of experience in public sector arts and culture leadership. As the former Executive Director of the Oregon Arts Commission, where she served for 20 years, her knowledge of cultural institutions throughout the West will be a valuable lens to facilitate the discussion.\newline
                \emph{ (confirmed) }
              

              
                \subsubsection*{ Irene Zenev }
                Presenter\newline
                Former Executive Director\newline
                Corvallis Museum, Benton County Historical Society, Corvallis, Oregon
                \newline
                irenezenev@gmail.com\newline
                
                (541) 554-3690\newline

                Irene was the catalyst for housing the OSU Horner Collection in a new building and championed the fundraising efforts to make the new Corvallis Museum a reality. She will speak to her experience leading a historical society institution, working with local collections and fundraising to realize a new and innovative building for the Benton County Historical Society.
                \emph{ (confirmed) }
              

              
                \subsubsection*{ Chelsea Grassinger }
                Presenter\newline
                Principal\newline
                Allied Works, Portland, Oregon
                \newline
                chelsea@alliedworks.com\newline
                
                (503) 888-2509\newline

                Chelsea is the Allied Works Principal-in-Charge on a variety of arts and cultural projects such as the Penn State University, Palmer Museum of Art; Corvallis Museum, Benton County Historical Society; National Veterans Memorial Museum; and the University of Michigan Museum of Art. She is an expert on what new architecture and innovative design thinking at all scales can bring to museum institutions to elevate their voice and mission.
                \emph{ (confirmed) }
              

              
                \subsubsection*{ Anne Bernard }
                Presenter\newline
                Principal\newline
                Project Renate, Portland, Oregon
                \newline
                abernard@projectrenate.com\newline
                
                (917) 974-9516\newline

                Renate’s Principal Anne leads the development of master planning and museum audience engagement through creative content, exhibit development and spatial planning. Her expertise in shaping and managing experiences has developed through projects that require the highest levels of visitor engagement, scientific accuracy, and cultural sensitivity. She brings in-depth knowledge of museum collections and research, as well as an understanding of institutional and educational challenges and goals.
                \emph{ (confirmed) }
              

              
        
          \newpage
          \section{ Powering Culture by Cutting Carbon: Data Tools and Policy Snapshot }
            \begin{description}
              \item [ID:]
              WMA2022\_220

              \item [Assigned to:]
                \item [Track:]
              \end{description}

              Measuring and reducing GHG emissions from energy use is critical to empower cultural institutions to reduce their contributions to climate change. <em>Culture Over Carbon</em> is an IMLS National Leadership Grant project born in the PNW. Join this introductory-level session on why and how to measure and monitor your building’s energy and hear examples of how future energy management and local policy expectations can affect energy and capital planning.

              \subsection*{Session Information}
                \begin{description}
                  \item [Format:] Regular session/panel (roundtable, single speaker, etc.)
							    
							    \item [Uniqueness:]Understanding how energy usage data and local building policies allows museum leaders to make forward-looking strategic energy management decisions to save money and GHG emissions.
							    \item [Objectives:]1. Take advantage of Energy Star Portfolio Manager (a free resource) to measure energy and water use and evaluate energy greenhouse gas emissions.
2. Understand why measuring and monitoring institutional energy use relates to climate change, your institution’s bottom line, and planning.
3. Learn why preparing for future building energy policy is an opportunity for cultural institutions.
							    \item [Engagement:]- Conduct poll (raise hands):
    - Attendees collect energy data
    - Attendees use EnergyStar Portfolio Manager
    - Have made decisions re: energy use and planning
    - Knowledge of energy policies
- Audience Q\&A (15 mins)
							    \item [Relationship to Theme:]Forward: Culture Over Carbon will establish an energy carbon footprint of the sector and provide a broader understanding of the sector’s energy practice to recommend ways to plan for future changes related to energy availability, policies, and regulations. Being able to acquire and understand this data allows museum leaders to make forward-looking strategic energy management decisions which can save money and/or reduce energy GHG emissions which contribute to climate change.
							    
                \end{description}
              \subsection*{Audience}
                \begin{description}
                  \item [Audiences:]Board Members~Curators/Scientists/Historians~Development and Membership Officers~Directors/Executive/C-Suite~Educators~Emerging Museum Professionals~Events Planning~Exhibit Designers, Installers, Fabricators~Facilities Management Personnel~General Audience~Marketing \& Communications (Including Social Media)~Programs~Registrars, Collections Managers~Technology~Visitor Services~
                  \item[Professional Level:]All levels~
                \item[Scalability:] Any organization can record energy data from energy bills in the free software from the EPA: Energy Star Portfolio Manager. Using this resource allows an institution to measure and monitor energy use (and water and waste) and to calculate carbon emissions from them. Institutions can then make decisions related to reducing energy use and planning using the data and learning about future energy policy.

							
              \item[Other Comments:] Any level of museum professional interested in helping their institution: measure energy use, plan to reduce energy use, contribute to or make capital planning decisions, consider climate-smart practices, and learn about how energy policy is related to the cultural sector.
              \end{description}
            \subsection*{Participants}
              \subsubsection*{ Sarah Sutton }
              Submitter, Moderator, Presenter\newline
              CEO\newline
              Environment \& Culture Partners, Tacoma, WA
              \newline
              sarah@ecprs.org\newline
              hello@ecprs.org\newline
              978-505-4515\newline

              Sarah represents Environment and Culture Partners who is part of the Culture Over Carbon leadership team. Sarah helped to design the project and develop the IMLS application. ECP is one of the partners of the project and will talk how important this work is to the field and in the frame of climate change, and will set the scene for the other panelists.\newline


              
                \subsubsection*{ Sarah Sutton }
                Moderator, Presenter\newline
                CEO\newline
                Environment and Culture Partners, Tacoma, WA
                \newline
                sarah@ecprs.org\newline
                hello@ecprs.org\newline
                978-505-4515\newline

                Sarah represents Environment and Culture Partners who is part of the Culture Over Carbon leadership team. Sarah helped to design the project and develop the IMLS application. ECP is one of the partners of the project and will talk how important this work is to the field and in the frame of climate change, and will set the scene for the other panelists.\newline
                \emph{ (confirmed) }
              

              
                \subsubsection*{ Webly Bowles }
                Presenter\newline
                Senior Project Manager\newline
                New Buildings Institute (NBI), Portland, OR
                \newline
                webly@newbuildings.org\newline
                
                503-999-7520\newline

                Webly helps manages the technical aspects of Culture Over Carbon. She reads and interprets energy data, understands how buildings use energy, and can communicate how the impact of understanding energy policies.
                \emph{ (confirmed) }
              

              
                \subsubsection*{ Cory Gooch }
                Presenter\newline
                Chief Registrar/Head of Collections\newline
                Frye Museum of Art, Seattle, WA
                \newline
                cgooch@fryemuseum.org\newline
                
                206-432-8248\newline

                Cory is one of the eight advisory panel members for the CoC project. She can speak to the challenges of recruiting cultural institutions to the project. Additionally, the Frye Art Museum is one of the institutional participants in the project.  Cory is interested in lowering museum energy uses and costs, especially for art museums that have traditionally required very strict internal climate parameters.
                \emph{ (confirmed) }
              

              
                \subsubsection*{ Victoria Coats }
                Presenter\newline
                Manager of Exhibit Research and Development\newline
                Oregon Museum of Science and Industry, Portland, OR
                \newline
                TBD\newline
                vcoats@omsi.edu\newline
                503-797-4584\newline

                OMSI is a participant in the Culture Over Carbon project and is also a leader in environmental sustainability and climate-smart practice work. OMSI can share their experiences while also focusing on the importance of monitoring energy data, making energy management decisions, and how policy impacts the institution.
                \emph{ (not confirmed) }
              

              
        
          \newpage
          \section{ Tribal Consultation: Returning Digital Records and Building Authentic Relationships  }
            \begin{description}
              \item [ID:]
              WMA2022\_221

              \item [Assigned to:]
                \item [Track:]Indigenous~
              \end{description}

              Tribal consultation, employed by museums as a NAGPRA requirement over thirty years ago, has expanded to include all forms of cultural heritage including archival material. Tribes and museums routinely expect it. However, the actual implementation of consultation remains challenging based on the continued lack of understanding of relationship building. Panelists will discuss lessons learned and issues related to the return of digital records that impact reparative language, metadata and search terms.

              \subsection*{Session Information}
                \begin{description}
                  \item [Format:] Regular session/panel (roundtable, single speaker, etc.)
							    
							    \item [Uniqueness:]Consultation from a museum perspective and a Tribal perspective can be vastly different. Being aware of the nuances from both perspectives can influence collection practices.
							    \item [Objectives:]The audience will be introduced to key concepts of the Indigenous perspective of consultation (time, place, context, communication, relationship) and learn how to apply them to their own consultation process.
Using some of the key concepts of consultation, the audience will learn how archival terms and other information systems can be improved to promote user accessibility, especially when working with Indigenous collections.
							    \item [Engagement:]The audience will have time to engage with a fifteen-minute Q \&A/ sharing of experiences, after the panelists complete their presentation. The session moderator will be responsible for facilitating this portion.
							    \item [Relationship to Theme:]The session relates to the theme of “Forward”, by encouraging relationship building. This skill is critical to embracing diverse perspectives.
							    
                \end{description}
              \subsection*{Audience}
                \begin{description}
                  \item [Audiences:]Registrars, Collections Managers~
                  \item[Professional Level:]All levels~
                \item[Scalability:] The target audience includes, museum librarians and archivists, collection managers, repatriation programs

							
              \item[Other Comments:] The target audience includes, museum librarians and archivists, collection managers, repatriation programs
              \end{description}
            \subsection*{Participants}
              \subsubsection*{ Alyce Sadongei }
              Submitter, Moderator, Presenter\newline
              \newline
              Arizona State Museum and American Indian Language Development Institute, Tucson, AZ
              \newline
              sadongei@arizona.edu\newline
              
              5206264145\newline

              I will be contributing to the presentation on key principles related to tribal consultation based on an Indigenous perspective that I will also provide.\newline


              
                \subsubsection*{ Alyce Sadongei }
                Moderator, Presenter\newline
                Co-Director/Project Manager\newline
                Arizona State Museum/American Indian Language Development Insitute, Tucson, AZ
                \newline
                sadongei@arizona.edu\newline
                
                5206264145\newline

                The moderator has extensive experience in working with Tribes, tribal museums, consultation and repatriation. The moderator is currently the co-director of an Arizona State Museum grant funded project focusing on returning digital recordings to Tribes.\newline
                \emph{ (confirmed) }
              

              
                \subsubsection*{ Molly Stothert-Maurer }
                Presenter\newline
                Associate Librarian/Archivist, Head of Library and Archives\newline
                Arizona State Museum, Tucson, AZ
                \newline
                stothert@arizona.edu\newline
                
                5206214695\newline

                She is currently the co-director of of an Arizona State Museum grant funded project focusing on returning digital recordings to Tribes.
                \emph{ (confirmed) }
              

              
                \subsubsection*{ Kate Stewart }
                Presenter\newline
                Project Archivist\newline
                Arizona State Museum, Tucson, AZ
                \newline
                kvstewart@arizona.edu\newline
                
                5206217142\newline

                of an  She is currently the project archivist for an Arizona State Museum grant fund project focusing on returning digital recordings to Tribes.
                \emph{ (confirmed) }
              

              

              
        
          \newpage
          \section{ Is there a Rembrandt in My Wine Glass? (Proposed as an Happy Hour Experience) }
            \begin{description}
              \item [ID:]
              WMA2022\_222

              \item [Assigned to:]
                \item [Track:]
              \end{description}

              During this one-hour multimedia art and wine pairing experience, you will hear uplifting stories, taste wine and view works of art that capture the essence of key themes found in art history. This high-energy session parallels experimental artists and winemakers pushing the boundaries of technology in their respective worlds. The perfect way to fire up your synapses, hit the escape button and connect with colleagues new and old.

              \subsection*{Session Information}
                \begin{description}
                  \item [Format:] Regular session/panel (roundtable, single speaker, etc.)
							    
								  \item [Fee:]Price of a glass of wine or alcohol-free beverage. No speaker fee. 
							     
							    \item [Uniqueness:]A wholly-original experience designed as a riddle for your senses that takes you on a playful exploration of surprising connections between the visual and culinary arts.
							    \item [Objectives:]By the end of this experience, your attendees will:

discover how to expand their “sensory toolkits” and reconnect with the magic of the world around them.

jumpstart conversations that generate optimism for the future and widen the WMA circle of community.

cultivate a new appreciation for West Coast artists and winemakers and leave equipped with ways to explore on their own. 

reboot their creative thinking by exploring the interplay of the senses leaving refreshed and re-energized.
							    \item [Engagement:]"Is there a Rembrandt in My Wine Glass?" is proposed and built in a one-hour facilitated networking (cocktail reception) format. A combination of guided sensory exploration and storytelling (20-30minutes) is interspersed with creative prompts devised to spark conversation and build community. Break-outs give guests 30-40 minutes to mix and mingle.

Resources needed: flat-panel TV screen (minimum 50”), lavalier, speaker (for voice and music with input from Apple laptop). Assist with restaurant location and beverage coordination.
							    \item [Relationship to Theme:]Content will align with WMA’s 2022 aim to connect and move FORWARD. Attendees will find out what artists do when they run out of ideas, how they move forward, and what it takes to break new ground. We will test snap reactions and generate new ideas you can take into your community. West Coast artists and winemakers will be spotlighted.
							    
                    \item [Additional Comments: ]This presentation stands on its own, but I'm open to ideas.

                \end{description}
              \subsection*{Audience}
                \begin{description}
                  \item [Audiences:]Development and Membership Officers~Emerging Museum Professionals~General Audience~Marketing \& Communications (Including Social Media)~
                  \item[Professional Level:]All levels~
                \item[Scalability:] Is there a Rembrandt in My Wine Glass? will satisfy attendees who enjoy: 
-a fun and relaxing event at the end of the day
-the chance to socialize within a format that provides conversation icebreakers
-light one-on-one conversation with fellow attendees (group sharing optional)
-seeing a unique high production value production that features photography, video and music; built by a woman entrepreneur.
-conversation topics grounded in sensory discovery and optimism for the future

							
              \end{description}
            \subsection*{Participants}
              \subsubsection*{ Kirsten Shilakes }
              Submitter, Moderator, Presenter\newline
              Curator of Multimedia, Art, Food \& Beverage Experiences\newline
              The Arts Lounge, Mill Valley, California
              \newline
              kshilakes@artslounge.com\newline
              kirsten@shilakes.net\newline
              415-465-0309\newline

              Kirsten Shilakes has led bespoke multi-sensory art and wine experiences for scores of audiences including Four Seasons hotel guests, SFMOMA members, Napa Valley wine enthusiasts. She is an American Alliance of Museums, Gold MUSE Award recipient for “excellence in recognition of excellence in the use of media \& technology for public relations and development.”  Kirsten is a contributor to “Muse in a Stem Glass” in Wine \& Philosophy, Blackwell Publishing featuring iconic wine-related artworks from San Francisco museum collections. 
 
She has delivered on complex creative projects ranging from large-scale stage events to small, intimate gatherings. Kirsten studied wine at the Culinary Institute of America, Napa Valley and the art of sake in Kamakura, Japan; she is a certified sake specialist. As a former docent lecturer at the Asian Art Museum and Fine Arts Museums of San Francisco she underwent six years of museum training ranging from contemporary American to the art of East Asia. Kirsten has served on the Jade Circle, Asian Art Museum and ArtTable of Northern California development committees. She earned her B.A. from U.C. Berkeley in Political Economy.

I’m keenly interested in finding innovative ways to bring forward artists, artisans, winemakers and the arts organizations behind them. I’d be honored to present this post-Covid evolution of "Is there a Rembrandt in My Wine Glass?" as a way to support the Western Museums Association. Thank you for your consideration.\newline


              

              
                \subsubsection*{ Kirsten  Shilakes }
                Presenter\newline
                Curator of Multimedia Art, Food \& Beverage Experiences\newline
                The Arts Lounge, Mill Valley, California
                \newline
                kshilakes@artslounge.com\newline
                kirsten@shilakes.net\newline
                415-465-0309\newline

                Kirsten Shilakes has led bespoke multi-sensory art and wine experiences for scores of audiences including Four Seasons hotel guests, SFMOMA members, Napa Valley wine enthusiasts. She is an American Alliance of Museums, Gold MUSE Award recipient for “excellence in recognition of excellence in the use of media \& technology for public relations and development.” Kirsten is a contributor to “Muse in a Stem Glass” in Wine \& Philosophy, Blackwell Publishing featuring iconic wine-related artworks from San Francisco museum collections.

She has delivered on complex creative projects ranging from large-scale stage events to small, intimate gatherings. Kirsten studied wine at the Culinary Institute of America, Napa Valley and the art of sake in Kamakura, Japan; she is a certified sake specialist. As a former docent lecturer at the Asian Art Museum and Fine Arts Museums of San Francisco she underwent six years of museum training ranging from contemporary American to the art of East Asia. Kirsten has served on the Jade Circle, Asian Art Museum and ArtTable of Northern California development committees. She earned her B.A. from U.C. Berkeley in Political Economy.

I’m keenly interested in finding innovative ways to bring forward artists, artisans, winemakers and the arts organizations behind them. I’d be honored to present this post-Covid evolution of "Is there a Rembrandt in My Wine Glass?" as a way to support the Western Museums Association. Thank you for your consideration.
                \emph{ (confirmed) }
              

              

              

              
        
          \newpage
          \section{ Politics of Land Acknowledgements }
            \begin{description}
              \item [ID:]
              WMA2022\_223

              \item [Assigned to:]
                \item [Track:]
              \end{description}

              More and more museums, higher learning institutions, government entities, and organizations are adopting “land acknowledgements” to recognize indigenous populations associated with specific geographic areas.  While the intent is noble, land acknowledgments can be problematic in unintended ways.  This session will explore the benefits and pitfalls in creating land acknowledgments.  Panelists will discuss the process and challenges of developing land acknowledgments for their respective institutions.

              \subsection*{Session Information}
                \begin{description}
                  \item [Format:] Regular session/panel (roundtable, single speaker, etc.)
							    
								  \item [Fee:]No
							     
							    \item [Uniqueness:]Institutions are adopting land acknowledgements often as a form of moral exhibitionism.  Land acknowledgments ring hollow without actions and education to support them.
							    \item [Objectives:]Objective 1:  Recognize that First Americans live in the present tense. Land acknowledgements often mention Native peoples only in an historical context. We are still here!  

Objective 2:  Land acknowledgments have their roots in Judeo-Christian theology, that "man" has dominion over the land and all living things.  The Western world perceives land as a commodity, something that can be owned.  This is counter-intuitive in the way many First American communities view the world.  Land is not a commodity, but rather people belong to the land, not the other way around.

Objective 3:  Land Acknowledgements are just words unless accompanied by action. Consider how your group, business or organization can support First American communities and individuals through volunteering, awareness or material support.
							    \item [Engagement:]Panelists will discuss how to develop a land acknowledgment statement with a list of criteria that will be presented and distributed during the session.  Panelists will present samples of land acknowledgements and dissect them, highlighting elements that are appropriate, and ones that are problematic. Panelists will also consider the concept of "People Acknowledgements" vs. "Land Acknowledgements".
							    \item [Relationship to Theme:]To move "Forward", museums must reconcile the past regarding colonization of indigenous lands and the displacement of indigenous peoples.  Moving "Forward," land acknowledgements are a gesture in that direction and the beginning of the healing process for First American communities that carry inter-generational trauma.
							    
                    \item [Additional Comments: ]We have four panelists confirmed.

                \end{description}
              \subsection*{Audience}
                \begin{description}
                  \item [Audiences:]General Audience~
                  \item[Professional Level:]General Audience~
                \item[Scalability:] This session topic is pertinent to every institution that resides on indigenous land and/or has relationships with indigenous communities. The session will provide Native and non-Native perspectives on this topic.

							
              \item[Other Comments:] This session is of value to any museum professional or institution considering developing a land acknowledgment statement or for those institutions that have already developed a land acknowledgement statement.  The trend in these current times is that the majority of museums will eventually adopt a land acknowledgment statement.
              \end{description}
            \subsection*{Participants}
              \subsubsection*{ James Pepper Henry }
              Submitter, Moderator, Presenter\newline
              Executive Director/CEO\newline
              First Americans Museum, Oklahoma City, Oklahoma
              \newline
              jph@famok.org\newline
              jph@pepperandassociates.com\newline
              (918) 978-1877\newline

              The submitter will describe experiences developing land acknowledgment statements for various institutions, including First Americans Museum, and will initiate a discussion on 'land acknowledgments' vs. 'people acknowledgements.'\newline


              
                \subsubsection*{ James Pepper Henry }
                Moderator, Presenter\newline
                Vice-Chairman\newline
                Kaw Nation (Federally recognized Native American tribe), Oklahoma City, OK
                \newline
                jph@famok.org\newline
                jph@pepperandassociates.com\newline
                (918) 978-1877\newline

                The moderator has 38 years experience working in the museum profession, including 17 years as an executive director of some of the most prestigious Native American and western museums in the U.S. including the Heard Museum, Gilcrease Museum, Anchorage Museum, First Americans Museum, and as an Associate Director of the Smithsonian National Museum of the Americans Indian.  The Moderator is also Native American and the Vice-Chairman of the Kaw Nation, a federally recognized Native American tribe.\newline
                \emph{ (confirmed) }
              

              
                \subsubsection*{ James Pepper Henry }
                Presenter\newline
                Executive Director/CEO \& Vice-Chairman\newline
                First Americans Museum \& Kaw Nation, Oklahoma City, OK
                \newline
                jph@famok.org\newline
                jph@pepperandassociates.com\newline
                (918) 978-1877\newline

                Please see above
                \emph{ (confirmed) }
              

              
                \subsubsection*{ Adrienne Lalli Hills }
                Presenter\newline
                Director, Learning \& Community Engagement\newline
                First Americans Museum, Oklahoma City, OK
                \newline
                adriennelh@famok.org\newline
                
                (405) 594-2177\newline

                Adrienne is an experienced arts and museum leader with 15 years’ experience among art, children’s, history and science museums. She received her MA in Curriculum and Instruction from the University of Kansas City-Missouri and has held leadership positions in the Museum Education Roundtable, Museum Computer Network and the Association for Art Museum Interpretation. In addition to her role as Director of Learning \& Community Engagement at the First Americans Museum, Adrienne serves as an independent interpretive consultant for Delaware Art Museum, Denver Art Museum, National Museum of the American Indian and Peabody Essex Museum. She is an enrolled citizen of Wyandotte Nation.
                \emph{ (confirmed) }
              

              
                \subsubsection*{ Meranda Roberts }
                Presenter\newline
                Native American Museum Consultant\newline
                Formerly the co-curator of the Native American Hall, Field Museum of Natural History, Ontario, CA
                \newline
                m.roberts005@gmail.com\newline
                
                (909) 262-1283\newline

                Meranda Roberts is an enrolled member of the Yerington Paiute Tribe in Nevada, as well as Chicana. In 2018, she earned her PhD from the University of California, Riverside in Native American History. Over the past few years, she has been working as a post-doctoral researcher at the Field Museum of Natural History, where she is developing content for the renovation of the museum’s seventy-year-old Native American exhibition hall. Meranda’s passion lies in holding colonial institutions, like museums, accountable for the harmful narratives they have painted about Indigenous people. She is also dedicated to reconnecting Indigenous collection items with their descendants. Through the use of Indigenous methodologies, as well as public history pedagogy, Meranda examines the harm colonialism continues to inflict on Indigenous communities and how public institutions can correct these wrongs.
                \emph{ (confirmed) }
              

              
                \subsubsection*{ Mike Murawski }
                Presenter\newline
                Museum Consultant\newline
                Self-Employed Museum Educator, Portland, OR
                \newline
                murawski27@gmail.com\newline
                
                (503) 679-1090\newline

                Mike Murawski is a consultant and educator living in Portland, Oregon. He is the author of Museums as Agents of Change: A Guide to Becoming a Changemaker (2021), author of the Substack publication Agents of Change, and co-producer of Museums Are Not Neutral, a global advocacy campaign calling for equity-based transformation across museums. In 2016, he co-founded Super Nature Adventures, a place-based education and creative design agency that partners with parks, government agencies, schools, and non-profits to expand learning in the outdoors and public spaces.
                \emph{ (confirmed) }
              
        
          \newpage
          \section{ Building Community: Discovering Resources for Professional Support, Learning and Development  }
            \begin{description}
              \item [ID:]
              WMA2022\_224

              \item [Assigned to:]
                \item [Track:]Other~
              \end{description}

              Having a network of colleagues outside of our immediate co-workers was crucial when disaster struck. The members of the Museum Educators of Puget Sound have leaned into this community for support, resources, and information during the COVID pandemic. 
Join us for open conversation on what we learned about our identity as museum educators, how we supported each other during different phases of the pandemic, and how to decide what we as Educators can carry forward. 

              \subsection*{Session Information}
                \begin{description}
                  \item [Format:] Regular session/panel (roundtable, single speaker, etc.)
							    
								  \item [Fee:]n/a
							     
							    \item [Uniqueness:]MEPS, a regional professional community run by informal educators for informal educators, experimented to respond to the needs of its members through 2020 and today.
							    \item [Objectives:]Objectives:
1)   Participants will hear and discuss how local communities of practice responded to a variety of challenges during the COVID pandemic
2) Participants will collaboratively consider and discuss ways to create and support their agency in professional development – even without the support of formal structures- by using local communities of practice. 
3) Participants will identify actionable steps they can take to seek or build community resources for professional learning.   
Outcomes:
Participants will connect and converse with presenters and each other, and through this impromptu community of practice, reflect on the changes to our field and their experiences over the last few years, and how to develop and engage in responsive and locally meaningful professional development within their own networks, to better serve their immediate local professional needs.
Take-aways:
--  Examples of how MEPS and MEPS members collaborated to meet the changing and complicated needs of our local professional community of practice
- An opportunity to experience the MEPS practice of bringing people together for an open conversation with the goals of learning from each other and developing professional skills around current issues and practices.
-    A personal reflection on their own next steps to seek or build community resources for professional learning 
-  A reminder of actionable steps from themselves mailed in 6 months to touch base on their session takeaways and inspire action
							    \item [Engagement:]After the opening conversation, participants will self-select into small groups to discuss current conversations in the MEPS community, facilitated by presenters.
Participants will share and discuss their experience and ideas that guide their own DIY professional development and support informal local communities of practice, and a personal reflection on their own next steps. 
They will receive their personal reflection in the mail 6 months after this conference.
							    \item [Relationship to Theme:]The model we will discuss is an example of innovative engagement of how museum professionals can foster ongoing conversations, inspire forward-thinking ideas by taking agency in their professional development, by engaging with local communities of practice, and responsively supporting each other and promoting current and relevant learning.  
By empowering ourselves and our local peers, we as professionals support a more responsive and collaborative field, which then better serves our larger communities.
							    
                \end{description}
              \subsection*{Audience}
                \begin{description}
                  \item [Audiences:]Educators~Emerging Museum Professionals~General Audience~Programs~
                  \item[Professional Level:]All levels~
                \item[Scalability:] Museum Educators of Puget Sound (MEPS) is composed of educators from the Puget Sound region's many cultural, arts, historical, zoological, and scientific institutions. Our members come from institutions ranging from one-person organizations to those with over 100 employees. We have experience scaling learning and support to many different types of organizations and their unique needs and space in the learning ecosystem.

							
              \end{description}
            \subsection*{Participants}
              \subsubsection*{ Sondra  Snyder }
              Submitter, Moderator, Presenter\newline
              Director of Education\newline
              Museum of History and Industry, Seattle, WA
              \newline
              sondra.snyder@mohai.org\newline
              sondra.snyder@gmail.com\newline
              206 3241126 ext 113\newline

              Sondra has been the President of MEPS since May of 2014 and was part of steering MEPS during the COVID pandemic. She routinely facilitates MEPS meetings and conversations.\newline


              
                \subsubsection*{ Sondra Snyder }
                Moderator, Presenter\newline
                Director of Education\newline
                MOHAI - Museum of History \& Industry, Seattle, WA 
                \newline
                sondra.snyder@mohai.org\newline
                
                206.324.1126 ext 113\newline

                Sondra has been the President of MEPS since May of 2014 and was part of steering MEPS during the COVID pandemic. She routinely facilitates MEPS meetings and conversations.\newline
                \emph{ (confirmed) }
              

              
                \subsubsection*{ Emily  Turner  }
                Presenter\newline
                K-12 \& Youth Programs Coordinator\newline
                MOHAI - Museum of History \& Industry , Seattle, WA
                \newline
                emily.turner@mohai.org\newline
                
                206 324 1126 ext 112\newline

                Emily has been the MEPS Officer o’Fun for 3 years and was part of steering MEPS during the COVID pandemic. She routinely facilitates MEPS meetings, conversations, and fun-tivities.
                \emph{ (confirmed) }
              

              
                \subsubsection*{  TBD - we are putting a call out to our membership base for a member perspective  }
                Presenter\newline
                \newline
                , Seattle, WA 
                \newline
                museumeducators@gmail.com\newline
                
                

                We would like to present a well-rounded experience. Will have either 2 members, or another officer and a member
                \emph{ (not confirmed) }
              

              

              
        
          \newpage
          \section{ Sharing Stories: Collaborating with Tribes and Community Partners to Develop Cultural Curricula }
            \begin{description}
              \item [ID:]
              WMA2022\_225

              \item [Assigned to:]
                \item [Track:]
              \end{description}

              Our session will focus on collaborations between museums, tribes, and community partners to co-develop cultural programs. The content will focus on the development of the \_Big River \&amp; Its People \_curriculum about Celilo Falls. Presenters from each of the four partner organizations\&ndash;Confederated Tribes of Warm Springs, Museum at Warm Springs, Confluence, and High Desert Museum\&ndash;will share how we developed the curriculum together and what we\&rsquo;ve learned through the process.

              \subsection*{Session Information}
                \begin{description}
                  \item [Format:] Regular session/panel (roundtable, single speaker, etc.)
							    
								  \item [Fee:]NA
							     
							    \item [Uniqueness:]This session provides a concrete example from museum and partner perspectives of what it can look and feel like to Indigenize museum processes and programs.
							    \item [Objectives:]1. Attendees will understand the structure and purpose of working closely with Tribal and community partners to co-develop cultural programs/curricula.
2. Attendees will recognize “lessons learned” from the case study that they can take into their own practice.
3. Attendees will feel inspired to explore collaborations with Tribes and/or cultural organizations in their area to deepen community relationships and diversify the perspectives represented in their programs.
							    \item [Engagement:]The program will use storytelling to engage audiences and help them understand both the process and human elements of the collaborations. Presenters will include humor and acknowledge mistakes to highlight the lessons embedded in the experience.
 
We will also include opportunities for small group discussions for participants to share their own experiences and insights with museum/community collaborations.
 
We will leave time for Q/A at the end.
							    \item [Relationship to Theme:]Sharing and learning from examples of deep collaboration is critical for museums to move FORWARD. Our institutions can only truly reflect and engage our diverse communities by building relationships that help communities heal, tell important stories, and envision a more inclusive, just future.
							    
                    \item [Additional Comments: ]This proposal is a preliminary outline for the session. As with the project we are sharing, the final session would be created collaboratively by the four partners. Therefore, the theme and format could change during the collaborative process to reflect the needs and vision of the group. We intend to include representation from all four organizations, but do not want to include partner names until we've had more opportunity for discussion. We will confirm who will present from partner organizations as possible. If partners cannot participate in person, we will explore other ways to include them in the session via recorded stories, prepared statements, or video call.
  

                \end{description}
              \subsection*{Audience}
                \begin{description}
                  \item [Audiences:]Directors/Executive/C-Suite~Diversity and inclusion specialists~Educators~General Audience~Programs~
                  \item[Professional Level:]All levels~
                \item[Scalability:] All museums need to embrace community collaborations and can learn/share through this session.

							
              \item[Other Comments:] NA
              \end{description}
            \subsection*{Participants}
              \subsubsection*{ Kyrie Kellett }
              Submitter, Presenter\newline
              Principal\newline
              Mason Bee Interpretive Planning, Portland
              \newline
              connect@masonbeellc.com\newline
              kyrie.kellett@gmail.com\newline
              15034197735\newline

              Kyrie is one of the lead curriculum developers for the project. She will help tell the story and facilitate the collaborative process to develop the session.\newline


              

              
                \subsubsection*{ Kyrie Kellett }
                Presenter\newline
                Principal\newline
                Mason Bee Interpretive Planning, Portland, OR
                \newline
                connect@masonbeellc.com\newline
                kyrie.kellett@gmail.com\newline
                503.419.7735\newline

                Kyrie is one of the lead curriculum developers for the project. She will help tell the story and facilitate the collaborative process to develop the session.
                \emph{ (confirmed) }
              

              
                \subsubsection*{ Christina Cid }
                Presenter\newline
                Director of Programs\newline
                High Desert Museum, Bend, OR
                \newline
                ccid@highdesertmuseum.org\newline
                
                541.328.4754 x233\newline

                Christina plays a critical role at the High Desert Museum in building the partnerships necessary to support this project and cultivating the larger vision of Indigenizing the museum's processes and programs.
                \emph{ (confirmed) }
              

              

              
        
          \newpage
          \section{ Forward Thinking Disaster Preparation: PNW Case Studies }
            \begin{description}
              \item [ID:]
              WMA2022\_226

              \item [Assigned to:]
                \item [Track:]
              \end{description}

              <strong>The focus of this panel is to discuss forward-thinking ideas that can shape how we can address disasters in new ways. Hear case studies, lessons learned, and strategies in the face of common disasters as well as climate change. Discussions focus on one tribe\&rsquo;s response to climate change, and how regional networks aid in disaster preparation and response.</strong>

              \subsection*{Session Information}
                \begin{description}
                  \item [Format:] Regular session/panel (roundtable, single speaker, etc.)
							    
							    \item [Uniqueness:]In this unique session, hear how one Tribal Nation is preparing for impending disasters, and how regional emergency response networks can save collections.
							    \item [Objectives:]1) Learn from colleagues in the field about local disaster incidents and responses through case study presentations.

2) Consider how to prepare for future disasters that aren’t typically in current plans.

3) Promote awareness of potential emergency and disaster situations and measures that can be taken to prevent, prepare for, and respond to disaster.
							    \item [Engagement:]Audience will be invited to submit questions in advance of the session, in addition to time at the end of the sessions to promote active dialogue.
							    \item [Relationship to Theme:]Disaster preparation is all about forward thinking in order to minimize risk to people and collections. This session will also include one panelist speaking to future actions and plans when faced with sea level rise/climate change.
							    
                \end{description}
              \subsection*{Audience}
                \begin{description}
                  \item [Audiences:]Curators/Scientists/Historians~Emerging Museum Professionals~Facilities Management Personnel~General Audience~Library Staff~Registrars, Collections Managers~
                  \item[Professional Level:]All professional levels~Emerging Professional~General Audience~Mid-Career~Senior Level~Student~
                \item[Scalability:] **Sadly, disasters occur everywhere. Lessons learned in these case studies can be applied anywhere on any scale. Learning from others is an invaluable tool to preventing and preparing for disaster situations. This session will feature experiences from museums, a Tribal Nation and libraries.**

							
              \end{description}
            \subsection*{Participants}
              \subsubsection*{ Siri Linz }
              Submitter, Moderator\newline
              Assistant Archaeology Collections Manager\newline
              Burke Museum of Natural History and Culture, Seattle/WA
              \newline
              linzs@uw.edu\newline
              sirilinz@gmail.com\newline
              (206) 685-3849 x2\newline

              The submitter will be moderating the panel discussion as well as the questions from the audience.\newline


              
                \subsubsection*{ Siri Linz }
                Moderator\newline
                Assistant Archaeology Collections Manager\newline
                Burke Museum of Natural History and Culture, Seattle/WA
                \newline
                linzs@uw.edu\newline
                
                (206) 685-3849 x2\newline

                As chairperson of the Seattle Heritage Emergency Response Network, a graduate of the Smithsonian's Heritage Emergency and Response Training, and a new member of the National Heritage Responders, this moderator has experience with the concepts presented in this session.\newline
                \emph{ (confirmed) }
              

              
                \subsubsection*{ Sarah and Nicole Frederick and Davis }
                Presenter\newline
                Collections Manager and Supervisory Archivist\newline
                The Museum of Flight, Seattle/WA
                \newline
                sfrederick@museumofflight.org\newline
                ndavis@museumofflight.org\newline
                206-768-7195\newline

                Sarah Frederick is the Collections Manager at the Museum of Flight (MoF) in Seattle. This February, MoF suffered a sprinkler failure in a gallery above a collections storage area. 1000 gallons of water flooded the space, impacted approximately 5,000 artifacts, and severely damaged the gallery. MoF has been a SHERN (Seattle Heritage Emergency Response Network) member since 2015. Frederick will speak to the Collections response to this disaster and to the benefits of SHERN membership. This event expedited several long term projects, including vacating a sub-par storage area, consolidating staff work areas and updating a long term exhibit. 
Nicole helped with the initial response to the water event at the Museum of Flight in February 2022 as well as with follow up recovery efforts. She helped coordinate borrowing salvage supplies from the shared SHERN resources. She has been one of the Museum’s reps to SHERN for more than 4 years.
                \emph{ (confirmed) }
              

              
                \subsubsection*{ Lia Frenchman }
                Presenter\newline
                 Tribal Historic Preservation Technician\newline
                 Quinault Indian Tribe, Taholah/WA
                \newline
                LFrenchman@quinault.org\newline
                
                (360) 276-8215 \newline

                The Quinault Indian Nation, located on the Washington Coast, is expected to be directly and negatively impacted by sea-level rise in the near future. Panelist will talk about what that means for the museum planning process and how they plan to prepare.
                \emph{ (confirmed) }
              

              
                \subsubsection*{ Laura Phillips }
                Presenter\newline
                Archaeology Collections Manager\newline
                Burke Museum of Natural History and Culture, Seattle/WA
                \newline
                lphill@uw.edu\newline
                
                (206) 685-3849 x2\newline

                This panelist is one of the founding members of the Seattle Heritage Emergency Response Network (SHERN), and will talk about how the network of over 18 institutions in Seattle area has made responses to disasters an effective community effort
                \emph{ (confirmed) }
              

              
        
          \newpage
          \section{ Strategies for Surfacing Truth and Fostering Reconciliation for Racial Equity }
            \begin{description}
              \item [ID:]
              WMA2022\_227

              \item [Assigned to:]
                \item [Track:]
              \end{description}

              Museums \&amp; cultural institutions are often quick to celebrate the progress they have made toward racial equity, while struggling to dedicate time to pause and reflect on what might prevent them from moving forward. Museums \&amp; Race offers this session to help museum practitioners foster new dialogic skills to have more truthful conversations, as well as practical ways to move from naming the issues to developing practical strategies to combat harmful behaviors.

              \subsection*{Session Information}
                \begin{description}
                  \item [Format:] Regular session/panel (roundtable, single speaker, etc.)
							    
							    \item [Uniqueness:]This session moves forward from fostering honest dialogue to identifying targeted tools and strategies, including the Museums & Race Report Card, to combat counterproductive behaviors.
							    \item [Objectives:]Participants will identify and comprehend how they might be contributing to blocking progress on racial equity in their institution.

Participants will be better equipped with ways to have more open and honest dialogue and learn to use a tool for helping their institutions better commit to ongoing equity work. 

Participants will share and discuss with their peers strategies that have been helpful or successful at moving their institution forward towards racial equity.
							    \item [Engagement:]We intend to spark dialogue in a structured way by posing an open-ended question and having session participants work in small groups to respond. We will use a dialogic technique that allows for everyone to participate and fosters a “brave space” for honest conversations. We’ll close with a tutorial of our report card tool as one option participants can use to continue working on the issues they surfaced throughout the session and work toward moving their institution forward.
							    \item [Relationship to Theme:]In our last seven years of equity work, Museums \& Race has learned that before organizations can make true progress, they need to name-and be honest about-the behaviors, policies, environments, and systems that inhibit them from moving forward. Before rushing to try new things, make new hires, or create new policies, it’s important to address what we are already doing. Without honest and authentic reflection, any work toward equity will miss the mark.
							    
                \end{description}
              \subsection*{Audience}
                \begin{description}
                  \item [Audiences:]Board Members~Curators/Scientists/Historians~Development and Membership Officers~Directors/Executive/C-Suite~Diversity and inclusion specialists~Educators~Exhibit Designers, Installers, Fabricators~Facilities Management Personnel~General Audience~HR Personnel~Marketing \& Communications (Including Social Media)~Programs~Registrars, Collections Managers~Visitor Services~Volunteer Managers~
                  \item[Professional Level:]All professional levels~
                \item[Scalability:] The dialogic techniques we will demonstrate and use have broad applications in museums regardless of the target issue - creating space and shifting institutional culture toward honest and open communication will only help museums better identify and strategize around racial equity issues and beyond. Our report card tool can be used by museums of all sizes or even by teams within larger museums.

							
              \item[Other Comments:] This session is for all levels of museum staff, board members, volunteers, and practitioners, with an eye to leadership and board members who may have responsibility for decision-making and shaping institutional culture.
              \end{description}
            \subsection*{Participants}
              \subsubsection*{ Dr. Karlisa  Callwood }
              Submitter, Moderator, Presenter\newline
              Director, Community Conservation Education \& Action; Museums \& Race Steering Committee Member\newline
              Perry Institute for Marine Science, Bahamas/Miami, FL
              \newline
              callwoodk@gmail.com\newline
              
              786-223-6694\newline

              Dr. Karlisa Callwood is a scientist and informal educator who has worked in the science museum field for over 15 years, most recently serving as the VP of Science Engagement and Outreach at the Pacific Science Center. She brings experience in developing and managing museum programs, community engagement and partnership development, managing cross-departmental efforts, and developing trainings and facilitating conversations around DEAI issues within cultural organizations. Additionally, since joining the Museums \& Race Steering Committee in 2020, she has primarily supported the roll-out of the updated Report Card tool.\newline


              

              
                \subsubsection*{ Jackie Peterson }
                Presenter\newline
                Owner \& Chief Excellence Officer; Museums \& Race Steering Committee Member\newline
                Jackie Peterson | Exhibit Services, Seattle, WA
                \newline
                jackie.a.peterson@gmail.com\newline
                hello@jackiepeterson-exhibitservices.com\newline
                347-903-1165\newline

                Jackie Peterson is currently the longest-serving member of Museums \& Race steering committee and has significant “institutional memory” about the evolution of Museums \& Race and the challenges this group has addressed over the last 4.5 years. Jackie has also worked in the museum field for 15 years as a fundraiser and as an exhibit developer and writer, for and with a wide range of museums-always through a lens of equity and inclusion. She also works with the Empathetic Museum, which has informed her practice of working with museums to develop exhibits through an empathetic framework.
                \emph{ (confirmed) }
              

              

              

              
        
          \newpage
          \section{ Keys to Successful and Collaborative Grant Writing }
            \begin{description}
              \item [ID:]
              WMA2022\_228

              \item [Assigned to:]
                \item [Track:]Business~
              \end{description}

              Writing grant proposals for Foundations and Government Agencies require collaboration between multiple staff from development, curatorial, collections, education, and our community partners. This session will combine individual grant writing instruction with exercises to guide us all toward more collaborative work. Audience members will leave this session having learned how to approach your next grant proposal by improving your own focus and writing as well as by listening more actively to your colleagues’ ideas and needs. 

              \subsection*{Session Information}
                \begin{description}
                  \item [Format:] Regular session/panel (roundtable, single speaker, etc.)
							    
							    \item [Uniqueness:]This is a practical workshop where participants walk away with examples of how to fundraise in collaboration with their colleagues.
							    \item [Objectives:]Participants will learn basic grant writing, including how to prepare for and construct a typical grant proposal. We will also discuss different techniques to use when working with colleagues and community partners to solicit a diversity of language, ideas, needs, activities, goals and objectives to include in grant proposals. Lastly, we will review common professional writing tips to help organize our thoughts and write more clearly and persuasively. This session will help both fundraisers and general Museum staff members understand that soliciting income from Foundations and Government Agencies is a process that includes many elements and questions, and is best accomplished by including multiple voices. Fundraising is most successful when we can do it together. Using a grant proposal structure and purpose, participants will learn techniques to find better focus, write more clearly, and listen more actively.
							    \item [Engagement:]Participants will actively participate in this session: They will have opportunities to 1) focus through meditation, 2) write from writing prompts, 3) actively listen to one other person with guided instruction, and 4) edit their work with specific writing tips. They will receive one handout to take with them at the end of the session.
							    \item [Relationship to Theme:]Fundraising is always about forward-thinking. This session will encourage participants to also consider how to fundraise differently, with more self-awareness, collaboration, and inclusion of multiple voices.
							    
                \end{description}
              \subsection*{Audience}
                \begin{description}
                  \item [Audiences:]Development and Membership Officers~General Audience~
                  \item[Professional Level:]All levels~
                \item[Scalability:] The outcomes are most helpful for those who hold any grant writing responsibility regardless of the size of the organization. 

							
              \end{description}
            \subsection*{Participants}
              \subsubsection*{ Ariel Weintraub }
              Submitter, Moderator, Presenter\newline
              Associate Director, Institutional Giving\newline
              Oakland Museum of California, Oakland, CA
              \newline
              aweintraub@museumca.org\newline
              
              415-312-0188\newline

              The submitter is the lead presenter. There will be no other presenters.\newline


              

              
                \subsubsection*{  Weintraub }
                Presenter\newline
                Associate Director, Institutional Giving\newline
                Oakland Museum of California, Oakland, CA
                \newline
                
                
                

                Ariel Weintraub has been the primary grant writer and grants manager at OMCA for over 15 years. She plans, directs, writes and reports on grants from foundations and government agencies, raising nearly \$3 Mil. annually from grants ranging between \$5,000 to \$2 Mil. These grants vary widely -funding education, collections, technology, programming and exhibitions. By working in collaboration with her colleagues, she’s helped the Museum receive over a dozen grants from the IMLS, multi-year grants from the Irvine, Ford, Luce and Mellon Foundations, multiple NEH and NEA grants in addition to grants from local family foundations.
                \emph{ (confirmed) }
              

              

              

              
        
          \newpage
          \section{ Digital Strategy: A Means for Museum Transformation }
            \begin{description}
              \item [ID:]
              WMA2022\_229

              \item [Assigned to:]
                \item [Track:]Technology~
              \end{description}

              Museums crafting digital strategies for the first time can find the process daunting so hearing from members of the museum community who have done this work can be a great place to start. This session will address how an institution can develop a successful digital strategy, including how to leverage technology for institutional impact, how to ensure digital efforts are serving a need, and how digital efforts can protect, enhance, and showcase content.

              \subsection*{Session Information}
                \begin{description}
                  \item [Format:] Regular session/panel (roundtable, single speaker, etc.)
							    
							    \item [Uniqueness:]This session is a conversation between two museum professionals who are developing a digital strategy and the digital strategy consultants they are working with.
							    \item [Objectives:]1.    Attendees will come away with a clear understanding of what a digital strategy is and is not. It is far more than a slate of virtual programming, rather it is a cohesive vision and plan for an entire organization that supports mission-driven work. It is also important for attendees to understand how a digital strategy complements activities in the physical museum space while not serving as a substitute for an in-person visit.
2.    Attendees will leave the session with a clear understanding of how to begin the process of developing a digital strategy by identifying internal and external goals and objectives, establishing timelines for developing and implementing a strategy, and approaching digital in a scalable way that is appropriate for their specific organization.
3.    Attendees will return to their organization empowered with a “best practices tool kit” including technology tools and tips to help organizations of all sizes, source materials for reference, and a checklist so organizations can focus on what they need to do to move forward with developing a digital strategy for their museum.
							    \item [Engagement:]The audience will engage with the four speakers during a Q\&A session following the discussion. The session might require a projector. Attendees will leave the session with handouts containing resources and tangible next steps museums can take to work towards developing a digital strategy.
							    \item [Relationship to Theme:]We are working in a time of immense digital transformation. Thriving in this digital age means becoming comfortable and proficient with the tools technology affords and leveraging digital tools and innovation to be creative and proactive with concepts and projects. By keeping digital at the fore of museum work, we can incorporate it into exhibition and project conversations early, to create compelling, integrated experiences that will serve our museums well now and into the future.
							    
                \end{description}
              \subsection*{Audience}
                \begin{description}
                  \item [Audiences:]Directors/Executive/C-Suite~Educators~Marketing \& Communications (Including Social Media)~Programs~Registrars, Collections Managers~Technology~
                  \item[Professional Level:]All professional levels~
                \item[Scalability:]  The session is relevant for organizations of all sizes since digital strategies can be scaled to suit museums of any size, regardless of focus or mission. The digital strategy consultants have experience working with a wide range of museums and will be able to speak to the myriad opportunities and challenges all types of organizations need to consider when working on digital strategy. 
  

							
              \end{description}
            \subsection*{Participants}
              \subsubsection*{ Gail Mandel }
              Submitter, Presenter\newline
              Deputy Director\newline
              Oregon Jewish Museum and Center for Holocaust Education, Portland, OR
              \newline
              gmandel@ojmche.org\newline
              
              503-226-3600 ext 104\newline

              I am currently serving on OJMCHE's task force to develop and implement the museum's new digital strategy. I will continue to oversee the museum's digital strategy once implemented.\newline


              

              
                \subsubsection*{ Alisha Babbstein }
                Presenter\newline
                Archivist\newline
                Oregon Jewish Museum and Center for Holocaust Education, Portland, OR 
                \newline
                ababbstein@ojmche.org\newline
                
                503-226-3600 ext 107\newline

                Babbstein is currently serving on OJMCHE's task force to develop and implement the museum's new digital strategy. She is also overseeing the project to bring the museum's entire collection on line.
                \emph{ (confirmed) }
              

              
                \subsubsection*{ Nik Honeysett }
                Presenter\newline
                CEO\newline
                Balboa Park Online Collaborative, San Diego, CA
                \newline
                nhoneysett@bpoc.org\newline
                
                619-819-5143 \newline

                Honeysett is CEO of Balboa Park Online Collaborative, a technology nonprofit consultancy that provides support, development, and strategy for museums and cultural institutions.
                \emph{ (confirmed) }
              

              
                \subsubsection*{ Jack Ludden }
                Presenter\newline
                Senior Strategist and Innovation Specialist\newline
                Balboa Park Online Collaborative, San Diego, CA
                \newline
                jludden@bpoc.org\newline
                
                619-819-5143 \newline

                Ludden is Senior Strategist and Innovation Specialist at Balboa Park Online Collaborative, a technology nonprofit consultancy that provides support, development, and strategy for museums and cultural institutions.
                \emph{ (confirmed) }
              

              
        
          \newpage
          \section{ Moving Exhibits and Programs Forward in Response to Climate Change }
            \begin{description}
              \item [ID:]
              WMA2022\_230

              \item [Assigned to:]
                \item [Track:]
              \end{description}

              Protecting our climate and biodiversity is not just a science problem—it’s a cultural problem with relevance for all cultural institutions. As social spaces and trusted sources of accurate information, museums are well positioned to support productive, solution-focused visitor engagement with climate change and conservation. Join us to share strategies for building climate literacy and exploring community solutions. Presenters will discuss lessons learned from exhibits and programs to move our collective response to climate change forward.

              \subsection*{Session Information}
                \begin{description}
                  \item [Format:] Regular session/panel (roundtable, single speaker, etc.)
							    
								  \item [Fee:]no
							     
							    \item [Uniqueness:]The urgency and relevance of the climate and biodiversity crisis is growing. Our stories of the past, present, and future are being reshaped by climate change.
							    \item [Objectives:]Our session will explore effective strategies for climate change and conservation communication in a variety of museum exhibits and experiences. Presenters will share examples of exhibit evaluation and research, traveling and permanent exhibitions, and public programs and events that provide the foundation for engaging visitors with climate change. We will discuss lessons learned from programs and temporary exhibitions the High Desert Museum, the development of The Natural History Museum of Utah’s new permanent exhibit, A Climate of Hope (working title), and OMSI’s traveling exhibitions, Snow: Tiny Crystals, Global Impact and Under the Arctic: Digging into Permafrost. In addition, presenters will provide resources for finding common ground with broad audiences on this often-controversial topic. Participants in the session will be able to:
 
·      List effective strategies for increasing climate and conservation communication in programs and exhibits.

·      Identify resources for solution-focused climate and conservation communication with visitors.
 
·      Reference examples of climate and conservation communication resources.
							    \item [Engagement:]The audience will be invited to share questions and ideas about climate communication in their museums. Presenters will discuss examples of resources and strategies for engaging visitors in climate and conservation solutions. After brief presentations, the audience and presenters will explore applications and ideas for a variety of museum settings using a Knowledge Café format.
							    \item [Relationship to Theme:]Yes, our session is focused on moving museums forward in responding to climate change and engaging their audiences in climate solutions. As the climate crisis advances, its impact on institutions and individuals will grow. Our collective action must grow in response to its impact.
							    
                \end{description}
              \subsection*{Audience}
                \begin{description}
                  \item [Audiences:]Curators/Scientists/Historians~Educators~Evaluators~Exhibit Designers, Installers, Fabricators~General Audience~Programs~
                  \item[Professional Level:]All professional levels~
                \item[Scalability:] Presenters will share examples of events and activities that are scalable to smaller organizations. The Knowledge Café format will generate ideas from the diversity of organizations represented in the audience.

							
              \end{description}
            \subsection*{Participants}
              \subsubsection*{ Victoria Coats }
              Submitter, Moderator, Presenter\newline
              Research, Development \& Advancement Manager\newline
              OMSI, Portland, Oregon
              \newline
              vcoats@omsi.edu\newline
              vjcoats@gmail.com\newline
              5035041315\newline

              Vicki will moderate the session and share lessons learned from OMSI’s traveling exhibitions, Snow: Tiny Crystals, Global Impact and Under the Arctic: Digging into Permafrost about climate communication in exhibits.\newline


              
                \subsubsection*{ Victoria  Coats }
                Moderator, Presenter\newline
                Research, Development \& Advancement Manager\newline
                Oregon Museum of Science \& Industry (OMSI), Portland, Oregon
                \newline
                vcoats@omsi.edu\newline
                
                5035041315\newline

                Vicki will moderate the session and share lessons learned from OMSI’s traveling exhibitions, Snow: Tiny Crystals, Global Impact and Under the Arctic: Digging into Permafrost about climate communication in exhibits.\newline
                \emph{ (confirmed) }
              

              
                \subsubsection*{ Dana Whitelaw }
                Presenter\newline
                Executive Director\newline
                The High Desert Museum, Bend, Oregon
                \newline
                dwhitelaw@highdesertmuseum.org\newline
                
                541-382-4754 ext. 326\newline

                Dana will provide art and culture perspectives and approaches to climate change and conservation content from the High Desert Museum’s blend of wildlife, history, and art. She will share the museum’s approach to integrating conservation messaging into temporary exhibitions, programs and communication plans.
                \emph{ (confirmed) }
              

              
                \subsubsection*{ Lisa Thompson }
                Presenter\newline
                Exhibit Developer\newline
                The Natural History Museum of Utah (NHMU), Salt Lake City, UT
                \newline
                thompson@nhmu.utah.edu\newline
                
                801-587-3611\newline

                Lisa will share the results of front-end evaluation and prototyping with visitors that have informed NHMU’s approach to developing a climate exhibit with a goal of inspiring hope and empowering visitors to take meaningful climate action in their communities.
                \emph{ (confirmed) }
              

              

              
        
          \newpage
          \section{ IMLS Grants: Funding Community Engagement in the Time of COVID }
            \begin{description}
              \item [ID:]
              WMA2022\_231

              \item [Assigned to:]
                \item [Track:]
              \end{description}

              Museums are facing unprecedented challenges – and opportunities – to engage with their communities in new and innovative ways. But what makes for successful community engagement and how can we adapt to ensure our strategies remain meaningful and relevant despite limitations posed by the COVID pandemic? This session will highlight the experiences of recent Institute of Museum and Library Services (IMLS) grant recipients with community-focused projects and share lessons learned for future practitioners.

              \subsection*{Session Information}
                \begin{description}
                  \item [Format:] Regular session/panel (roundtable, single speaker, etc.)
							    
								  \item [Fee:]N/A
							     
							    \item [Uniqueness:]The central role that museums play in their communities is increasingly recognized and increasingly critical. IMLS funds research and projects that help identify and serve as examples of the ways the field continues to adapt to what that role will be in the future.
							    \item [Objectives:]Attendees will:
- Increase knowledge and understanding of some of the characteristics of successful community engagement in today’s ever-changing environment.
- Be inspired and informed by recent museum outreach and engagement projects undertaken by peer institutions from within the Western Region.
- Learn about the most recent grant opportunities offered by IMLS, including those that support museums’ community engagement efforts.
							    \item [Engagement:]The panelists will use interpersonal engagement strategies to connect with session participants, such as icebreakers, real-time polls, or discussion prompts, to provoke deeper thought and a more engaging experience.
							    \item [Relationship to Theme:]This session directly supports “FORWARD” through its focus on the challenges and successes museums have faced as they’ve had to transform how they interact with their communities. Current examples of community engagement projects from peer institutions highlight what it means to be inclusive and participatory, while balancing the need to connect safely in the face of a constantly changing public health crisis.
							    
                    \item [Additional Comments: ]IMLS is open to reconfiguring this session based on WMA’s needs. 
Speakers/presenters are not yet confirmed. 
Travel and attendance is dependent upon latest Federal guidance regarding COVID-19 pandemic, Federal budget, and IMLS leadership discretion.

                \end{description}
              \subsection*{Audience}
                \begin{description}
                  \item [Audiences:]Curators/Scientists/Historians~Development and Membership Officers~Directors/Executive/C-Suite~Diversity and inclusion specialists~Evaluators~General Audience~Programs~
                  \item[Professional Level:]All professional levels~General Audience~Mid-Career~
                \item[Scalability:] IMLS funds museums and libraries of all sizes and types. This session will highlight - through example projects from diverse grantee institutions - the funding opportunities available to museums, with a particular emphasis on those that help further community engagement efforts. 

							
              \item[Other Comments:] N/A
              \end{description}
            \subsection*{Participants}
              \subsubsection*{ Sarah Glass }
              Submitter, Moderator, Presenter\newline
              Senior Program Officer\newline
              Institute of Museum and Library Services, Washington, DC
              \newline
              sglass@imls.gov\newline
              
              202-653-4668\newline

              An IMLS Senior Program Officer (or higher) in the Office of Museum Services will introduce the presenters and topics to be covered, as well as provide background for the Agency’s initiatives in the area of Community Engagement. This person will also provide the latest information on upcoming (FY23) funding opportunities relevant to these topics.\newline


              

              
                \subsubsection*{ Kate Clyde }
                Presenter\newline
                Senior Director of Exhibits and Operations\newline
                San Diego Museum of Us, San Diego, CA
                \newline
                kclyde@museumofus.org\newline
                
                619-239-2001\newline

                Kate is a longtime staff member of the Museum of Us with extensive exhibit and project management experience. She is Project Director of an FY20 IMLS Museums for America “Community Anchors” grant titled, “Creating Content Through Community Collaboration.” The project is a prime example of how museums have had to pivot in response to the COVID-19 pandemic and change the nature of how they engage with their community partners.
                \emph{ (not confirmed) }
              

              
                \subsubsection*{ Dana Whitelaw }
                Presenter\newline
                Executive Director\newline
                High Desert Museum , Bend, OR
                \newline
                dwhitelaw@highdesertmuseum.org\newline
                
                (541) 382-4754 x326\newline

                As Executive Director of the High Desert Museum, Dana has overseen the implementation of multiple successful IMLS grants in recent years. She serves as Project Director of an FY20 IMLS Museums for America “Community Anchors” grant titled, “High Desert Project: Experiences that Catalyze Connection and Conversation.” This project is another example of an instance where a museum has had to adapt to the COVID-19 pandemic, and change the nature of how they engage with their community partners.
                \emph{ (not confirmed) }
              

              

              
        
          \newpage
          \section{ Weaving a Net(work) of Care for Oceanic Collections }
            \begin{description}
              \item [ID:]
              WMA2022\_232

              \item [Assigned to:]
                \item [Track:]
              \end{description}

              This session explores <em>Weaving a Net(work) of Carefor Oceanic Collections: A Native Hawaiian and Pacific Islander Museum Institute</em>.  Participants will discuss the challenges, joys and outcomes of training a cohort of 20 early-to-mid career professionals from 13 island communities, all while navigating COVID. The project, which focused on professional development and indigenous approaches to collections, exhibitions, and conservation, began with virtual programming across 7 time zones and culminated in a 4 week in-person institute.  

              \subsection*{Session Information}
                \begin{description}
                  \item [Format:] Regular session/panel (roundtable, single speaker, etc.)
							    
								  \item [Fee:]N/A
							     
							    \item [Uniqueness:]Participants will share, from conception to implementation, the profoundly collaborative nature of this multi-faceted, indigenous-centered project, which involved over 60 individuals and institutions
							    \item [Objectives:]Outcomes include sharing experiences, recommendations and best practices for: (1) collaborative planning and grant writing; (2) creating positive team dynamics for a complex project; (3) designing and tending to an array of institutional, community, and museum collaborations; (4) developing and managing a diverse cohort of 20 early to mid-career museum professionals, artists and practitioners from 13 island communities; (5) including Western notions of “best practices” alongside discussions on indigenization and decolonization; and most importantly (6) learning about indigenous notions of care in the Pacific. A roundtable/fishbowl approach involving six people embodies the ethos of the project and will enable active discussion around these learning objectives from multiple perspectives, including one of the cohort members, while also fielding questions from the audience.
							    \item [Engagement:]This will not be a series of talking heads. After opening with an informative presentation on the project (which includes video footage), we will enter into an engaging dialogue among a group of six people who were actively involved in the institute. In keeping with the nature of the institute, there will be the sharing of indigenous songs and stories. Audience members will be encouraged to actively participate in the discussion.
							    \item [Relationship to Theme:]This session addresses the conference theme by demonstrating that the best way to move Forward is through the creation of  Net(work)s of Care that enable individuals and institutions to support one another across cultural, geographical, fiscal, and political divides. Relationships between and among individuals and institutions are critical in order to enable our field to remain inspired and engaged, so that we can serve our communities and their collections.
							    
                    \item [Additional Comments: ]More on session format: We are envisioning a roundtable or fishbowl-type approach. We will begin with an overall presentation of the \_Weaving a Net(work) of Care for Oceanic Collections\_ project, which will involve various speakers and will include video footage and photographs (15 minutes). We will then engage in a group discussion with six people who were involved in the project, from UH Manoa and East-West Center, as well as a graduate student in museum studies and a member of the cohort. Central questions will be: (1) how was this project envisioned and did it live up to everyone's expectations; (2) what were the benefits and challenges of such a complex multi-faceted project?; (3) how were Indigenous Pacific notions of care addressed and supported? and (4) what does the future hold?  At each juncture, we will open up the conversation for audience questions, ensuring an open and engaging session. 

                \end{description}
              \subsection*{Audience}
                \begin{description}
                  \item [Audiences:]Curators/Scientists/Historians~Diversity and inclusion specialists~Emerging Museum Professionals~Exhibit Designers, Installers, Fabricators~General Audience~Registrars, Collections Managers~
                  \item[Professional Level:]All professional levels~Student~
                \item[Scalability:] Every organization, regardless of size, hopefully seeks more equity and inclusion. They seek to connect their collections to the communities from which they originated, and to honor indigenous notions of care. Our 20 cohort members come from museums and institutions that are large and small and we learned from each other how to create a Net(work) of Care in support of both our respective collections and each other as individuals and institutions.  The lessons we learned and wish to share thus will apply to a multitude of organizations. 

							
              \item[Other Comments:] This session would benefit anyone interested in indigenous notions of care, from collections to exhibitions to conservation. It would especially benefit those interested in professional development programs and in diversity, equity and inclusion - both in terms of creating a more inclusive staff and supporting those already there.
              \end{description}
            \subsection*{Participants}
              \subsubsection*{ Noelle  Kahanu }
              Submitter, Moderator, Presenter\newline
              Associate Specialist, Public Humanities and Native Hawaiian Programs\newline
              University of Hawai'i at Manoa, Honolulu, Hawai'i
              \newline
              nmkahanu@hawaii.edu\newline
              mooinanea22@gmail.com\newline
              (808) 375-9125\newline

              I will be both participating and facilitating the discussion.\newline


              
                \subsubsection*{ Noelle  Kahanu }
                Moderator, Presenter\newline
                Associate Specialist, Public Humanities and Native Hawaiian Programs\newline
                University of Hawai'i at Manoa, Honolulu, Hawai'i
                \newline
                nmkahanu@hawaii.edu\newline
                mooinanea22@gmail.com\newline
                (808) 375-9125\newline

                Noelle has been a frequent presenter and moderator at WMA, with more than 20 years of experience in the museum field. She is the project manager for Weaving a Net(work) of Care and was the primary author of the grant, which was funded by the National Endowment for the Humanities. She has been involved in all facets of this project and has born witness to its profoundly collaborative nature. Hermfacilitation skills will enable her to engage the roundtable/fishbowl participants andmthe audience in explore the many facets of this innovative project.\newline
                \emph{ (confirmed) }
              

              
                \subsubsection*{ Karen Kosasa }
                Presenter\newline
                Associate Professor and director of the Museum Studies Graduate Certificate Program\newline
                University of Hawai’i at Manoa, Honolulu, Hawai'i
                \newline
                kosasa@hawaii.edu\newline
                
                (808) 349-3067\newline

                As director of the Museum Studies Graduate Certificate Program, Karen has been integral to the development and implementation of the project, particularly in coordinating with a multitude of organizations and individuals who shared their expertise and resources with our institute and 20 member cohort. As someone engaged in the museum field for decades, she can also attest to the innovation and significance of his project.
                \emph{ (confirmed) }
              

              
                \subsubsection*{ Eric Chang }
                Presenter\newline
                Coordinator, East-West Center Arts Program \newline
                East-West Center, Honolulu, HI
                \newline
                ChangE@EastWestCenter.org\newline
                
                808.944.7584\newline

                Eric Chang has been a vital member of the planning committee and served as the project manager for the EWC subaward. He can speak to the historical and contemporary importance of this partnership between UH Manoa and EWC as well as the logistical and institutional complexities. He also has been involved in numerous international and Pacific programs and can speak to this project's unique approach and outcomes.
                \emph{ (confirmed) }
              

              
                \subsubsection*{ Annie Reynolds }
                Presenter\newline
                Exhibitions and Collections Curator\newline
                East-West Center, Honolulu, Hawai'i
                \newline
                reynoldsa@eastwestcenter.org\newline
                
                (808) 944-7341\newline

                Annie Reynolds, PhD, is a key member of our planning team. As the exhibitions and collections curator of the East-West Center Arts program, she was the primary individual who worked with the cohort to facilitate the creation and installation of a gallery exhibition on Weaving a Net(work) of Care. Cohort members contributed objects from their communities to the exhibit, wrote labels and participated in all aspects of the installation. The institute culminated in the opening of the exhibition on the final weekend in July of 2022. Annie will address the process and outcomes of this important facet of the project.
                \emph{ (confirmed) }
              

              
                \subsubsection*{ Roldy Aguero  Ablao }
                Presenter\newline
                 Cohort member and artist/educator/cultural practitioner \newline
                 Native Hawaiian and Pacific Islander Museum Institute , Seattle, WA
                \newline
                hafaroldy@gmail.com\newline
                
                (206) 854-5961\newline

                Roldy is the only member of the cohort to reside in the continental U.S. He is CHamoru from the Mariana Islands, currently residing in Duwamish Territory (Seattle). As an emerging indigenous Pacific museum professional who has worked at the Wing Luke Museum, the Burke Museum of Natural History and Culture, and the Pacific Island Ethnic Art Museum in Long Beach, CA, Roldy will be able to share his experiences as a member of the cohort. Note: We have yet to approach Roldy about presenting given that we have 19 others who would probably want to be invited as well.
                \emph{ (not confirmed) }
              
        
          \newpage
          \section{ Mindsets, tools, and practices moving museum educators forward }
            \begin{description}
              \item [ID:]
              WMA2022\_233

              \item [Assigned to:]
                \item [Track:]Other~
              \end{description}

              Join our discussion with authors from the new AAM book, <em>Museum Education for Today’s Audiences</em>. Each presenter will briefly share an example of how a mindset, tool, or practice from their chapter is moving museum education forward. Topics will include empowering educators to be co-leaders/coauthors of their work experience, new approaches to facilitating family learning on-site and remotely, and tools for utilizing human-centered improvement practices to create inclusive visitor experiences centered in equity.

              \subsection*{Session Information}
                \begin{description}
                  \item [Format:] Regular session/panel (roundtable, single speaker, etc.)
							    
								  \item [Fee:]n/a
							     
							    \item [Uniqueness:]This session features contributors to the new AAM book specifically addressing how the field of museum education is moving forward, focused on the needs of visitors.
							    \item [Objectives:]By the end of the session participants will:

- gain knowledge and experiences with mindsets, tools, and practices that empower museum educators, operationalize equity, and engage audiences (with a special focus on families) to build meaningful relationships with communities
- be able to compare and contrast examples of how museum education is moving forward and responding to the changing needs and expectations of visitors with the goal of a more inclusive, welcoming, and responsive educational experience 
- apply ideas from session to their own experience and institution to create personalized goals and action plans
							    \item [Engagement:]The presenters will engage the audience through a series of initial guiding questions and peer-to-peer discussions and then, after presenting their case studies and research, to exchange in groups with members of the audience. The format will allow audience members to talk collaboratively about their plans for applying the presenters’ ideas and strategies to their own work.
							    \item [Relationship to Theme:]This session features several authors from the recent AAM book specifically addressing how the field of museum education is moving forward to address the needs of our changing audiences; the focus is specifically on highlighting the relevant context of the dual pandemics of Covid and racism. The session will focus on how a time of unparalleled change in the field, museum educators are uniquely positioned to address the needs and expectations of today’s visitors and lead our intuitions into the future.
							    
                    \item [Additional Comments: ]We are always open to ideas!

                \end{description}
              \subsection*{Audience}
                \begin{description}
                  \item [Audiences:]Diversity and inclusion specialists~Educators~Emerging Museum Professionals~Programs~
                  \item[Professional Level:]All professional levels~
                \item[Scalability:] This session is not specific to any one discipline and ideas can be implemented by museum educators in institutions of all sizes. 

							
              \item[Other Comments:] This session is designed to address the needs of museum educators. However, content addressing staff empowerment, training, and improvement practices will also be relevant to all managers and museum leaders.
              \end{description}
            \subsection*{Participants}
              \subsubsection*{ Mary Kay Cunningham }
              Submitter\newline
              Co-Editor, Museum Education For Today's Audiences\newline
              Dialogue, Portland, OR
              \newline
              marykay@visitordialogue.com\newline
              mkpdx10@gmail.com\newline
              503-975-4020\newline

              


              
                \subsubsection*{ Jason Porter }
                Moderator, Presenter\newline
                Kayla Skinner Deputy Director for Education and Public Engagement\newline
                Seattle Art Museum, Seattle, WA
                \newline
                jasonp@seattleartmuseum.org\newline
                
                310-367-5581\newline

                Jason is the co-editor of the book, Museum Education for Today’s Audiences, from which the session’s presenters are drawn. The book’s focus is on providing strategies, tools, and resources to further the work of museum educators during an age in which responsiveness, relevance, and connection to audiences must be front and center. Part of editing the book was looking across the field and convening a group of thought leaders and change-makers and engaging them in contributing to the book, and Jason played a large role in finding and supporting the 25 authors to contribute to the project.\newline
                \emph{ (confirmed) }
              

              
                \subsubsection*{ Lorie Millward }
                Presenter\newline
                VP of Possibilities\newline
                Thanksgiving Point, Lehi, UT
                \newline
                lmillward@Thanksgivingpoint.org\newline
                
                801.768.4979\newline

                Lorie is an agent of change and has championed free-choice learning and mold-breaking throughout her 30+ years in the field. She has authored several articles and book chapters related to her work and was the inaugural recipient of the STEM Innovator of the Year award from the Utah Governor’s Office of Economic Development. She has served the field as President of the Utah Museums Association, WMA’s Vice President of Programs and Innovation, and as a member of the AAM’s Diversity Committee. Her range of experience and expertise includes museum design and construction, creation of immersive learning environments, exhibition development, formal and informal education, audience research, evaluation, and the natural sciences.
                \emph{ (confirmed) }
              

              
                \subsubsection*{ Scott Pattison }
                Presenter\newline
                Research Scientist\newline
                TERC, Portland, OR
                \newline
                scott_pattison@terc.edu\newline
                
                541-520-7140\newline

                Scott Pattison, PhD co-authored a chapter in the book related to family learning  and is a social scientist who has been studying and supporting STEM education and learning since 2003, as an educator, program and exhibit developer, evaluator, and researcher. His current work focuses on engagement, learning, and interest and identity development in free-choice and out-of-school environments, including museums, community-based organizations, and everyday settings. Dr. Pattison specializes in using qualitative and quantitative methods to investigate the processes and mechanisms of learning in naturalistic settings. He has partnered with numerous organizations across the country to support learning for diverse communities.
                \emph{ (confirmed) }
              

              
                \subsubsection*{ Julile Smith }
                Presenter\newline
                Founder\newline
                Community Design Partners, Portland, OR
                \newline
                julie@communitydesignpartners.com\newline
                
                503-201-7177\newline

                For nearly 30 years Julie has been serving as a connector of educators, government leaders, and community members, with the goal of shifting mindsets around improvement that are grounded in empathy, racial equity \&  focused on systemic change. Julie works with organizations to build capacity for change and believes that organizations can realize their goals when they build them with those they aim to serve. Julie has worked as a facilitator, advisor, and coach to a diverse range of projects including helping organizations implement anti-racist frameworks, community-based school solutions, educator effectiveness systems, and designing state educator networks around continuous improvement.
                \emph{ (confirmed) }
              

              
        
          \newpage
          \section{ Artists-in-Residence: Living Creativity in the Museum }
            \begin{description}
              \item [ID:]
              WMA2022\_234

              \item [Assigned to:]
                \item [Track:]Other~
              \end{description}

              This session explores the breadth and depth of artist-in-residence programs at museums. Artists bring a wealth of knowledge, creativity and unique perspectives to the museum experience - whether they are invited to explore the collection, benefit from a unique space, or share their creations with visitors. They can activate spaces, reveal new truths and engage fresh audiences. In this panel, a brief overview of artist-in-residency programs will be provided, and then a panel of professionals and museums will share their stories, experience and advice. Key takeaways include the benefits of inviting artists into the museum, learning about the variety of models out there, and tips on implementing an artist-in-residency program at your institution.

              \subsection*{Session Information}
                \begin{description}
                  \item [Format:] Regular session/panel (roundtable, single speaker, etc.)
							    
							    \item [Uniqueness:]Artist-in-residency programs at museums offer a unique chance to use creativity and art in many different discipline-type museums. This session is unique because it focuses on art, but benefits professionals and institutions from all kinds of backgrounds including culturally-specific museums, zoos and gardens, science museums, historical houses, archival collections and more. It also has the potential to be fun and creative! There is also this cool interdisciplinary aspect to artist in residency programs where art can be used to explore science concepts or history or social movements...
							    \item [Objectives:]- Attendees learn about innovative museum programs that benefit artists, museums and museum visitors
- Attendees have the chance to think about ways that artists can help re-think how museums work, how we use our collections and spaces, and how we engage visitors
- Attendees receive tips on how to implement an artist-in-residency program at their institution
- Attendees enjoy hearing stories about specific artists and their projects
							    \item [Engagement:]This is still being determined, but there will definitely be visuals with examples of artist projects and some resources for attendees to take away. There will be brief presentations, then a moderated conversation with the panelists, followed by a Q\&A with the audience. It could be fun to have an artist come to do an interactive activity (maybe a printmaker such as from Cargo or the Independent Publishing Resource Center?). They could even be doing their thing quietly in the background and print some kind of card attendees could take away.
							    \item [Relationship to Theme:]Artists are creative thinkers that directly contribute to innovation in society, as well as helping us reflect on our past and current experience and imagine our future. Artist-in-residency programs bring this forward thinking and doing into the museum.
							    
                    \item [Additional Comments: ]This is really preliminary, but I'm excited about the possibilities. I have some pretty solid ideas, but would also be open to suggestions for other presenters. Jason encouraged us to submit even if preliminary, but I do have a track record of presenting at OMA and feel pretty confident in finding some great presenters, as well as talking about working with artists.

                \end{description}
              \subsection*{Audience}
                \begin{description}
                  \item [Audiences:]Curators/Scientists/Historians~Diversity and inclusion specialists~Educators~Emerging Museum Professionals~Exhibit Designers, Installers, Fabricators~General Audience~Programs~Registrars, Collections Managers~
                  \item[Professional Level:]All professional levels~
                \item[Scalability:] I hope to have panelists that provide a variety of experiences in different types of institutions and different sizes of institutions. This will be considered in the development of questions asked by the moderator and in resources to be provided.

							
              \item[Other Comments:] I'm hoping this will be both an entertaining session and a learning session so that it can benefit a variety of people -- from those just interested in how artists engage with museums to those interested in developing a program themselves.
              \end{description}
            \subsection*{Participants}
              \subsubsection*{ Eleanor Sandys }
              Submitter, Moderator\newline
              Public Art \& Artist Programs Coordinator\newline
              Oregon Arts Commission, Milwaukie, OR
              \newline
              eleanor.sandys@biz.oregon.gov\newline
              eleanorsandys@gmail.com\newline
              503-351-8550\newline

              I will provide an overview of what an artist-in-residence program is and present a variety of different examples. I will facilitate a moderated conversation among the panelists. I can answer questions related to application processes and speak to compensating artists and the variety of artists and project types out there.\newline


              
                \subsubsection*{ Eleanor Sandys }
                Moderator\newline
                Public Art \& Artist Programs Coordinator\newline
                Oregon Arts Commission, Milwaukie, OR
                \newline
                eleanor.sandys@biz.oregon.gov\newline
                eleanorsandys@gmail.com\newline
                503-351-8550\newline

                I work for the State of Oregon for the state Arts Commission running artist grant programs and our public art program. We are a convener and support statewide efforts to advance the arts. I have experience as a panel and committee facilitator, including several museum conference sessions. I also sit on the Oregon Museums Association board.\newline
                \emph{ (confirmed) }
              

              
                \subsubsection*{ TBD TBD }
                Presenter\newline
                TBD\newline
                TBD, TBD
                \newline
                
                
                

                I have ideas for presenters, some of whom I know, but have not yet reached out to them. Jason encouraged submitting proposals, even if not fully fleshed out. I don't feel right putting in the peoples' names as presenters without reaching out to them first, so just I am providing you a list of those I have thought of. Potentially we could do pecha kucha with more presenters? So much good stuff here.
- Residency program at Stevens-Crawford Heritage House in Oregon City; run by Art in Oregon; I know the two people who run this organization and would ask Tammy Jo Wilson (also an artist) to come speak. https://artinoregon.org/2021/03/18/stevens-crawford-house-artist-residency/ 
- Residency program run at the Portland City Archives: I know people at the Regional Arts \& Culture Council and also could ask the artist featured in this article, Sabina Haque. She has spoken on other panels (experience with public speaking) and we have corresponded. https://www.portlandoregon.gov/archives/article/575419 
- Tacoma Museum of Glass: https://www.museumofglass.org/visiting-artist-lineup2021; I know Katie Buckingham, Curator through collaboration with WaMA, but haven't asked her about this.
- Burke Museum: https://www.burkemuseum.org/northwest-native-art-gallery/artist-studio.
- The Sitka Center for Art \& Ecology partners with the Oregon State University Hatfield Marine Science Center for a residency on an ocean vessel. I have corresponded with the ED of Sitka, Alison Dennis and could ask her who might be a good rep. https://www.sitkacenter.org/residencies/artist-at-sea-residency 
- I'm a huge fan of the residency programs at the Exploratorium in SF, but don't know who to contact there.
                \emph{ (not confirmed) }
              

              
                \subsubsection*{ See comments for presenter 1 See comments for presenter 1 }
                Presenter\newline
                \newline
                , Portland, OR
                \newline
                See comments for presenter 1\newline
                
                

                See comments for Presenter 1
                \emph{ (not confirmed) }
              

              
                \subsubsection*{  See comments for presenter 1 }
                Presenter\newline
                See comments for presenter 1\newline
                , Tacoma, WA
                \newline
                See comments for presenter 1\newline
                
                

                See comments for presenter 1
                \emph{ (not confirmed) }
              

              
        
          \newpage
          \section{ Show Me the Money: Integrating Pay into Museum Best Practices  }
            \begin{description}
              \item [ID:]
              WMA2022\_235

              \item [Assigned to:]
                \item [Track:]Other~
              \end{description}

              Paid internships, salary transparency, and livable pay rates have the opportunity to become the rule, rather than the exception at museums. This session will explore these crucial topics and how they are impacting the next generation of museum professionals. Hear from a group of Emerging Museum Professionals with lived experiences, gather information and resources from the National Emerging Museum Professionals Network, and learn how to become an advocate for these best practices at your organization.

              \subsection*{Session Information}
                \begin{description}
                  \item [Format:] Regular session/panel (roundtable, single speaker, etc.)
							    
							    \item [Uniqueness:]This session is one of three parts dedicated to Emerging Museum Professionals and the increasingly important and prevalent topics listed in the session description.
							    \item [Objectives:]1. Participants will be provided with an opportunity to connect and discuss with fellow emerging museum professionals and others, about some of the obstacles facing them in the field, especially surrounding the topic of pay with a focus on unpaid vs. paid internships. 
2. Participants will be provided with information, resources, and actionable ways to combat some of these issues such as advocating for paid internships, finding ways to educate institutions on the importance of wage and salary transparency, and where to find support.
3. Participants will help brainstorm additional ways to support emerging museum professionals and how they can continue to move this conversation and action forward at their own institutions and in the museum field at large.
							    \item [Engagement:]Participants will hear from presenters about the topic of pay in museums, with a significant portion focused on internships. Presenters will provide first-hand experiences on the topic as well as provide information about what is currently happening in the field to combat issues such as unpaid internships. Participants will be engaged through questions about professional experiences, breakout discussions about the topic, and a larger group discussion.
							    \item [Relationship to Theme:]The concept and theme of FOWARD fits this topic in many ways as museums continue to adapt and move forward into the future. Myself and the other presenters/facilitators look forward to a time when all internships are paid and reasonably so, as well as salaries for all positions regardless of museum size or location. This session aims inspire new and thought provoking conversations around important issues in best practices, fair pay, and sustainability in museums.
							    
                    \item [Additional Comments: ]We may possibly have an additional presenter from the Western chapter of the NEMPN, but if they are unable to join, they will still provide information, resources, and other content for our session. 

                \end{description}
              \subsection*{Audience}
                \begin{description}
                  \item [Audiences:]Emerging Museum Professionals~
                  \item[Professional Level:]All levels~Emerging Professional~Student~
                \item[Scalability:] We plan to discuss how museums of all sizes and locations should and can move toward providing paid internships and that topic of pay in general in museums is and will continue to be crucial to new generations of museum professionals. EMPs enter institutions of various types, sizes, and locations, but many times face the same issues making them relevant across these different organizations.

							
              \item[Other Comments:] While the main audience we wish to engage is emerging museum professionals, we also hope for a variety of people involved in museums at various levels. This includes students and professors who may be directly involved in things like museum internships and new careers, as well as directors and board members who may have more decision making power when it comes to pay at museums.
              \end{description}
            \subsection*{Participants}
              \subsubsection*{ Ariel  Peasley }
              Submitter, Presenter\newline
              Education and Community Engagement Coordinator\newline
              Coos History Museum and Oregon Museums Association, Coos Bay, Oregon
              \newline
              education@cooshistory.org\newline
              
              503-951-3981\newline

              As a presenteralong with the others in my group/panel, I can offer the perspective of an emerging museum professional. I have first hand knowledge and experience in terms of internship requirements and pay, or lack their of, in many museums, and education and experience requirements compared to salaries. I will also be able to share some resources and information from the NEMPN if a representative is not able to join us.\newline


              

              
                \subsubsection*{ Peter Kukla }
                Presenter\newline
                Planetarium Manager\newline
                Eugene Science Center and Oregon Museums Association , Eugene, Oregon
                \newline
                peterj.kukla@gmail.com\newline
                
                

                Peter is also an emerging museum professional who will be hosting a Happy Hour for EMPs as well. Peter is also an OMA board member, and works with WMA. More information about Peter, his background, and contribution can/will be provided by Peter.
                \emph{ (confirmed) }
              

              
                \subsubsection*{ Tamara  Maxey }
                Presenter\newline
                Collections Specialist\newline
                Gold Nugget Museum , Paradise, California
                \newline
                maxey.tg@gmail.com\newline
                
                

                Tamara is also an emerging museum professional and represents a museum in California. As an EMP she has helped restart a museum decimated by wildland fires. She also has experience in museum internships, pay scales including contract work, and a variety of experiences in museums.  More information about Tamara, her background, and contribution can/will be provided by Tamara.
                \emph{ (confirmed) }
              

              

              
        
          \newpage
          \section{ “Resurrecting a Diverse Ghost Town” }
            \begin{description}
              \item [ID:]
              WMA2022\_237

              \item [Assigned to:]
                \item [Track:]
              \end{description}

              Maxville is a former timber town in remote northeast Oregon, which in its heyday, from 1924 to 1933, was the largest city in Wallowa County. Maxville was a timber town like so many others in the Pacific Northwest, except Maxville was home to African American and white loggers who worked side-by-side felling timber and lived in a segregated community with their families during a period when black exclusionary and sundown laws existed in Oregon. Learn how Maxville Heritage Interpretive Center is bringing the ghost town of Maxville back to life to tell its fascinating history by purchasing 240 acres of the forestland that contains the original town site.

              \subsection*{Session Information}
                \begin{description}
                  \item [Format:] Regular session/panel (roundtable, single speaker, etc.)
							    
								  \item [Fee:]n/a
							     
							    \item [Uniqueness:]Through our discussions about bringing Maxville back to life, the presenters intend for the audience to better understand the state’s racist origins and how using the tool of land conservation can help tell that story.
							    \item [Objectives:]Mr. Paulus, the attorney providing pro bono representation for MHIC with the property transaction, will discuss the legal aspects of the property purchase, including the purchase agreement preparation and acquisition due diligence with title, appraisal, land use, and environmental compliance.
							    \item [Engagement:]Ms. Trice, Executive Director of Maxville Heritage Interpretive Center, will first give a presentation about the history of Maxville and her family connection to the site when she learned her dad had been one of the African American loggers at Maxville. She also will discuss the goals of MHIC in purchasing the town site and how it intends to restore a historic log building on the property, the last remaining building from Maxville that is currently deconstructed and in storage. 
Mr. Paulus, , will discuss the legal aspects of the property purchase
							    \item [Relationship to Theme:]We are literally creating learning and healing we are working to address Black trauma  our environment is grounded in a Brave Space for all.
							    
                    \item [Additional Comments: ]no

                \end{description}
              \subsection*{Audience}
                \begin{description}
                  \item [Audiences:]Curators/Scientists/Historians~Diversity and inclusion specialists~Educators~Emerging Museum Professionals~Exhibit Designers, Installers, Fabricators~General Audience~HR Personnel~Library Staff~Visitor Services~
                  \item[Professional Level:]All professional levels~Student~
                \item[Scalability:] There will be a joint 30-minute question and answer session with the audience to address individual questions regarding scalability of their own projects.

							
              \end{description}
            \subsection*{Participants}
              \subsubsection*{ Gwendolyn Trice }
              Submitter, Moderator, Presenter\newline
              Executive Director\newline
              Maxville Heritage Interpretive Center, Enterprise
              \newline
              gwen@maxvilleheritage.org\newline
              trice.g@gmail.com\newline
              5414263545\newline

              First hand experience as a 3rd generation rural Oregon, African American descendant of a logging family that has founded and grown this non-profit to fruition. I will share with Oregonians about this hidden history and the legacy we build today.\newline


              

              
                \subsubsection*{ Gwendolyn Trice }
                Presenter\newline
                Executive Director\newline
                Maxville Heritage Interpretive Center, Enterprise
                \newline
                gwen@maxvilleheritage.org\newline
                gwen@maxvilleheritage.org\newline
                5414263545\newline

                Subject Matter Expert and Founder of the organization creating space for others.
                \emph{ (confirmed) }
              

              
                \subsubsection*{ Fritz Paulus }
                Presenter\newline
                 Attorney\newline
                Maxville Heritage Interpretive Center, Portland, OR
                \newline
                fritz@wfpauluslaw.com\newline
                
                5414263545\newline

                Directly related to the work, Subject Matter Expert
                \emph{ (confirmed) }
              

              

              
        
          \newpage
          \section{ Trophies and the Legacy of War in Museum Collections }
            \begin{description}
              \item [ID:]
              WMA2022\_238

              \item [Assigned to:]
                \item [Track:]Collections~
              \end{description}

              <strong>Many museum workers have encountered war trophies or looted objects in their collections, but there is no field-wide consensus on how to address them. War booty can take many forms. Does all booty occupy the same ethical territory, or are there “good” and “bad” kinds? Is my institution the right place to interpret these items? These are questions we will address from multiple angles, with a panel of museum professionals from a variety of backgrounds.</strong>

              \subsection*{Session Information}
                \begin{description}
                  \item [Format:] Regular session/panel (roundtable, single speaker, etc.)
							    
								  \item [Fee:]N/A
							     
							    \item [Uniqueness:]The question of how museums should deal with war booty is little discussed despite how pervasive and ethically fraught it is in museum collections.
							    \item [Objectives:]Rather than hard answers, attendees will leave with a set of questions they can ask when they encounter war booty in their own collections. This session will introduce some common types of war trophies and suggest several lenses through which to view them, including the violence that led to their acquisition and their meta-history as museum objects.  It will problematize the (lack of) interpretation much war booty has received in the past and discuss how behind-the-scenes institutional practices can shape the public perception of war in general and specific wars in particular. Perhaps most of all, it will encourage curators to look outside their museums for answers and build relationships with community stakeholders.
							    \item [Engagement:]Audience members will be encouraged to ask questions and bring their own experiences for discussion.
							    \item [Relationship to Theme:]As many museums struggle to move past their colonial roots, war booty often gets swept aside or placed in storage.  However, to truly move forward and gain community trust, it is necessary to honestly and transparently evaluate these items. This is part of the transition from a past in which museums celebrated imperialism, into a future where they work to dismantle its legacy.
							    
                    \item [Additional Comments: ]**I would love some help finding presenters, as a relative newcomer to this field! I was hoping to include presenters from multiple different types of museums—art, military, and Tribal/culturally specific, at least. Any suggestions on who to ask would be welcome. I have potential interest but no confirmation from anyone.**

                \end{description}
              \subsection*{Audience}
                \begin{description}
                  \item [Audiences:]Curators/Scientists/Historians~General Audience~Registrars, Collections Managers~
                  \item[Professional Level:]All professional levels~
                \item[Scalability:] Workers in different kinds of museums are likely to encounter different types and amounts of war booty. Small local museums may have individual soldiers’ keepsakes; art museums may have priceless art looted by Nazis; and military museums may have significant holdings of looted objects from many wars. Museums of anthropology may have war booty that they have been treating as ethnographic material. Umbrella museums may have all of the above. All are likely to face challenges in common with and distinct from other types of museums.

							
              \end{description}
            \subsection*{Participants}
              \subsubsection*{ Silvie Andrews }
              Submitter, Moderator\newline
              Museum Director\newline
              Gresham Historical Society, Gresham, OR
              \newline
              andrewssm2@gmail.com\newline
              
              (503)867-5966\newline

              I have recently completed my master’s thesis on a particular collection of war booty in the Oregon Historical Society museum, and as a part of my studies I have done a significant amount of research and thinking about war booty in museums. This would make me a better moderator than presenter as the organization I now work for has little, if any, war booty and my knowledge base does not largely come from personal experience.\newline


              
                \subsubsection*{ Silvie Andrews }
                Moderator, Presenter\newline
                Museum Director\newline
                Gresham Historical Society, Gresham, OR
                \newline
                andrewssm2@gmail.com\newline
                
                (503)867-5966\newline

                See above\newline
                \emph{ (confirmed) }
              

              
                \subsubsection*{  none confirmed }
                Presenter\newline
                n/a\newline
                n/a, n/a
                \newline
                
                
                

                n/a
                \emph{ (not confirmed) }
              

              

              

              
        
          \newpage
          \section{ Beyond Land Acknowledgements: Real Collaboration with Tribes and Tribal Museums }
            \begin{description}
              \item [ID:]
              WMA2022\_239

              \item [Assigned to:]
                \item [Track:]Indigenous~
              \end{description}

              Tribal land acknowledgements are rapidly growing in popularity among institutions and organizations, taking the form of opening statements in meetings and conferences, signage, or website messages. One might ask why land acknowledgements are being made in a growing number of settings, including the museum. Acknowledgement is a simple, powerful way of showing respect and is intended as a step toward correcting the practices that erase or freeze Indigenous people’s history and culture while inviting and honoring the truth. However, the land acknowledgement is also at risk of ending where it began, perhaps well-conceived and received, but merely a symbolic gesture with little to no follow through of engagement and real change. While land acknowledgements are well meaning, they are no substitute for substantive and ongoing Tribal relationships and understandings of tribal land claims.

              \subsection*{Session Information}
                \begin{description}
                  \item [Format:] Regular session/panel (roundtable, single speaker, etc.)
							    
							    \item [Uniqueness:]Land acknowledgments seek to recognize and respect tribes as traditional stewards of the land, yet they can also introduce complications for many native individuals and communities.
							    \item [Objectives:]Quite often these days, people open a meeting or read a web page that includes a variation of an indigenous land acknowledgement. This may include asking local tribal representatives to “perform” a land acknowledgement by way of an opening prayer or song in ceremonial regalia, which risks continuing a negative stereotype prevalent today, with the added potential to reproduce negative legacies of the museum trope. Rather than use a standard template with tribal names inserted, land acknowledgements must be unique and may require recognizing broader uses of the land by multiple tribes over time. Simply put, these lands were once occupied by indigenous people and many indigenous people still live on and use this land today. Museums and organizations can acknowledge the original inhabitants of the land but must also seek an understanding of current tribal land claims. Many are confused on whose tribal (whether aboriginal, ceded, or treaty reserved) lands they are actually upon. What, if any, issues arise with state and federal museums acknowledging they exist on tribal land? Panelists will engage with these issues and evaluate the phenomenon of land acknowledgements. Attendees will be given tools for learning whose land they occupy, ideas for thinking through the acknowledgment, procedures for writing  them and consulting about them. The ultimate goal is to move beyond the well-constructed land acknowledgment into the development of fruitful partnerships with tribes.
							    \item [Engagement:]Session attendees will hear tribal perspectives of land acknowledgements from both tribal and non-tribal museum professionals, with discussion varying between what honors tribes and what might fail them in this process, and with heartfelt intentions, work together towards concepts of moving this current trend forward. Takeaways will be distributed to audience members, including:  1) distributing written samples of a variety of current land acknowledgements, 2) sharing Guidelines for Collaboration recently developed by the Society for Anthropological Research. Attendees will learn how to find out which tribes are to be spoken for in a land acknowledgement and how to reach out to the appropriate tribal contacts through resources such as the inventory of tribal museums, tribal historic preservation officers and tribal language programs. In addition, ample time for questions and discussion between panelists and attendees will be built into the panel session and directed by the moderator.
							    \item [Relationship to Theme:]The current trend towards tribal land acknowledgements has the potential to perpetuate museums and institutions with long lasting and positive effect. Working through the concept and getting it right now in ways that are forward thinking will have larger implications for museum settings of the future.
							    
                \end{description}
              \subsection*{Audience}
                \begin{description}
                  \item [Audiences:]General Audience~
                  \item[Professional Level:]All professional levels~
                \item[Scalability:] Land Acknowledgements are for all sized institutions 

							
              \end{description}
            \subsection*{Participants}
              \subsubsection*{ Bobbie Conner }
              Submitter, Presenter\newline
              Executive CDirector\newline
              Tamastslikt Cultural Institute, Pendleton, OR
              \newline
              Bobbie.Conner@tamastslikt.org\newline
              
              541-429-7709\newline

              NA\newline


              
                \subsubsection*{ Elizabeth  Woody  }
                Moderator\newline
                Executive Director\newline
                The Museum at Warm Springs, Warm Springs, OR
                \newline
                NA\newline
                
                NA\newline

                NA\newline
                \emph{ (confirmed) }
              

              
                \subsubsection*{ Sven  Haakanson   }
                Presenter\newline
                Curator of North American Anthropology\newline
                Burke Museum, Seattle, WA
                \newline
                NA\newline
                
                

                NA
                \emph{ (confirmed) }
              

              
                \subsubsection*{ Rebecca  Dobkins  }
                Presenter\newline
                Faculty Curator\newline
                Hallie Ford Museum of Art, Salem, OR
                \newline
                NA\newline
                
                

                NA
                \emph{ (confirmed) }
              

              

              
        
          \newpage
          \section{ Origin Stories: Immersive Environments for Origin Stories of First Nations }
            \begin{description}
              \item [ID:]
              WMA2022\_240

              \item [Assigned to:]
                \item [Track:]
              \end{description}

              **Reveal the process of engaging with 4 First Nations, including the Seneca, Yup’ik, O’odham and Navajo in developing a new permanent exhibit at the Heard Museum to convey the unique origin story of each. **
** **
<strong>- Engage audiences through digital media</strong>
<strong>- Empower the voice of first nations through storytelling</strong>
<strong>- Engagement and collaboration amongst 4 nations to both unite around common themes and celebrate differences</strong>

              \subsection*{Session Information}
                \begin{description}
                  \item [Format:] Regular session/panel (roundtable, single speaker, etc.)
							    
							    \item [Uniqueness:]This session will celebrate difference and diversity through presenting 4 unique origin stories and utilizes technology through a digitally immersive environment to engage audiences.
							    \item [Objectives:]1.    Identify a collaborative process in assembling an exhibit with multiple, unique perspectives.
2.    Empowering the voice of First Nations to share their origin story.

Through immersive media, expose broader audiences to a variety of origin stories, with an acceptance that MULTIPLE stories can co-exist. Differences are celebrated and celebrates a diversity of thought.
							    \item [Engagement:]Questions / Answers – we would like to open an informal dialogue with the audience and have them share and discuss their experiences as part of the engagement process.
							    \item [Relationship to Theme:]Relates to the conference theme, FORWARD by highlighting the voices and stories of First Nations
							    
                \end{description}
              \subsection*{Audience}
                \begin{description}
                  \item [Audiences:]General Audience~
                  \item[Professional Level:]General Audience~
                \item[Scalability:] NA	

							
              \end{description}
            \subsection*{Participants}
              \subsubsection*{ Dan  Clevenger }
              Submitter, Moderator, Presenter\newline
              Principal\newline
              DLR Group, Phoenix. AZ
              \newline
              dclevenger@dlrgroup.com\newline
              
              6025076834\newline

              Dan is an architect with DLR Group and worked with the Heard Museum and Exhibit Design team to implement this vision.\newline


              
                \subsubsection*{ Dan  Clevenger }
                Moderator, Presenter\newline
                Principal\newline
                DLR Group, Phoenix, AZ
                \newline
                dclevenger@dlrgroup.com\newline
                
                6025076834\newline

                Dan is an architect with DLR Group and worked with the Heard Museum and Exhibit Design team to implement this vision.\newline
                \emph{ (confirmed) }
              

              
                \subsubsection*{ Sean  Mooney }
                Presenter\newline
                Exhibit Designer\newline
                The Rock Foundation – Chief Curator and Exhibit Designer, New York, New York
                \newline
                smooney@therockfoundation.org\newline
                
                

                Sean is an exhibit designer who has worked hand-in-hand with the first nations identified for this exhibit
                \emph{ (confirmed) }
              

              
                \subsubsection*{ Chuna  McIntyre }
                Presenter\newline
                Director; Conservation Advisor \newline
                Nunamta Yup’ik Eskimo Singers and Dancers; Sharing Our Knowledge Project , Eek, Alaska
                \newline
                NA\newline
                
                

                Chuna has advised as a representative of the Yup’ik
                \emph{ (confirmed) }
              

              
                \subsubsection*{ John  Bulla }
                Presenter\newline
                Chief Operating Officer and Deputy Director\newline
                Heard Museum, Phoenix, AZ
                \newline
                jbulla@heard.org\newline
                
                

                John has led the vision and implementation of the new exhibit, with a goal of providing visitors with a unique story which further celebrates the unique world-view of each of the first-nations, celebrating their differences.
                \emph{ (confirmed) }
              

              
        
          \newpage
          \section{ Decolonizing Initiatives in Action at the Burke Museum and The Museum of Us }
            \begin{description}
              \item [ID:]
              WMA2022\_243

              \item [Assigned to:]
                \item [Track:]Other~
              \end{description}

              Discussions about decolonization/decolonizing initiatives often rely on abstract, academic language that fails to connect with real world practices. 
In this session, Burke Museum and Museum of Us staff members will model a decolonized conversation, transparently sharing their experiences in a way that can be applied to many museums.
Join us for a robust conversation about how our values impact the way we work, and how we navigate change in ways that build community rather than foster conflict.

              \subsection*{Session Information}
                \begin{description}
                  \item [Format:] Regular session/panel (roundtable, single speaker, etc.)
							    
							    \item [Uniqueness:]This session takes the abstract concept of decolonization/decolonizing initiatives and illustrates how it can be tangibly applied to museum operations and initiatives.
							    \item [Objectives:]Decolonization/decolonizing initiatives is often a vague and intimidating term, so much so that many museums who wish to embark on this journey feel stuck. Instead of a top-down presentation format, the Burke Museum and the Museum of Us will model a decolonized conversation, sharing what has worked for our museums in hopes to inspire a different approach to decolonization/decolonizing initiatives and DEAI work.  

The Burke’s Tribal Liaison and the Museum of Us CEO will interview decolonization/decolonizing initiatives specialists at the partner’s museum, leading session attendees in a discussion of the practical applications of their decolonization/decolonizing initiatives work. Session attendees will learn:

1. What decolonization/decolonizing initiatives looks like in practice at the Washington State Museum of Natural History and Culture and the Museum of Us, and
2. The interrelatedness of key concepts like consultation, representation, the meaning of ownership, the impact of colonization, the purpose of collections, and an institution’s responsibility to community needs and knowledge.

Participants will leave inspired to start or continue their own decolonization/decolonizing initiatives journey, with the tools needed to begin identifying areas in their own institutions where concrete, action-oriented conversations can take place, and with some of the language that will help those conversations take shape. They will also learn the importance of preserving and sharing data about this work, in order to use that collective wisdom to inform creative, innovative, collaborative steps towards a more equitable, inclusive museum.
							    \item [Engagement:]Audience members will see some of the practical areas that the Burke Museum and the Museum of Us have found success in, and in small groups, will discuss creative ways they can either apply some of the strategies to their museums, or ways in which they can creatively decolonize too. There will also be an opportunity to ask questions of the presenters at the end.
							    \item [Relationship to Theme:]The process of decolonization/decolonizing initiatives is not simply one of reckoning with the past, but in transmuting the past into a better future. Both the Burke Museum and the Museum of Us are actively looking for new, creative ways to improve the museums and move into a new future of museums.
							    
                    \item [Additional Comments: ]This session will be a dual-interview format, with one person from the Burke Museum interviewing someone from the Museum of Us, and vice versa for a total of four presenters. There will also be an opportunity for members of the audience to engage in similar ways with other audience members, and ask questions of the presenters. 

The concept here is to model what a decolonized transfer of information looks like, where the information is applicable and helpful to the participants as well. 

                \end{description}
              \subsection*{Audience}
                \begin{description}
                  \item [Audiences:]Other~
                  \item[Professional Level:]All professional levels~
                \item[Scalability:] The Burke Museum is a larger museum associated with a university, while the Museum of Us is a smaller, independent museum. They both focus on creative approaches to decolonization/decolonizing initiatives that can be adapted, modified, or learned from to suit any type of organization.

							
              \item[Other Comments:] This work cross-cuts the museum and is relevant to professionals at all levels, from all departments: leadership, curatorial, administrative, education etc. It can even be applied to non-museum organizations as well.
              \end{description}
            \subsection*{Participants}
              \subsubsection*{ Aaron McCanna }
              Submitter, Presenter\newline
              Diversity, Equity, Access, Inclusion and Decolonization Coordinator\newline
              Burke Museum of Natural History and Culture, Seattle, WA
              \newline
              amccanna@uw.edu\newline
              
              

              I will be one of the individuals interviewed about the Burke Museum's projects, initiatives and strategies relating to decolonization, as I am one of the people who's job it is to oversee, plan and facilitate this work. My approach to this work is also different due to my scientific background, so can be useful to other organizations that might be newer to decolonization.\newline


              
                \subsubsection*{ Polly Olsen }
                Moderator, Presenter\newline
                Tribal Liaison and Director of Diversity, Equity, Access, Inclusion and decolonization/decolonizing initiatives\newline
                Burke Museum of Natural History and Culture, Seattle, WA
                \newline
                polly@uw.edu\newline
                
                206.543.5946\newline

                Polly Olsen leads the Burke Museum’s efforts in decolonization and has made a career out of navigating cultural consultation in a way that recognizes and benefits communities outside the organization. In her time at the Burke she has been a strong proponent of inclusion and has successfully raised the status of the Burke Museum’s Native American Advisory Board to one of reciprocal benefit and collaboration.\newline
                \emph{ (confirmed) }
              

              
                \subsubsection*{ Micah Parzen }
                Presenter\newline
                CEO\newline
                Museum of Us, San Diego, CA
                \newline
                mparzen@museumofus.org\newline
                
                

                As the CEO of the Museum of Us, Micah has lead a museum that has put decolonizing initiatives at the forefront of their work. His experience from a leadership point of view is valuable, and can model potential opportunities for other museum leaders.
                \emph{ (confirmed) }
              

              
                \subsubsection*{ Brandie McDonald }
                Presenter\newline
                Director of Decolonizing Initiatives\newline
                Museum of Us, San Diego, CA
                \newline
                bmacdonald@museumofus.org\newline
                
                

                Brandie McDonald’s strong connections Tribal communities and her experience in director-level decolonizing initiatives at the Museum of Us enable her to put decolonizing initiatives into action in a way that is effective, innovative, and feasible. The museum field can learn from initiatives she has created and the way in which they were developed.
                \emph{ (confirmed) }
              

              

              
        
          \newpage
          \section{ Virtual Programming:  A Whole New World of IP! }
            \begin{description}
              \item [ID:]
              WMA2022\_244

              \item [Assigned to:]
                \item [Track:]
              \end{description}

              <strong>Virtual programming is here to stay, and with it comes a whole new world of IP creation and utilization! Hear about why virtual programming is in the future of museums, get inspired by real-world examples, find out what really is the metaverse, and get primed on IP basics to steer clear of legal pitfalls.  Turn your camera and microphone on and get up to speed on best practices for virtual programming!</strong>

              \subsection*{Session Information}
                \begin{description}
                  \item [Format:] Regular session/panel (roundtable, single speaker, etc.)
							    
								  \item [Fee:]NA
							     
							    \item [Uniqueness:]This session is unique because it will address best practices with IP in addition to discussing different types of virtual programming, and why it is here to stay.
							    \item [Objectives:]Participants will:
·      Get inspired by examples of innovative virtual programming done at other museums.
·      Learn what is the metaverse and how it relates to virtual programming.
·      The benefits of various live-streaming platforms.
·      Understand why virtual programming are in museums’ futures.
  

Get acquainted with IP basics to avoid common legal pitfalls when creating virtual programming.
							    \item [Engagement:]The subject matter of this session presents a unique opportunity to engage with the audience by inviting participants to share their experiences and successes with virtual programming as part of the moderated discussion.
							    \item [Relationship to Theme:]This workshop relates to the theme of FORWARD by presenting on virtual programming, a new yet entirely necessary format that exploded because of the COVID pandemic.  This session will specifically discuss why virtual programming is the future.  By showcasing innovative examples of virtual programming, participants will become inspired to develop creative virtual programming in their museums.
							    
                \end{description}
              \subsection*{Audience}
                \begin{description}
                  \item [Audiences:]General Audience~
                  \item[Professional Level:]General Audience~
                \item[Scalability:]   The portion of the session where the benefits of various livestreaming platforms are discussed will be of interest to museums large and small to help them narrow down the best platforms for their size and audience. The portion on IP best practices will be especially useful for small museums with limited legal resources.
  

							
              \end{description}
            \subsection*{Participants}
              \subsubsection*{ Barron Oda }
              Submitter, Moderator, Presenter\newline
              General Counsel\newline
              Bishop Museum, Honolulu, HI
              \newline
              barron.oda@bishopmuseum.org\newline
              
              (808) 387-0467\newline

              Barron Oda has been chosen for his background in museum administration and experience with museum law. Barron has previously presented at the American Alliance of Museums’ 2016 Annual Meeting and two AAM webinars, at the ABA’s 2017 Annual Meeting on museum law, and at the 2018 WMA Annual Meeting (both a session and a workshop). Barron has also been a guest lecturer for Harvard Extension School’s graduate museology program and has published museum-specific articles. His practice areas include art law, museum law, cultural property, intellectual property, and nonprofit governance. Barron is admitted to practice law in Washington and Hawaii.\newline


              

              
                \subsubsection*{ Michelle  Pham }
                Presenter\newline
                Partner\newline
                Helsell Fetterman LLP, Seattle, WA
                \newline
                mpham@helsell.com\newline
                
                (206) 689-2139\newline

                Michelle Q. Pham has been chosen for her background in museum administration, museum law, art law, intellectual property, and litigation affecting nonprofits, businesses, and probate matters. She is former counsel to the Seattle Art Museum where she helped revise the museum’s policies, assessed its intellectual property licensing program, dealt with employment matters, among others. She is also a volunteer attorney for Washington Lawyers for the Arts and a member of the American Bar Association’s Museums and the Arts Law Committee. Michelle is a past presenter of a session and a workshop at the Western Museums Association’s 2018 Annual Meeting. Michelle is admitted to practice law in Washington and Texas
                \emph{ (confirmed) }
              

              

              

              
        
          \newpage
          \section{ Leveraging Major Gifts to Support Transformational Campaigns }
            \begin{description}
              \item [ID:]
              WMA2022\_245

              \item [Assigned to:]
                \item [Track:]Other~
              \end{description}

              Many museums are beginning to reemerge in a new, hybrid world of fundraising and philanthropy. While some strategies and tactics have shifted, the heart of successful campaigns remains the same: major gifts. Major gifts propel capital campaigns to transformational heights while also inspiring a strong culture of philanthropy. Join us to hear from two museums about the cultivation and stewardship strategies that helped them secure largest-ever gifts in support of their transformational campaigns.

              \subsection*{Session Information}
                \begin{description}
                  \item [Format:] Regular session/panel (roundtable, single speaker, etc.)
							    
								  \item [Fee:]0
							     
							    \item [Uniqueness:]As museums reemerge from the pandemic, exploring how to secure major gifts in a hybrid world is critical to the success and impact of museums.
							    \item [Objectives:]After the session, attendees will..
 
1. Understand how and when to build a major gifts strategy.
2. Know how to pursue major gifts through thoughtful, creative cultivation strategies in a hybrid fundraising environment.
3. Walk away with successful examples for securing transformational philanthropic gifts that they will be able to adjust and/or replicate with their own donors.
							    \item [Engagement:]Host an audience Q\&A after the panel; we might build live polls into the presentation depending on content needs
Resources might include: microphones for Q\&A, live polling software (we could provide)
							    \item [Relationship to Theme:]Philanthropy is a fundamentally forward-looking pursuit. Our session will encourage attendees to align their fundraising goals to their organization's vision and future impact. For volunteer leaders or board members, this session provides a window into fundraising strategies that might benefit their organization and prompts them to consider what role they might play in strengthening a culture of philanthropy at their organization.
							    
                \end{description}
              \subsection*{Audience}
                \begin{description}
                  \item [Audiences:]Board Members~Development and Membership Officers~Directors/Executive/C-Suite~
                  \item[Professional Level:]Mid-Career~Museum board members~Senior Level~
                \item[Scalability:] Organizations of all sizes can engage in capital campaigns and secure major gifts at a scale that is appropriate to their goals. Our presenters would encourage all organizations to build a culture of philanthropy to either support current or upcoming campaigns or lay a foundation for future campaign efforts.

							
              \end{description}
            \subsection*{Participants}
              \subsubsection*{ Courtney Martin }
              Submitter\newline
              Director of Marketing\newline
              CCS Fundraising, Seattle, WA
              \newline
              CMartin@ccsfundraising.com\newline
              
              8157662070\newline

              


              
                \subsubsection*{ Frederic J. 'Rick' Happy }
                Moderator, Presenter\newline
                Principal and Managing Director\newline
                CCS Fundraising, San Francisco, CA
                \newline
                FJHappy@ccsfundraising.com\newline
                ccssanfrancisco@ccsfundraising.com\newline
                4153925395\newline

                Rick Happy brings more than 35 years of fundraising and campaign management experience to the panel conversation. Rick is leading the work with High Desert Museum and will help drive a nuanced conversation around the major gift strategies and campaign management. Frederic J. “Rick” Happy is a Principal \& Managing Director of CCS. Rick has directed over 100 capital campaigns, which have raised a combined total exceeding \$2 billion. His select client experience in the arts and culture sector includes High Desert Museum, Bay Area Discovery Museum, Crocker Art Museum, Denver Center for the Perfuming Arts, and the Utah Symphony.\newline
                \emph{ (confirmed) }
              

              
                \subsubsection*{ Karie Burch }
                Presenter\newline
                Director of Development\newline
                Portland Art Museum, Portland, OR
                \newline
                Karie.Burch@pam.org\newline
                
                5032764240\newline

                The Portland Art Museum (PAM) has secured more than \$80M in support of their historic \$100M campaign. The success of the campaign is driven by major gifts, which Karie played a significant role in cultivating and securing. Karie has served the Portland Art Museum for more than 15 years in various capacities, with four years as the museum’s Director of Development. Her scope of responsibilities at the Museum have included major gifts, corporate development, planned giving, special events, and, most notably, managing and growing the Patron Society program into a key pillar of support for the Museum’s community mission.
                \emph{ (confirmed) }
              

              
                \subsubsection*{ Dana Whitelaw }
                Presenter\newline
                Executive Director\newline
                High Desert Museum, Bend, OR
                \newline
                DWhitelaw@highdesertmuseum.org\newline
                
                5413824754\newline

                The High Desert Museum is conducting a \$40M capital expansion project to renovate indoor areas, reimagine space for its Native American exhibit, and design an immersive Forest Canopy Walk. Dana has played a crucial role in setting the vision, pace, and success of the campaign to-date. Dana has been at the High Desert Museum for more than 13 years and the Executive Director for five. Under Dana’s leadership, the High Desert Museum has become a Smithsonian Affiliate, experienced record attendance and was a 2018 finalist for the prestigious National Medal for Museum and Library Service.
                \emph{ (confirmed) }
              

              
                \subsubsection*{ Katlyn Torgerson }
                Presenter\newline
                Senior Vice President\newline
                CCS Fundraising, Seattle, WA
                \newline
                KTorgerson@ccsfundraising.com\newline
                
                2068768523\newline

                Katlyn Torgerson is working with Portland Art Museum to help manage their historic \$100M campaign in support of access, exhibitions, and programing at the Museum. Katlyn will join Rick as a co-moderator of the panel. In a decade at CCS, Katlyn has provided strategic counsel to CCS clients across nonprofit sectors, including education, healthcare, foundations, human services, and the arts. During her tenure, she has partnered with organizations to raise over \$150 million in capital and programmatic funding. Her select client experience includes Portland Art Museum, High Desert Museum, UCSF Benioff Children’s Hospital, PeaceHealth, Literary Arts, and World Forestry Center.
                \emph{ (confirmed) }
              

              
        
          \newpage
          \section{ Preserving History Through Restoration: Making difficult choices with collections }
            \begin{description}
              \item [ID:]
              WMA2022\_246

              \item [Assigned to:]
                \item [Track:]Collections~
              \end{description}

              Traditional museum conservation fails to address some of the challenges and opportunities inherent in our institutions' outdoor, macro, and  functional artifacts. Whether a lightship, a submarine, a submersible, or your artifact, they sometimes withstand non-standard museum display environments and practices to keep them relevant. Join us for a discussion about preserving historical and educational significance through atypical preservation projects.

              \subsection*{Session Information}
                \begin{description}
                  \item [Format:] Regular session/panel (roundtable, single speaker, etc.)
							    
							    \item [Uniqueness:]Highlighting our different projects and solutions, we will help others tackle the mindset of these use vs preservation challenges.
							    \item [Objectives:]- A recognition that many preservation projects require thinking outside the conservation box.
- Understanding that there is a ratio of value in consumption to use, and sometimes we have to consider how our projects not only preserve the object, but also preserve the history we are trying to share.
							    \item [Engagement:]We will have a Powerpoint presentation, and intend that audience participation will be through Q\&A, including prompting audience members to share similar projects they have initiated.
							    \item [Relationship to Theme:]The session if FORWARD. We are looking at different approaches to caring for our collections, and accepting that restoration has a role in the preservation of history.
							    
                \end{description}
              \subsection*{Audience}
                \begin{description}
                  \item [Audiences:]Curators/Scientists/Historians~Directors/Executive/C-Suite~Facilities Management Personnel~General Audience~Registrars, Collections Managers~
                  \item[Professional Level:]All professional levels~
                \item[Scalability:] The learning outcomes are hugely scalable as we will discuss realistic approaches to preserving our collections items that fall outside of collections care standards. It does not matter what size institution, we all have to look at what we should do, can do, and want to do, and make the best choices for our objects.	

							
              \end{description}
            \subsection*{Participants}
              \subsubsection*{ Beth Sanders }
              Submitter, Moderator, Presenter\newline
              Collections Manager\newline
              U.S. Naval Undersea Museum, Keyport, WA
              \newline
              beth.a.sanders5.civ@us.navy.mil\newline
              sanders.bethann@gmail.com\newline
              360-315-1179\newline

              Having overseen the restoration of DSRV-1 Mystic, Beth has tackled many of these questions of restoration vs conservation. It has given her the opportunity to think outside of the collections management box, recognizing that she is not a ship repair specialist, and emphasizing the need to work with professionals who can best care for the museum's submersible.\newline


              
                \subsubsection*{ Beth Sanders }
                Moderator, Presenter\newline
                Collections Manager\newline
                U.S. Naval Undersea Museum, Keyport, WA
                \newline
                beth.a.sanders5.civ@us.navy.mil\newline
                sanders.bethann@gmail.com\newline
                3603151179\newline

                Having overseen the restoration of DSRV-1 Mystic, Beth has tackled many of these questions of restoration vs. conservation. It has given her the opportunity to think outside of the collections management box, recognizing that she is not a ship repair specialist, and emphasizing the need to work with professionals who can best care for the museum's submersible.\newline
                \emph{ (confirmed) }
              

              
                \subsubsection*{ Richard Pekelney }
                Presenter\newline
                Co-Chair, USS Pampanito\newline
                San Francisco Maritime National Park Association Board of Trustees, San Francisco, CA
                \newline
                pekelney@rspeng.com\newline
                
                415-812-5928\newline

                Rich's work with USS Pampanito has emphasized the importance of completeness and functionality of historic items for learning. From supporting efforts to keeping the boat afloat, to studying and restoring items on board, Rich emphasizes the value of experience and the importance of the real thing in engaging with history.
                \emph{ (confirmed) }
              

              
                \subsubsection*{ Bruce Jones }
                Presenter\newline
                Deputy Director\newline
                Columbia River Maritime Museum, Astoria, OR
                \newline
                jones@crmm.org\newline
                
                503-741-5914\newline

                Bruce has been the guiding light for the preservation and restoration work on Lightship Columbia, the museum's largest exhibit. This project supports continued long-term engagement with and on the boat, while reinforcing the artifact's structural stability. This experience has highlighted the unique challenges that arise in large restoration and preservation projects.
                \emph{ (confirmed) }
              

              

              
        
          \newpage
          \section{ Mentorship: Best Practices and Indigenous Notions of Care }
            \begin{description}
              \item [ID:]
              WMA2022\_247

              \item [Assigned to:]
                \item [Track:]
              \end{description}

              **This session will highlight the mentorship experiences of four indigenous museum professions who have at various times served as mentors and mentees. What makes a good mentor? What qualities should they have? And when it comes to indigenous staff, are different notions of care involved? Panelists span the gamut from a recent Mellon fellow to a 20+ year museum professional, from a graduate student to the director of a renowned Native American Fellowship program. **

              \subsection*{Session Information}
                \begin{description}
                  \item [Format:] Regular session/panel (roundtable, single speaker, etc.)
							    
							    \item [Uniqueness:]Few sessions, if any, have explored mentorship through an indigenous lens - while still distilling best practices that are applicable to all.
							    \item [Objectives:]Objectives of this session are to (1) share the mentorship experiences of four indigenous museum folks; (2) sharing specific mentorship program information, as applicable; (3) identify what qualities make a good mentor; (4) discuss indigenous notions of care and mentorship; (5) distill what best practices in mentorship look like that can be applied beyond an indigenous context.
							    \item [Engagement:]Audience members will learn about mentorship through an indigenous lens, as experienced by actual mentor/mentee relationships between the panelists. This will not be a standard panel but a moderated one with questions posed throughout, followed by audience Q\&A.
							    \item [Relationship to Theme:]This session looks "Forward" by examining how we can best grow and nurture the next generation of indigenous museum staff members, and by extension, all staff members.
							    
                    \item [Additional Comments: ]I am awaiting confirmation from one of the presenters. If she falls through, I will be looking for an additional panelist. 

                \end{description}
              \subsection*{Audience}
                \begin{description}
                  \item [Audiences:]Diversity and inclusion specialists~Emerging Museum Professionals~General Audience~
                  \item[Professional Level:]All professional levels~Mid-Career~Student~
                \item[Scalability:] All institutions have mentorship opportunities - opportunities which can be strengthened and expanded, benefiting the individual and the organization.

							
              \end{description}
            \subsection*{Participants}
              \subsubsection*{ Noelle Kahanu }
              Submitter, Moderator\newline
              Associate Specialist, Public Humanities and Native Hawaiian Programs\newline
              University of Hawaii at Manoa, Honolulu, HI
              \newline
              nmkahanu@hawaii.edu\newline
              mooinanea22@gmail.com\newline
              (808) 375-9125\newline

              Noelle has been a frequent presenter and moderator at WMA, with more than 20 years of experience in the museum field. She is connected in one way or another to all the panelists and seeks to deeply engage in an important conversation around mentorship and indigeneity.\newline


              
                \subsubsection*{ Noelle Kahanu }
                Moderator\newline
                Associate Specialist, Public Humanities and Native Hawaiian Programs\newline
                University of Hawai'i at Manoa, Honolulu, Hawai'i
                \newline
                nmkahanu@hawaii.edu\newline
                mooinanea22@gmail.com\newline
                (808) 375-9125\newline

                Noelle has been a frequent presenter and moderator at WMA, with more than 20 years of experience in the museum field. She is connected in one way or another to all the panelists and seeks to deeply engage in an important conversation around mentorship and indigeneity.\newline
                \emph{ (confirmed) }
              

              
                \subsubsection*{ Kamalu  du Preez }
                Presenter\newline
                Cultural Resource Specialist\newline
                Bishop Museum, Honolulu, HI
                \newline
                kamalu@bishopmuseum.org\newline
                mooinanea22@gmail.com\newline
                (808) 847-8267\newline

                Kamalu du Preez (kanaka 'oiwi) has over 20 years of experience working at
Bishop Museum. She has mentored numerous individuals, both native and not, over the years. She will be sharing her mentorship practices, which extent to both panelist Halena-Kapuni Reynolds and Hattie Hapai.
                \emph{ (confirmed) }
              

              
                \subsubsection*{ Halena  Kapuni-Reynolds }
                Presenter\newline
                Graduate Assistant, Museum Studies Graduate Certificate Program\newline
                University of Hawai'i at Manoa, Hilo, HI
                \newline
                halena@hawaii.edu\newline
                
                (808) 956-8570\newline

                Halena Kapuni-Reynolds is an emerging indigenous museum professional and current graduate student. He was a PEM Native American Fellow who has also interned with several museums, including the Bishop Museum (under Kamalu du Preez), Lyman Museum in Hilo, and the Denver Museum of Natural History.  He also ran a summer institute for several years, serving as a mentor to dozens of Native Hawaiian students.
                \emph{ (confirmed) }
              

              
                \subsubsection*{ Hattie Hapai }
                Presenter\newline
                Mellon Fellow\newline
                Peabody Essex Museum, Salem, MA
                \newline
                hattie_hapai@pem.org\newline
                hattiehapai.17@gmail.com\newline
                (808) 222-0053\newline

                Hattie will share her experiences, having been mentored by two of the panelists. After an internship at Bishop Museum under Kamalu du Preez, Hattie participated in PEM's Native American Fellowship Program under Jennifer Himmelreich. She will share how her experiences with these mentors altered her goals and set her on a path in the museum field.
                \emph{ (confirmed) }
              

              
                \subsubsection*{ Jennifer  Himmelreich }
                Presenter\newline
                Native American Fellowship Program Manager\newline
                Peabody Essex Museum, Salem, MA
                \newline
                nafellowship@pem.org\newline
                
                866-745-1876\newline

                As Program Manager, Jennifer Himmelreich draws on her own experiences as an Alumna of the 2011 Native American Fellowship program at PEM and other fellowships. Her decade-long professional work with tribal museums, cultural heritage institutions, and mainstream oral history and mapping projects, helps guide current Fellows, Alumni, and future participants through each stage of the program while implementing its day-to-day operations. She is currently a mentor to Hattie Hapai.
                \emph{ (not confirmed) }
              
        
          \newpage
          \section{ Shared Leadership: Models for Success }
            \begin{description}
              \item [ID:]
              WMA2022\_248

              \item [Assigned to:]
                \item [Track:]Leadership/Careerpath~
              \end{description}

              Come hear from three organizations in various stages of implementing shared leadership models: Five Oaks Museum, San Diego Museum of Natural History, and Nesika Wilamut (Portland area non-profit). They will discuss successes, pitfalls, and new ways of approaching leadership in non-profits. Bring your questions, as there will time at the end of the session for Q\&amp;A and group discussion.

              \subsection*{Session Information}
                \begin{description}
                  \item [Format:] Regular session/panel (roundtable, single speaker, etc.)
							    
								  \item [Fee:]No
							     
							    \item [Uniqueness:]Shared leadership models are gaining interest in the museum field, but there are relatively few examples. This session brings 3 organizations with share leadership models to the stage for discussion and Q&A.
							    \item [Objectives:]Attendees will hear about the successes and pitfalls in implementing shared leadership and have the opportunity to ask questions of the panelists.
							    \item [Engagement:]To be developed over the summer
							    \item [Relationship to Theme:]To move FORWARD, we need to rethink leadership models in museums
							    
                \end{description}
              \subsection*{Audience}
                \begin{description}
                  \item [Audiences:]Directors/Executive/C-Suite~General Audience~
                  \item[Professional Level:]General Audience~
                \item[Scalability:] Panelists represent a large and small museum, and a small indigenous lead non-profit

							
              \end{description}
            \subsection*{Participants}
              \subsubsection*{ Jason B. Jones }
              Submitter\newline
              Executive Director\newline
              Western Museums Association, Tulsa
              \newline
              jason@westmuse.org\newline
              
              

              


              

              
                \subsubsection*{   Ariel  Hammond  }
                Presenter\newline
                Library Director \& Curator\newline
                San Diego Museum of Natural History, San Diego, CA
                \newline
                ahammond@sdnhm.org\newline
                
                

                At SDNHM we're restructuring our management group into an internal Advisory Council, with nominated members representing their departments regardless of rank. We think this will improve communication, transparency, and inclusion (areas we've scored poorly on in staff surveys), as well as innovation and professional development. I'm heading the transition, and we hope to officially launch within the next month or two.
                \emph{ (confirmed) }
              

              
                \subsubsection*{ Molly  Alloy  }
                Presenter\newline
                Co-Director\newline
                Five Oaks Museum, Portland, OR
                \newline
                molly@fiveoaksmuseum.org\newline
                
                

                Co-Director at a museum
                \emph{ (confirmed) }
              

              
                \subsubsection*{ Nathanael Andreini }
                Presenter\newline
                Co-Director\newline
                Five Oaks Museum, Portland, OR
                \newline
                nat@fiveoaksmuseum.org\newline
                
                

                Co-Director at a museum
                \emph{ (confirmed) }
              

              
                \subsubsection*{ Tana  Atchley Culbertson }
                Presenter\newline
                Director of Network Coordination\newline
                Nesika Wilamut , Willamette, OR
                \newline
                tana@nesikawilamut.org\newline
                
                

                Co-director of a non-profit
                \emph{ (confirmed) }
              
    \newpage
    \chapter*{ Half-day workshop (9:00 a.m. – 1:00 p.m.) }

      
        
        
        
          \newpage
          \section{ Dabbling in the Data - Hands-On Data Analysis for Teams }
            \begin{description}
              \item [ID:]
              WMA2022\_172

              \item [Assigned to:]
                \item [Track:]
              \end{description}

              Data can help museums and cultural institutions target their services and tell their story.  However, the prospect of collecting and analyzing data can be intimidating and frustrating, especially for those with limited training in program evaluation. 
In this session, participants will learn interactive, participatory activities to engage teams with their data, yielding actionable insights and powerful stories. Participants will have the opportunity to practice activities and learn how to link the methods to specific learning needs in their organizations.

              \subsection*{Session Information}
                \begin{description}
                  \item [Format:] Half-day workshop (9:00 a.m. – 1:00 p.m.)
							    
								  \item [Fee:]N/A
							     
							    \item [Uniqueness:]It’s as simple as adding 2+2! Dabbling in the Data turns data analysis from a dreaded chore to a mission-critical conversation focused on using data to enhance practice.
							    \item [Objectives:]- Participants will gain familiarity with participatory data analysis activities they can use in their museums or cultural institutions.
- Participants will learn how participatory data analysis supports more equitable and inclusive practice by engaging more people in interpreting information and recommending next steps.
- Participants will develop an implementation plan for their own organization, linking their learning goals to specific activities from the Dabbling in the Data guide.
							    \item [Engagement:]Session participants will spend nearly all of the session in guided hands-on practice using different data interpretation activities from the guide. Participants will receive a complimentary copy of the guide. Attendees will reflect on how they can use Dabbling activities in their own organizations to engage colleagues in meaningful, rigorous conversations about data and its implications for practice.
							    \item [Relationship to Theme:]When museums democratize data by opening the interpretation process, they move FORWARD toward their goals of being more inclusive and responsive to both their visitors and their staff.
							    
                    \item [Additional Comments: ]**Welcome and Introduction (30 minutes)**
Participants will do a warm-up activity and hear an overview of the session’s agenda and learning objectives. Participants will share current challenges in engaging teams with data, and strategies they have used to date. The presenters will offer a framework for a three-step method to make meaning of data (Scan, Diagnose, Prioritize). Participants will then learn an overview of how the participatory data analysis activities in the \_Dabbling in the Data\_ guide that can be used at each of the three steps. 
**Guided Practice (2 hours)**
Using sample data from a museum education program, session participants will follow the three-step method to analyze data (Scan, Diagnose, Prioritize). At each step participants will: 1) Learn participatory data-analysis activities that can be used for the step; 2) Practice one of the activities in-depth; and 3) Reflect on how their organization’s learning goals could be supported by one of these activities. 
Scan: Participants will start by getting a 10,000-foot view of their data landscape. These activities will help participants make better informed decisions about which aspects of their program deserve a deeper dive. 
Practice Activity: In Mind the Gap, participants will explore differences between two sets of sample data, mapping the data sets, creating dot plots to visualize differences between data sets and discussing what they see. This exercise allows users to explore systematic relationships between data sets, such as differences in visitor feedback among different exhibits. 
Diagnose: Next participants will figure out what’s driving the patterns they picked in the Scan phase. These activities will help participants get to the underlying reasons. 
Practice Activity: In Iceberg Analysis, participants will identify the factors that contribute to an observable phenomenon, such as increases in museum visitorship. Using the iceberg as an analogy, this exercise helps teams to identify what's "beneath the surface" of something they observe in their museum.
Prioritize: Finally, participants will use what they learned from the Diagnose phase to identify the highest value opportunities to make a lasting change. 
Practice Activity: In Magic Quadrant, participants will engage in a structured conversation activity to support group decision-making. Participants will generate potential steps and then categorize each step by the level of effort and impact. By categorizing potential solutions to reach a goal, participants will gain an understanding of the impact and related level of effort of each possible step.
Break (20 minutes) 
**Connecting to Organizational Values (30 minutes)**
Session participants will explore the ways in which participatory data interpretation can support their organization’s values, such as a great visitor experience, attracting and engaging a variety of audiences, and supporting the professional development of staff members. Participants will practice their “pitch” for more participatory approaches to data interpretation based on the connections they find.
**Closing and Reflection (30 minutes)**
Participants will reflect on how the activities can support their own work. Participants will use our Take-It-Back planner to identify a meaning-making opportunity in their organization and to develop an implementation plan.

                \end{description}
              \subsection*{Audience}
                \begin{description}
                  \item [Audiences:]Directors/Executive/C-Suite~Emerging Museum Professionals~Other~
                  \item[Professional Level:]Emerging Professional~Mid-Career~
                \item[Scalability:] The \_Dabbling in the Data\_ activities can be scaled to engage teams of all sizes by using report-back approaches that we will model in the hands-on practice component of the session.

							
              \item[Other Comments:] This session will be particularly useful for museum professionals who work with data or guide continuous improvement initiatives in their organization.
              \end{description}
            \subsection*{Participants}
              \subsubsection*{ Corey Newhouse }
              Submitter, Presenter\newline
              Founder \& Principal\newline
              Public Profit, Oakland, CA
              \newline
              corey@publicprofit.net\newline
              
              541.246.6850\newline

              Corey founded Public Profit to help changemakers use data to make better decisions and improve their practice. As the Founder and Principal of Public Profit, Corey Newhouse leads the team’s strategic direction, external relationships, and business development. Corey serves as an internal thought partner to project teams, assisting with the design of Public Profit’s engagements in evaluation, capacity building, and strategic program design. She is co-author of Public Profit’s Creative Ways to Solicit Stakeholder Feedback and Dabbling in the Data, and a contributor to Evaluation Failures: 22 Tales of Mistakes Made and Lessons Learned.\newline


              

              
                \subsubsection*{ Corey Newhouse }
                Presenter\newline
                Founder \& Principal\newline
                Public Profit, Oakland, CA
                \newline
                corey@publicprofit.net\newline
                
                541.246.6850\newline

                Corey founded Public Profit to help changemakers use data to make better decisions and improve their practice. As the Founder and Principal of Public Profit, Corey Newhouse leads the team’s strategic direction, external relationships, and business development. Corey serves as an internal thought partner to project teams, assisting with the design of Public Profit’s engagements in evaluation, capacity building, and strategic program design. She is co-author of Public Profit’s Creative Ways to Solicit Stakeholder Feedback and Dabbling in the Data, and a contributor to Evaluation Failures: 22 Tales of Mistakes Made and Lessons Learned.
                \emph{ (confirmed) }
              

              

              

              
        
          \newpage
          \section{ Inclusion for People with Disabilities – Creating Your Roadmap }
            \begin{description}
              \item [ID:]
              WMA2022\_176

              \item [Assigned to:]
                \item [Track:]
              \end{description}

              <strong>Looking to improve accessibility for people with disabilities at your museum? Learn how the Palo Alto Junior Museum and Zoo created a truly inclusive and accessible facility. Participate in activities to practice the process and define your roadmap by drafting your own access and inclusion plan. Topics will include disability rights, engaging users and advisors, access and inclusion strategies and plans, gaining institutional commitment, staff training, prototyping, and fundraising.</strong>

              \subsection*{Session Information}
                \begin{description}
                  \item [Format:] Half-day workshop (9:00 a.m. – 1:00 p.m.)
							    
								  \item [Fee:]no fee
							     
							    \item [Uniqueness:]Inclusion of people with disabilities is a social justice issue. We will guide museums to create a concrete roadmap with the goal of authentic inclusion.
							    \item [Objectives:]1.      Learn how to engage people – understand the history of the disability rights movement and its current standing as a social justice issue, learn about disability types and community, outreach, and advisors – with an outcome of authentic and meaningful inclusion
2.      Develop institutional commitment – put it in writing, create an access and inclusion plan, understand the role of staff training and staffing, discuss funding strategies – with an outcome of embedding access in all strategic and business practices
3.      Explore testing and improving ideas – try out prototyping and understand this is ongoing work to create engagement and diversity, learn from the summative evaluation findings from the comprehensive Junior Museum \& Zoo project and how to apply the lessons – with an outcome of a growth-mindset that serves your audience’s needs
							    \item [Engagement:]Participation activities will include small and large group discussions, partner interviews, small group brainstorming, quiet writing time with individuals using worksheets, prototype building and iteration, group share outs and a time for questions and answers. Resources will include craft materials.
							    \item [Relationship to Theme:]Inclusion is forward thinking as a social justice issue and for museum relevancy. Making museum more accessible not only engages people with disabilities but also people with diverse income levels, ages, genders and racial or cultural backgrounds. Disability does not discriminate and affects people across the demographic spectrum. Research shows museums must stay relevant and engage new audiences, and inclusion of the disabled community is a key strategy due to the dearth of accessible experiences.
							    
                    \item [Additional Comments: ]**Since 2010 the JMZ has been committed and working toward becoming a more accessible institution. Improvements have occurred over time with various projects and evaluation initiatives. **

**Please visit the [Palo Alto Junior Museum and Zoo’s website](mailto:https://www.paloaltozoo.org/Visit/Accessibilityhttps://www.paloaltozoo.org/Visit/Accessibility) if you would like to fully understand the accessibility offerings. The recently completed, 4.5 year IMLS-funded \_Access from the Ground Up\_ project included a broad array of accommodations throughout the newly rebuilt facility that support people with disabilities, and it transformed business practices at all levels. The process included extensive exhibit and accessibility support prototyping, special events for the disability community, comprehensive staff training, long-term engagement of an Accessibility Advisory Team, and in-depth project evaluation. The commitment to inclusion and accessibility innovation is ongoing. **

**This is a stand-alone session and due to its breadth will not merge with another session.**

                \end{description}
              \subsection*{Audience}
                \begin{description}
                  \item [Audiences:]General Audience~
                  \item[Professional Level:]All professional levels~
                \item[Scalability:]  The outcomes are scalable to an institution of any size or type because we will guide participants in how to create their own roadmap, even with limited access to funds or resources. Those with more means can create a different roadmap and strategies to achieve their defined outcomes. For participants that do not yet have leadership support for inclusion, they can affect change within their realm of responsibility. We will offer strategies for influencing leadership with the goal of institutional support.

							
              \item[Other Comments:] The intended audience should be interested in and ready to act within their museums to implement inclusion of people with disabilities.
              \end{description}
            \subsection*{Participants}
              \subsubsection*{ Tina  Keegan }
              Submitter, Presenter\newline
              Exhibits Director\newline
              Palo Alto Junior Museum \& Zoo, Palo Alto
              \newline
              tina.keegan@cityofpaloalto.org\newline
              tinacosby@gmail.com\newline
              650-329-2624\newline

              Tina Keegan was the project director for the IMLS-funded project and co-author of the grant proposal and project development. She started the accessibility initiative at the Junior Museum \& Zoo. As the Exhibit Director, she implemented design solutions to complex access problems. Throughout her 23 year career, she has been dedicated to accessible exhibit design but has learned that one department in a museum cannot solve access issues and inclusion must be a comprehensive approach.\newline


              

              
                \subsubsection*{ Lisa Eriksen }
                Presenter\newline
                Former JMZ Accessibility Coordinator/Principal\newline
                Formerly Palo Alto Junior Museum \& Zoo (2018-2022)/Lisa Eriksen Consulting, Palo Alto, CA/Oakland, CA
                \newline
                lisaeriksen@mac.com\newline
                
                510-219-5070\newline

                Lisa Eriksen was the Accessibility Coordinator of the Junior Museum \& Zoo IMLS-funded project, as well as the co-author of the grant proposal and project development. Her museum career has included various roles, including museum worker, professor of museum studies, and thought leader in the field. Lisa has a deep understanding of inclusion within the broader context of the current museum field and operational roles, as well as the nuances of the disability community. Lisa managed the JMZ Accessibility Advisory Team and many of the project initiatives.
                \emph{ (confirmed) }
              

              
                \subsubsection*{ Maia  Werner-Avidon }
                Presenter\newline
                Principal\newline
                MWA Insights, Alameda, CA
                \newline
                maia@mwainsights.com\newline
                
                510-207-8803\newline

                Maia Werner-Avidon, principal of MWA Insights, is an independent museum evaluator with over 15 years of museum research and evaluation experience. Maia managed a comprehensive summative evaluation of the Access from the Ground Up project, gathering in-depth feedback and data from staff, advisors, families with children with disabilities, and the general public. She will offer insight based on her findings on best practices for advisor engagement and staff training as well as share out impactful findings from the project.
                \emph{ (confirmed) }
              

              

              
        
          \newpage
          \section{ Insufficient Data??? How Google Forms can transform your museum processes }
            \begin{description}
              \item [ID:]
              WMA2022\_183

              \item [Assigned to:]
                \item [Track:]
              \end{description}

              Most of us don\&rsquo;t have the resources to buy data tracking or survey systems. But what if you could do the basics for free using Google Forms and Sheets? This workshop will introduce ways to use these tools to start collecting the data you need for your organization. Bring a laptop!

              \subsection*{Session Information}
                \begin{description}
                  \item [Format:] Half-day workshop (9:00 a.m. – 1:00 p.m.)
							    
								  \item [Fee:]I would like to charge a small fee. Maybe $10. This would help cover the time I will spend creating example tools for them to use in the workshop.
							     
							    \item [Uniqueness:]Google tools are powerful and many of our colleagues don't know how to implement them. Learning the basics will start them on their data journey.
							    \item [Objectives:]1. Participants will be able to create a Google Form to collect visitor, employee, or survey data and look at it through the basic tools.
2. Participants will be able to view, edit, and manage the Google Sheet created by their form.
3. Participants will be able to use the query system in Google Sheets to do basic data analysis.
							    \item [Engagement:]The workshop will have three parts: setting up a form, managing the spreadsheet, and prompting to analyze data. After each, the participants will work on their own form and spreadsheets, and/or use ones available from the instructor. At the start of the session, each participant will write down a type of data they want to collect. At the end they will work in small groups to brainstorm solutions with their new skills.
							    \item [Relationship to Theme:]Data is what drives successful organizations in our age. By learning these intro skills, the participants will be able to participate in forward-thinking approaches to decision making by using data at their own organizations.
							    
                    \item [Additional Comments: ]I am willing to merge with another workshop if that would help, or make this into a typical session and talk about how we use these tools. My preference is to keep it as a workshop, though, so people are learning hands-on skills rather than just hearing "how we do it."

                \end{description}
              \subsection*{Audience}
                \begin{description}
                  \item [Audiences:]Directors/Executive/C-Suite~Emerging Museum Professionals~Evaluators~General Audience~Technology~
                  \item[Professional Level:]All professional levels~
                \item[Scalability:] Although my intent is to help smaller museums that have less resources, these tools can be applied at any size organization for free. Considering we all need to use data more, and we are all strained on having enough resources for the tools, there's scalability. This is meant to be a introductory workshop for beginners at any professional level so that it's applicable to a number of situations.

							
              \item[Other Comments:] I encourage all participants to bring a laptop. It would be easiest to do the hands on portion on a computer. I can help people with tablets (and even phones!) but it's a lot slower and the hands-on skills are slightly different.
              \end{description}
            \subsection*{Participants}
              \subsubsection*{ Joaquin Ortiz }
              Submitter, Moderator, Presenter\newline
              Senior Director of Strategic Initiatives\newline
              Museum of Photographic Arts, San Diego, CA
              \newline
              ortiz@mopa.org\newline
              
              619-238-7559\newline

              Joaquin is a leader in the application of technology in museums, and the use of data for decision making. He has used the Google data tools for 10+ years and built many of the systems MOPA uses for data collection and analysis.\newline


              

              
                \subsubsection*{ Laura Brooks }
                Presenter\newline
                Visitor Engagement Coordinator\newline
                Museum of Photographic Arts, San Diego, CA
                \newline
                brooks@mopa.org\newline
                
                619-238-7559\newline

                Laura is the day-to-day coordinator for many of MOPA's Google data projects, including the tracking and reporting of our programmatic statistics and visitation. She has collaborated with Joaquin on improving the use of these tools for the last few years, has taken classes about business data practices.
                \emph{ (confirmed) }
              

              

              

              
        
          \newpage
          \section{ Breaking Barriers: A Cultural Accessibility Project }
            \begin{description}
              \item [ID:]
              WMA2022\_200

              \item [Assigned to:]
                \item [Track:]
              \end{description}

              Museums need to address disability and work to create exceedingly accessible spaces for people with disabilities. During this workshop, participants will learn how a state agency and non-profit teamed up to create accessibility training for cultural professionals across Utah. Participants will dive into some of the accessibility and accommodations content before working to identify deficiencies in their own spaces. Participants will leave having started an Accessibility Plan for their museum. We recommend 2+ participants/museum.

              \subsection*{Session Information}
                \begin{description}
                  \item [Format:] Half-day workshop (9:00 a.m. – 1:00 p.m.)
							    
							    \item [Uniqueness:]In the context of diversity, equity and inclusion training, disability education equips cultural institutions with best practices to be equitable for the broad diversity of patrons.
							    \item [Objectives:](1) Participants will learn disability definitions and models to better understand what disability encompasses and how to competently discuss disability within their organizations. (2) Participants will learn about universal design - along with other accessibility accommodations - in order to reimagine their programs and physical spaces beyond ADA compliance and towards equitable accessible practices. (3) Participants will  be supported in identifying deficiencies in their own institutions to create actionable steps toward a personalized accessibility plan.
							    \item [Engagement:]We will spend 1.5 hours diving into disabilities (history, language, etc.), accommodations, and scenario planning. During the latter, attendees will share how they’d handle different scenarios with patrons who are disabled, with an opportunity to learn from one another and engage in deep conversation. Attendees will then work in small groups to identify accessibility deficiencies at their museum and begin drafting an accessibility plan with support and guidance from the presenters.
							    \item [Relationship to Theme:]Diversity, equity, inclusion, and accessibility are top-of-mind for museums and will continue to be for years to come. This session addresses accessibility - how we think about it, how we talk about it, and how we address it in our museums. Attendees will leave this workshop with a draft of an accessibility plan that they and their museum can implement and build upon over time, making their space more relevant and welcoming to more community members.
							    
                \end{description}
              \subsection*{Audience}
                \begin{description}
                  \item [Audiences:]General Audience~
                  \item[Professional Level:]All levels~
                \item[Scalability:] Accessibility is relevant no matter the size or scope of the organization. Too often, museums consider only ADA compliance. This session will address different disabilities and accommodations for various disabilities, with the hopes that museums of any size can implement better policies. The accessibility plan attendees will begin during the workshop includes short, intermediate, and long-term goals based on the idea that not all organizations have the resources (staff or funds) to meet the same goals on the same timeline. Personalizing a plan to meet their needs is what we encourage, not only with cultural organizations here in Utah who have completed the training but also through this workshop. Finally, the workshop provides attendees the opportunity to meet staff from a Utah non-profit who can assist in future accessibility work so the learning and experience does not have to end when the workshop does.

							
              \end{description}
            \subsection*{Participants}
              \subsubsection*{ Michelle Mileham }
              Submitter, Moderator, Presenter\newline
              Project Manager\newline
              Utah Division of Arts \& Museums, Salt Lake City, UT
              \newline
              mmileham@utah.gov\newline
              
              

              Michelle is the Accessibility Coordinator for the Utah Division of Arts \& Museums and is involved in the facilitation of the Breaking Barriers training this workshop is based upon.\newline


              

              
                \subsubsection*{ Jason Bowcutt }
                Presenter\newline
                Community Programs \& Performing Arts Manager\newline
                Utah Division of Arts \& Museums, Salt Lake City, UT
                \newline
                
                
                

                Jason helped develop the Breaking Barriers training and is a facilitator of the program.
                \emph{ (confirmed) }
              

              
                \subsubsection*{ Gabriella Huggins }
                Presenter\newline
                Executive Director\newline
                Art Access, Salt Lake City, UT
                \newline
                gabriella@artaccessutah.org\newline
                
                

                Gabriella is the Executive Director of the non-profit who co-created and co-facilitates the Breaking Barriers training.
                \emph{ (confirmed) }
              

              
                \subsubsection*{ Stan Clawson }
                Presenter\newline
                Consultant\newline
                Art Access, Salt Lake City, UT
                \newline
                stan@artaccessutah.org\newline
                
                

                Stan has completed the Breaking Barriers training and brings the perspective of a person with a disability.
                \emph{ (confirmed) }
              

              
        
          \newpage
          \section{ Rethinking Lightings Role, Color, and Technology in Museums through Practice }
            \begin{description}
              \item [ID:]
              WMA2022\_208

              \item [Assigned to:]
                \item [Track:]
              \end{description}

              <strong>While you might not always draw a straight line from lighting to visitor experience, there are countless ways that that system influences the museum environment. Much of influence is changing with LEDs and new technology. To harness this, one must identify the role of lighting in the who, when, what, why, where, and how of your collection, mission, and visitors. This practical workshop will demonstrate, discuss, and posit the role of lighting in those answers.</strong>

              \subsection*{Session Information}
                \begin{description}
                  \item [Format:] Half-day workshop (9:00 a.m. – 1:00 p.m.)
							    
								  \item [Fee:]The only potential fee would be any costs associated with a potential use of a gallery space within a museum. ERCO Lighting would be happy to work with the conference committee to help offset those costs. Equipment costs would be covered by ERCO through donated material.
							     
							    \item [Uniqueness:]Lighting quality and technology has changed rapidly. By offering a unique demonstration opportunity beyond the what-ifs of a presentation, attendees will get a visual experience.
							    \item [Objectives:]1.)   Attendees will learn how LED light is different from traditional halogen sources and what that means for collection exhibition and conservation. This will be done from a technical standpoint as well as using a demonstration system.
2.)   The presentation will discuss and demonstrate the impact of lighting styles and color on perception. Attendee feedback will be incorporated through the use of anonymous, electronic voting/feedback software to provide a chance for audience participation.
3.)	With the evolving technology landscape around the lighting marketplace, it is easy to make an exhibit lighting system smarter and more operational friendly. This presentation will demonstrate and discuss ways this can be add both tangible and intangible value.
							    \item [Engagement:]Ideally this would occur at a museum or gallery space where a demonstration could occur using an existing exhibition or artwork. If selected, ERCO would provide all of the necessary lighting equipment (fixtures, track, etc.) as well as electronic feedback technology mentioned previously. The session would include time for verbal feedback as well as discussions. The presentation would not seek to provide a specific answer or right/wrong but rather a launch pad for each institution.
							    \item [Relationship to Theme:]Lighting in museums hasn’t undergone a significant rethink with the introduction of LEDs. The same metrics and approaches are implemented. But this source, and the technology imbedded therein, will let museums move forward and adapt to the rapidly changing visitor expectations.
							    
                    \item [Additional Comments: ]  **ERCO has done variations of these workshops with several museums directly, including MOCA, and would be happy to provide references as to the impact of seeing the lighting in person. We believe this sort of conversation is important due to the role of light in a museum and its impact of the artwork. While we do have an independent lighting designer/consultant as part of our panel, we would be happy to include a curator or visitor experience voice in the workshop.**
  

                \end{description}
              \subsection*{Audience}
                \begin{description}
                  \item [Audiences:]Curators/Scientists/Historians~Directors/Executive/C-Suite~Exhibit Designers, Installers, Fabricators~Facilities Management Personnel~Registrars, Collections Managers~Technology~
                  \item[Professional Level:]General Audience~Mid-Career~Senior Level~
                \item[Scalability:] The impact and issues surrounding lighting extend across all museums as they greatly impact visitor experience through better visual acuity, improved conservation, and reduced operational impacts.

							
              \end{description}
            \subsection*{Participants}
              \subsubsection*{ Richard Fisher }
              Submitter, Moderator, Presenter\newline
              National Manager, Culture Cluster\newline
              ERCO Lighting, Edison, NJ
              \newline
              r.fisher@erco.com\newline
              
              917-371-5461\newline

              Rich is taking on the role of coordinator and organizer for the presentation and has led this discussion and workshop with several museums directly over the past two years. With a background in lighting design and experience in the development of advanced lighting solutions, he is able to not only discuss the technical but also the artistic and subjective issues at the heart of this topic.\newline


              

              
                \subsubsection*{ Richard Fisher }
                Presenter\newline
                National Manager, Culture Cluster\newline
                ERCO Lighting, Edison, NJ
                \newline
                r.fisher@erco.com\newline
                
                917-371-5461\newline

                Rich is taking on the role of coordinator and organizer for the presentation and has led this discussion and workshop with several museums directly over the past two years. With a background in lighting design and experience in the development of advanced lighting solutions, he is able to not only discuss the technical but also the artistic and subjective issues at the heart of this topic.
                \emph{ (confirmed) }
              

              
                \subsubsection*{ Jason Edling }
                Presenter\newline
                Founding Principal\newline
                Niteo Lighting, Seattle, WA
                \newline
                jason@niteolighting.com\newline
                
                650-291-5437\newline

                Jason has spent the majority of his professional design career working on museum and art lighting projects. He has a collaborative design approach that brings technical savvy to project in order to deliver the desired visual experience.
                \emph{ (confirmed) }
              

              

              
        
          \newpage
          \section{ Moving Forward With Communities: Co-Creating More Meaningful and Equitable Experiences }
            \begin{description}
              \item [ID:]
              WMA2022\_214

              \item [Assigned to:]
                \item [Track:]
              \end{description}

              Museums face barriers as they consider how to \_move forward \_in the context of the dual pandemics of COVID and racism. Utilizing community-centered design strategies, museums can improve visitor satisfaction and operationalize commitments to inclusive practices. Join us for a practical and engaging experience as you deeply understand the barriers you’re facing, identify problems within your systems, explore root causes through empathy interviews, and reimagine the experiences, practices, and policies that will meaningfully engage visitors.

              \subsection*{Session Information}
                \begin{description}
                  \item [Format:] Half-day workshop (9:00 a.m. – 1:00 p.m.)
							    
								  \item [Fee:]No fee: we are local to Portland and believe this work is important for folks to understand and implement and want to make professional development accessible given the devastating financial impact Covid has had on the museum field
							     
							    \item [Uniqueness:]This workshop will help participants work with their communities rather than just for their communities. Participants will leave with tools to operationalize liberatory, community-centered design.
							    \item [Objectives:]- Participants will better understand the foundational mindsets and mental models of liberatory, community-centered, and regenerative systems change
- Participants will gain practical experiences with tools and processes that operationalize equity in their organization including but not limited to the following
    - Reckon with and formulate problems of practice related to their context; 
    - Understand why we should create space to hear the stories and lived experiences of the community; 
    - Learn and practice empathy interviews as a way to center and amplify these stories/experiences; 
    - Experience ways to analyze and incorporate empathy data; 
    - Engage in a root cause analysis to understand contextual problems more deeply; 
    - And co-create equitable and inclusive solutions to these root causes.
- Participants will also learn strategies and models of how to build capacity  in their organization and beyond for liberatory, community-centered change
							    \item [Engagement:]We believe participants learn by doing so all of our facilitation involves participants engaging in the actual tools and processes of community-centered design. The following are some ways participants will engage:
- Small groups: develop problems of practice
- Small group discussion: observe a model empathy interview with guiding discussion questions  
- Pairs: Practice empathy interviews with a partner 
- Team collaboration: develop a root cause analysis (poster paper and sticky notes)
- Partners and small groups: Reimagine current systems
							    \item [Relationship to Theme:]The practices and mindset in the workshop have everything to do with supporting museums and individuals in moving forward as they address social, political, cultural, environmental, and technological change. How can museums move forward without first understanding the problem they are trying to solve? How can museums enact change with communities instead of to communities? We believe our session will help participants answer these questions as they envision a more equitable, just, and liberatory future.
							    
                \end{description}
              \subsection*{Audience}
                \begin{description}
                  \item [Audiences:]Board Members~Curators/Scientists/Historians~Directors/Executive/C-Suite~Diversity and inclusion specialists~Educators~Emerging Museum Professionals~Events Planning~Exhibit Designers, Installers, Fabricators~Programs~Visitor Services~
                  \item[Professional Level:]All levels~
                \item[Scalability:] Since our workshop focuses on mindsets and processes, we believe these translate to almost any role, organization, and context. We have worked at a variety of levels up and down the educational organizations and have seen liberatory and community-centered design help from the state-wide initiatives all the way down to an individual trying to make change.

							
              \item[Other Comments:] This workshop will be highly engaging, utilizing multiple grouping strategies, so space to move and group would be appreciated. We believe organizations would benefit from having several members present to collaborate regarding their context but this is not a requirement. Even single members of an organization will be able to engage fully in the process!
              \end{description}
            \subsection*{Participants}
              \subsubsection*{ Daniel Ramirez, PhD }
              Submitter, Moderator, Presenter\newline
              Senior Improvement Advisor\newline
              Community Design Partners, Portland, Oregon
              \newline
              daniel@communitydesignpartners.com\newline
              
              5034288998\newline

              Daniel Luis Ramirez, PhD., is a first-generation Latinx educator who works to amplify community-centered, antiracist practices and policies. He believes that stories matter, we should embrace complexity through multiple ways of knowing and being and that all work is deeply relational. Daniel is committed to using strength-based collective processes to disrupt oppressive power structures and recenter those most impacted by systems. Daniel has worked at all levels of educational systems, including as executive director of a statewide initiative focused on supporting and diversifying the educator workforce, leading a regional high school improvement network, and as an instructional coach and teacher for 14 years.\newline


              
                \subsubsection*{ Mary Kay Cunningham }
                Moderator, Presenter\newline
                Visitor Experience Specialist\newline
                Dialogue, Portland, Oregon
                \newline
                marykay@visitordialogue.com\newline
                
                503-975-4020\newline

                For the last 30 years Mary Kay has served over 35 different cultural institutions or attractions in the diverse roles of consultant, manager, museum educator, volunteer coordinator, and docent. She founded Dialogue Consulting in 2001 to support institutions improving their visitor experience through inclusive and collaborative interpretive planning, programming, and professional development. Her passion for facilitating group learning that brings together staff, volunteers, communities, and other stakeholders to navigate institutional change is the hallmark of her work.Mary Kay is the author of The Interpreters Training Manual for Museums (AAM, 2004) and co-editor for Museum Education for Today’s Audiences (AAM, 2022).\newline
                \emph{ (confirmed) }
              

              
                \subsubsection*{ Julie Smith }
                Presenter\newline
                Founder and Senior Improvement Advisor\newline
                Community Design Partners, Portland, Oregon
                \newline
                julie@communitydesignpartners.com\newline
                
                503-201-7177\newline

                For nearly 30 years Julie has been serving as a connector of educators, government leaders, and community members, with the goal of shifting mindsets around improvement that are grounded in empathy, racial equity \&  focused on systemic change. Julie works with organizations to build capacity for change and believes that organizations can realize their goals when they build them with those they aim to serve. Julie has worked as a facilitator, advisor, and coach to a diverse range of projects including helping organizations implement anti-racist frameworks, community-based school solutions, educator effectiveness systems, and designing state educator networks around continuous improvement.
                \emph{ (confirmed) }
              

              

              

              
        
          \newpage
          \section{ Museums for Social Change: A Participatory Workshop }
            \begin{description}
              \item [ID:]
              WMA2022\_215

              \item [Assigned to:]
                \item [Track:]
              \end{description}

              World Forestry Center and Upswell invite conference attendees to participate in a workshop focused on how an exhibition prototype can serve as a platform for new approaches to audience engagement and inform the transformation of museums into sites for social change.
The workshop focuses on the World Forestry Center’s vision for the future and renovation of their museum in Portland’s Washington Park, and will provide a framework that participants can apply to their own projects.

              \subsection*{Session Information}
                \begin{description}
                  \item [Format:] Half-day workshop (9:00 a.m. – 1:00 p.m.)
							    
								  \item [Fee:]No fee. However, participants will need to pay for a $5.00 round trip ticket for public transportation to/from conference hotel. 
							     
							    \item [Uniqueness:]Participants will experience an example of how museums are repositioning themselves from historically education focused organizations into sites for social change.
							    \item [Objectives:]The main objective of this workshop is to provide participants with a framework for designing exhibition prototypes that can help test the feasibility of different experiential approaches to audience engagement. 

Through this workshop, participants will learn:
1. How a prototype exhibition can mitigate risk, build internal consensus, and encourage creative innovation;
2. How to test experiences that are relevant to visitors and inform organizational transformation;
3. How to holistically connect programming and exhibition concepts to other organizational initiatives, such as marketing and operations.
							    \item [Engagement:]The workshop will take place at World Forestry Center and provide all requisite resources, including a reading packet for participants prior to the event. 

The agenda will include a 1) walkthrough of a community engagement exhibition, 2) facilitated conversation about the experience and WFC’s vision for the future, and, 3) design charrette for participants to apply the community engagement prototype model to the development of a new exhibition or program at their home institution.
							    \item [Relationship to Theme:]The workshop is designed to help participants move their exhibition or programming concepts forward. It’s a welcoming and fun platform that offers techniques and outcomes related to driving momentum and change.
							    
                    \item [Additional Comments: ]World Forestry Center is easily accessible via public transportation from the conference hotel by the Max red or blue line. The Max stations are located just one block from the Hilton Hotel and directly across the street from World Forestry Center in Portland’s Washington Park. Estimated travel time in 15-20 minutes.** **

We do not need additional presenters at this time. 

                \end{description}
              \subsection*{Audience}
                \begin{description}
                  \item [Audiences:]Curators/Scientists/Historians~Educators~Evaluators~Exhibit Designers, Installers, Fabricators~General Audience~Programs~
                  \item[Professional Level:]All levels~
                \item[Scalability:] The workshop is applicable to organizations of all types and sizes, although the workshop case study will be specifically focused on an exhibition prototype for an institution in the midst of transformation.

							
              \end{description}
            \subsection*{Participants}
              \subsubsection*{ Timothy Hecox }
              Submitter, Moderator, Presenter\newline
              Director of Experience \newline
              World Forestry Center, Portland
              \newline
              tnhecox@worldforestry.org\newline
              tnhecox@worldforestry.org\newline
              5033126054\newline

              As Director of Experience at World Forestry Center, I will welcome participants to our museum, provide an orientation, and facilitate aspects of the workshop relevant to our vision for the future and our efforts to transform the museum into a site for social change.\newline


              

              
                \subsubsection*{ Armando Manalo }
                Presenter\newline
                Director of Strategy\newline
                Upswell, Portland
                \newline
                armando@hello-upswell.com\newline
                
                

                Armando is the Director of Strategy at Upswell and was one of the lead developers for the community engagement pop-up exhibit on display at WFC. Armando has lead the development of numerous museum projects throughout the US and Canada focused on driving social change.
                \emph{ (confirmed) }
              

              
                \subsubsection*{ Elise Cramer }
                Presenter\newline
                Designer\newline
                Upswell, Portland
                \newline
                elise@hello-upswell.com\newline
                
                

                Elise served as the head designer for the community engagement pop-up exhibit on display at WFC. Elise will discuss various strategies for eliciting community feedback and how that feedback will inform the development of future exhibits.
                \emph{ (confirmed) }
              

              

              
        
          \newpage
          \section{ Leading Museums through Grief, Loss, and Change }
            \begin{description}
              \item [ID:]
              WMA2022\_236

              \item [Assigned to:]
                \item [Track:]Leadership/Careerpath~
              \end{description}

              ** **
<strong>We are in a grief pandemic, in a grief-illiterate and grief-averse society. Grief has tangible and long-lasting impact on leaders and their teams. Having a knowledgeable and non-pathologizing understanding of grief helps leaders increase their capacity to heal their own grief and better support the healing of grievers on their teams and communities.</strong>
** **
<strong>****</strong>
<strong>NOTES: Grief affects nearly every aspect of family wellbeing and solvency for millions across the United States.</strong> The unexpected death of a loved one poses a dual threat, as it is both among the most common major life stressors, and the single worst lifetime experience, reported by Americans in national surveys.
<strong>Losing a loved one is not only a personal tragedy, but casts a long shadow that can extend for decades</strong> as it places surviving parents, children, siblings, and spouses at significant risk for impaired health, premature death, and underachievement.
<strong>Grief and bereavement are an emerging public health crisis, with an impact on millions of families throughout the nation.</strong> They shares a powerful intersectionality with multiple national public health emergencies, including COVID-19, overdose, homicide, and suicide. As such, bereavement plays a key gatekeeping role in determining whether we as a nation can turn the corner on these ongoing public health crises towards national recovery and wellbeing.
**Perhaps most concerning, our national life expectancy\&ndash;an index of overall population health\&ndash;has dropped by more than one full year. **This last happened nearly 80 years ago following the United States’ entry into World War 2.
<strong>The implications of these statistics are sobering: They not only indicate that many middle-aged people of child-bearing and child-rearing years are dying, but that many children and adolescents are losing their parents, grandparents, aunts, uncles, and mentors.</strong> Recurring bereavement under tragic and often-traumatic circumstances has now become a commonplace fact of life for many US residents. Further, COVID and our other spiking epidemics have set back progress in closing the racial health disparities gap by some 20 years.
<strong>Racial inequalities in grief and bereavement are magnified across the life course as Black Americans are more likely than White Americans to experience the death of children, spouses, siblings, and parents.</strong> Black Americans are three times as likely as White Americans to have two or more family members die by the time they reach the age of 30. Black children are three times as likely to lose a mother and more than twice as likely to lose a father by age 10 when compared to White children.
<strong>Source:</strong> Testimony by Joyal Mulheron, Executive Director, Evermore, June 24, 2021 Submitted to Subcommittee on Labor, Health and Human Services, Education and Related Agencies, U.S. Senate. Pertaining to Centers for Disease Control and Prevention, Office of Surveillance, Epidemiology, and Laboratory Services/Division of Behavioral Health
** **
** **

              \subsection*{Session Information}
                \begin{description}
                  \item [Format:] Half-day workshop (9:00 a.m. – 1:00 p.m.)
							    
								  \item [Fee:]2.5 HOUR WORKSHOP  MAX 48 PARTICIPANTS (6 groups of 8) MIN 20 PARTICIPANTS   Can be scaled up or down, depending on audience interest, space capacity  PARTICIPANT FEE: $85  FEE COVERS: Professional fee, planning, customized materials, travel expenses, communication, set up, execution, follow-up with individuals and organizations
							     
							    \item [Uniqueness:]We are in a grief pandemic, in a grief-illiterate and grief-averse society. Grief has tangible and long-lasting impacts on leaders and their teams. Having a non-pathologizing understanding of grief helps leaders increase their capacity to deal with their own grief and better support the grievers on their teams, in their donor pools, and in their communities.
							    \item [Objectives:]At the end of the workshop, participants will be able to:
 
Define grief beyond the response to loss of life
Understand the connection between change and grief
Identify how myths about grief can impede healing
Understand grief as the normal and natural response to change and loss
Identify sources of hope and learning stemming from loss
							    \item [Engagement:]INTENDED AUDIENCE: Museum professionals who are people managers, team leaders, executives, department heads
 
PARTICIPATION STRATEGIES: Large-group introduction to concepts, breakout sessions and share-outs on discoveries, insights, tools
 
RESOURCES NEEDED: Small ballroom or meeting room w capacity 60+ people; 6 round tables (8-person capacity each); 50 chairs; 6 large post-it pads + easels; screen, projection system, mic, laptop stand; tech support; water, coffee, snacks for attendees.
 
Accommodations for hearing or visually impaired members?
							    \item [Relationship to Theme:]Grief is a healthy, normal response to the loss of someone or something you love. It is not an illness or a problem to be solved. It has no timeline or expiration date. With attention and care, we metabolize and integrate it. We don’t get over it. 

Moving forward, leaders working with this non-pathologizing understanding of grief can support grieving people on their teams, in their donor pools, in their communities, heal rather than feel broken or wrong,
							    
                    \item [Additional Comments: ]No additional comments. No merging.

                \end{description}
              \subsection*{Audience}
                \begin{description}
                  \item [Audiences:]Development and Membership Officers~Directors/Executive/C-Suite~Diversity and inclusion specialists~Marketing \& Communications (Including Social Media)~
                  \item[Professional Level:]Mid-Career~Museum board members~Senior Level~
                \item[Scalability:] **This workshop can be scaled and customized to address workplace teams of any size, in organizations of all types. This workshop is relevant to anyone who has experienced change and loss.**
** **
** **
** **
** **
 

							
              \item[Other Comments:] Executives
Senior Leadership
Trustees
Department heads
Development professionals
Team leaders
People managers
              \end{description}
            \subsection*{Participants}
              \subsubsection*{ Robin Held }
              Submitter, Moderator, Presenter\newline
              Principal, Robin Held Grief Coach LLC\newline
              Robin Held Grief Coach LLC, Seattle
              \newline
              robin@robinheldgriefcoach.com\newline
              robin@robinheldgriefcoach.com\newline
              206-427-6965\newline

              ROBIN HELD is the principal of Robin Held Grief Coach LLC, which provides grief coaching, leadership training, and transition support for people grieving the loss of a beloved human or a great job, navigating a major life transition, feeling overwhelmed by COVID-19 bereavement overload, or experiencing las of safety and trust in an institution, self, or others. Her certifications include: Advanced Specialist, Grief Recovery Method, Grief Recovery Institute; Kessler Grief Educator; and CCAR Recovery Coach.
 
Website: https://robinheldgriefcoach.com/
 
Linkedin: https://www.linkedin.com/in/robin-held-0a093812/
 
IG:           https://www.instagram.com/robinheld.griefcoach/
 
Prior to founding Robin Held Grief Coach, Held served in leadership positions in the arts for more than 30 years, with such organizations as the University of Washington’s Henry Art Gallery, The Frye Art Museum, and Washington State University’s Jordan Schnitzer Museum of Art. She led and transformed organizations in times of dramatic change, including expansion, reinvention, closure, and COVID-19-related distress.\newline


              

              
                \subsubsection*{ Robin Held }
                Presenter\newline
                Robin Held\newline
                Prior to founding Robin Held Grief Coach, Held served in museum leadership positions for 20+ years, at the University of Washington’s Henry Art Gallery, The Frye Art Museum, and Washington State University’s Jordan Schnitzer Museum of Art. She led and transformed organizations in times of dramatic change, including expansion, reinvention, closure, and COVID-19-related distress., Seattle, WA
                \newline
                robin@robinheldgriefcoach.com\newline
                robin@robinheldgriefcoach.com\newline
                206-427-6965\newline

                ROBIN HELD is the principal of Robin Held Grief Coach LLC, which provides grief coaching, leadership training, and transition support for people grieving the loss of a beloved human or a great job, navigating a major life transition, feeling overwhelmed by COVID-19 bereavement overload, or experiencing las of safety and trust in an institution, self, or others. Her certifications include: Advanced Specialist, Grief Recovery Method, Grief Recovery Institute; Kessler Grief Educator; and CCAR Recovery Coach.
 
Website: https://robinheldgriefcoach.com/
 
Linkedin: https://www.linkedin.com/in/robin-held-0a093812/
 
IG:           https://www.instagram.com/robinheld.griefcoach/
 
Prior to founding Robin Held Grief Coach, Held served in leadership positions in the arts for more than 30 years, with such organizations as the University of Washington’s Henry Art Gallery, The Frye Art Museum, and Washington State University’s Jordan Schnitzer Museum of Art. She led and transformed organizations in times of dramatic change, including expansion, reinvention, closure, and COVID-19-related distress.
                \emph{ (confirmed) }
              

              

              

              
        
          \newpage
          \section{ Museum Administration LARPing }
            \begin{description}
              \item [ID:]
              WMA2022\_241

              \item [Assigned to:]
                \item [Track:]Other~
              \end{description}

              <strong>Where do you fit in at your museum? Connect with peers while tackling a variety of museum administrative issues in this workshop. Presenters will discuss museum structure then break into groups where members will create their own theoretical museum.  We’ll end with connection and reflection.</strong>

              \subsection*{Session Information}
                \begin{description}
                  \item [Format:] Half-day workshop (9:00 a.m. – 1:00 p.m.)
							    
								  \item [Fee:]0
							     
							    \item [Uniqueness:]This is a choose your own adventure style workshop to provide participants in a way to engage with Museum 101 level content in a different way.
							    \item [Objectives:]Provide participants with a forum to:
1.     Build community, connection, and networking opportunities
2.     Understand how museums are created
3.     Help build an understanding of where each participant fits in at their institution 
4.     Understand how different departments within a museum can work together with the caveat that it’s different at every institution
5.     Develop a deeper understanding of museum structure, when it works, and when it doesn’t
6.     Build empathy for other museum staff and colleagues and the work that they do.
							    \item [Engagement:]Participants with different areas of expertise will be broken into randomly selected groups of varying size and work together to create their conceptual museum in a format that is TBD (open to ideas).
							    \item [Relationship to Theme:]The workshop will lay a foundation of what current museum structure looks like and how it could look in the future. 

There is also the potential for Emerging Professionals to build relationships that can be utilized later in their careers
							    
                    \item [Additional Comments: ]Open to all mergers and suggestions

                \end{description}
              \subsection*{Audience}
                \begin{description}
                  \item [Audiences:]Emerging Museum Professionals~General Audience~
                  \item[Professional Level:]Emerging Professional~Mid-Career~Student~
                \item[Scalability:] Different sized groups representing different sized institutions and how they work internally 

							
              \end{description}
            \subsection*{Participants}
              \subsubsection*{ Doug  Jenzen }
              Submitter, Moderator, Presenter\newline
              VP, Engagement and Access\newline
              Skirball Cultural Center, LA, CA
              \newline
              djenzen@skirball.org\newline
              doug.jenzen@gmail.com\newline
              

              I'm open to acting in a moderator capacity or simply helping to organize this session.\newline


              
                \subsubsection*{ Doug Jenzen }
                Moderator\newline
                VP, Engagement and Access\newline
                Skirball Cultural Center, LA, CA
                \newline
                djenzen@skirball.org\newline
                
                805-406-9774\newline

                I have lead a similar workshop at CAM - "Exhibit Planning Shark Tank Style" and it was successful.\newline
                \emph{ (confirmed) }
              

              
                \subsubsection*{ Doug Jenzen }
                Presenter\newline
                Niki Stwart\newline
                Frank Lloyd Wright Foundation, Scottsdale, AZ
                \newline
                nstewart@franklloydwright.org\newline
                
                

                If interested, please see Niki's bio here: Niki Ciccotelli Stewart Named Frank Lloyd Wright Foundation Vice President and Chief Learning \& Engagement Officer - Frank Lloyd Wright Foundation
                \emph{ (confirmed) }
              

              
                \subsubsection*{ Peter Kukla }
                Presenter\newline
                Planetarium Manager\newline
                Eugene Science Center, Eugene, OR
                \newline
                peterj.kukla@gmail.com\newline
                
                

                Peter brings the voice of an Emerging Professional to this group.
                \emph{ (confirmed) }
              

              

              
    \newpage
    \chapter*{ Full-day workshop (9:00 a.m. – 4:00 p.m.) }

      
        
        
        
          \newpage
          \section{ New Kids on the Blockchain: Cryptocurrency and NFTs in Your Museum }
            \begin{description}
              \item [ID:]
              WMA2022\_242

              \item [Assigned to:]
                \item [Track:]Business~
              \end{description}

              <strong>Cryptocurrency and NFTs (Non-Fungible Tokens) are here to stay. Get a headstart on Web 3.0 and get savvy on how this applies to your museum! Learn about blockchain, the technology underpinning cryptocurrency and NFTs. Discover how NFTs can strengthen your collections as well as become a part of it. Interested in accepting crypto at your upcoming gala? You’ll hear about emerging good practices and how to steer clear of pitfalls. Move your museum FORWARD by getting up to speed on crypto and NFTs!</strong>

              \subsection*{Session Information}
                \begin{description}
                  \item [Format:] Full-day workshop (9:00 a.m. – 4:00 p.m.)
							    
								  \item [Fee:]There will be a $40 fee which will cover the cost of printed materials, refreshments for two breaks, and a networking lunch for participants.
							     
							    \item [Uniqueness:]Cryptocurrency and NFTs are here to stay, whether we like it or not. Museums that invest in learning about them now will be better prepared for the future.
							    \item [Objectives:]Web 3.0 will depend on blockchain technology. FORWARD-thinking museums will want to embrace blockchain (and cryptocurrency and NFTs) to be better positioned for the mainstream arrival of Web 3.0, just as museums who embraced social media and mobile technology were leaders in our profession through Web 2.0. Learning outcomes for participants include:
 
- Understanding Web 3.0 and how it differs from Web 2.0.
- Being able to describe what blockchain is and how it works.
- Gaining knowledge on what cryptocurrency is, how it began, and how it is created and circulated.
- Being apprised on emerging good practices for accepting cryptocurrency as donations.
- Understanding what Non-Fungible Tokens are, how they are similar to physical objects, and how they differ from other digital assets
- Discovering the myriad ways NFTs could be integrated not just into museum collections, but museum operations.
- Appreciating the advantages that early adoption of blockchain technology will bring to the museum.
- Grasping the importance of drafting policies for NFTs and cryptocurrency.
- Nuts and bolts: how to get started accepting cryptocurrency and how to mint your first NFT
- What’s a “Smart Contract”
- Environmental Impact
- Understanding NFT domains and how they relate to receiving donations.
							    \item [Engagement:]The workshop could include the following strategies to engage participants and ensure they receive maximum value from their attendance:
·      Attractive, easy-to-follow multimedia presentations
·      Interactive instruction with participants using tactile teaching aids (such as having participants construct a blockchain with Legos to understand how it works)
·      Working in small groups to discuss considerations for drafting a crypto and NFT policy
·      By days’ end, having participants set up their own digital wallets using their laptops or mobile devices.
							    \item [Relationship to Theme:]This workshop relates to the theme of FORWARD by presenting up-to-the-minute information on this emerging technological subject that is currently only loosely regulated and not widely understood.  Cryptocurrency and NFTs, as well as their underlying blockchain technology, will be an integral part of Web 3.0.  History shows that museums who embraced social media and mobile technology in the 2000’s were well-prepared for Web 2.0, and the same will likely occur with Web 3.0 for museums who prepare today.
							    
                \end{description}
              \subsection*{Audience}
                \begin{description}
                  \item [Audiences:]Technology~
                  \item[Professional Level:]General Audience~
                \item[Scalability:] **Museums deal in public trust and knowledge authentication no matter the size of the institution. Just like how the web itself applies to all museums large and small, so will blockchain in the near future. It is important for museum professionals from institutions of all sizes to understand the implications of this new technology.**

							
              \end{description}
            \subsection*{Participants}
              \subsubsection*{ Barron Oda }
              Submitter, Moderator, Presenter\newline
              General Counsel\newline
              Bishop Museum, Honolulu, HI
              \newline
              Barron.oda@bishopmuseum.org\newline
              
              (808) 387-0467\newline

              Barron Oda has been chosen for his background in museum administration and experience with museum law. Barron has previously presented at the American Alliance of Museums’ 2016 Annual Meeting and two AAM webinars, at the ABA’s 2017 Annual Meeting on museum law, and at the 2018 WMA Annual Meeting (both a session and a workshop). Barron has also been a guest lecturer for Harvard Extension School’s graduate museology program and has published museum-specific articles. His practice areas include art law, museum law, cultural property, intellectual property, and nonprofit governance. Barron is admitted to practice law in Washington and Hawaii.\newline


              

              
                \subsubsection*{ Nicholas  Griffith }
                Presenter\newline
                Digital Developer\newline
                Bishop Museum, Honolulu, HI
                \newline
                Nicholasg@bishopmuseum.org\newline
                
                808-397-6575\newline

                Nicholas is the digital developer for Bishop Museum, and works on the implementation and navigation of new blockchain solutions for cultural institutions and art galleries. Having worked with myriad businesses in most industries, he has common hurdles to innovation. The only thing that’s certain is that paradigms are shifting and in this environment, museums and galleries really need to understand the new playing field. 
 
Nicholas works directly on bridging the gap between the cultural sector, private galleries, and blockchain solutions, and is well-suited to present on new solutions to institutional issues common to us all.
                \emph{ (confirmed) }
              

              
                \subsubsection*{ Nathanael  Smith }
                Presenter\newline
                Digital Asset Manager\newline
                Bishop Museum, Honolulu, HI
                \newline
                nathanael.smith@bishopmuseum.org\newline
                
                808-282-0459\newline

                Nathanael is the Digital Asset Manager for the Bernice Pauahi Bishop Museum Library \& Archives where he oversees digitization and data management. A serial nerd, Nathanael is also a doctoral candidate in Philosophy at the University of Rochester and has launched a business mining and staking crypto.
                \emph{ (confirmed) }
              

              
                \subsubsection*{ Richard  Pyle }
                Presenter\newline
                Senior Curator of Ichthyology / Director of XCoRE\newline
                Bishop Museum, Honolulu, HI
                \newline
                deepreef@bishopmuseum.org\newline
                
                808-848-4115\newline

                Dr. Pyle has several decades of experience as a researcher and developing databases and other informatics systems for museum and research objectives, including active participation in international data standards groups. He is directly involved with several aspects of blockchain, cryptocurrency and related technologies, and is a member of the Science Advisory Committee for Moonjelly Academy (https://www.moonjelly.io), an organization establishing a Decentralized Autonomous Organization (DAO) to leverage blockchain, cryptocurrencies, NFTs and related technologies to support ocean conservation. His role in this workshop will be to explain how these Web3 technologies can benefit and support scientific research and conservation aspects of museums.
                \emph{ (confirmed) }
              

              \end{document}
